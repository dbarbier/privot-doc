% Copyright (c)  2005-2010 EDF-EADS-PHIMECA.
% Permission is granted to copy, distribute and/or modify this document
% under the terms of the GNU Free Documentation License, Version 1.2
% or any later version published by the Free Software Foundation;
% with no Invariant Sections, no Front-Cover Texts, and no Back-Cover
% Texts.  A copy of the license is included in the section entitled "GNU
% Free Documentation License".


\newpage\section{FFT}

The objective of this class is to implement the Fourier transformations (direct and inverse) using external tools such as KissFFT.
We describe here the methods of this class.

\begin{description}

\item[Usage :] \rule{0pt}{1em}
  \begin{description}
  \item $FFT()$
  \item
  \end{description}

\item[Value :] FFT
  \begin{description}
  \item This instantiates the Fast Fourier Transform (FFT) class
  \end{description}

\item[Some methods :]  \rule{0pt}{1em}
  \begin{description}
  \item $getName$
    \begin{description}
    \item[Usage :] $getName()$
    \item[Arguments :] none
    \item[Value :] a string, the name of the FFT
    \end{description}
    \bigskip

  \item $inverseTransform$
    \begin{description}
    \item[Usage :] $inverseTransform(collection)$
    \item[Arguments :] NumericalComplexCollection 
    \item[Value :] a NumericalComplexCollection. This computes the Fourier inverse transformation of the values contained in collection. 
    \end{description}
    \bigskip

  \item $setName$
    \begin{description}
    \item[Usage :] $setName(name)$
    \item[Arguments :] name : a string
    \item[Value :] the FFT is named $name$
    \end{description}
    \bigskip

  \item $transform$
    \begin{description}
    \item[Usage :] $transform(collection)$
    \item[Arguments :] NumericalComplexCollection 
    \item[Value :] a NumericalComplexCollection. This computes the Fourier transformation of the values contained in collection. Notice also that the collection might be a NumericalScalarCollection
    \end{description}
    \bigskip
  \end{description}
\end{description}


% ==========================================================================

\newpage\subsubsection{KissFFT}
The KissFFT class inherits from the FFT class. The methods are the same as the FFT class (there is no additional method).
This class interacts with the \emph{kissfft} implemented and return results as OpenTurns objects (NumericalComplexCollection).

