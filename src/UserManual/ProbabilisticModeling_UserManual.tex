% Copyright (c)  2005-2010 EDF-EADS-PHIMECA.
% Permission is granted to copy, distribute and/or modify this document
% under the terms of the GNU Free Documentation License, Version 1.2
% or any later version published by the Free Software Foundation;
% with no Invariant Sections, no Front-Cover Texts, and no Back-Cover
% Texts.  A copy of the license is included in the section entitled "GNU
% Free Documentation License".

\newpage\section{Probabilistic modeling}

In this section, we describe all the objects necessary to model a random vector.

\subsection{Distribution}

\begin{description}

\item[Usage :] $Distribution(dist)$

\item[Arguments :] \rule{0pt}{1em}
  \begin{description}
  \item $dist$ : a DistributionImplementation which is particular distribution.
  \end{description}

\item[Value :] a Distribution

\item[Some methods :]  \rule{0pt}{1em}

  \begin{description}

  \item $computeProbability$
    \begin{description}
    \item[Usage :]  \rule{0pt}{1em}
      \begin{description}
      \item $computeProbability(interval)$
      \end{description}
    \item[Arguments :] \rule{0pt}{1em}
      \begin{description}
      \item $interval$ : an Interval
      \end{description}
    \item[Value :] \rule{0pt}{1em}
      \begin{description}
      \item it gives the evaluation (a scalar) of the probability for the given distribution to take values within the given interval $interval$
      \end{description}
    \end{description}
    \bigskip

  \item $computeCDF$
    \begin{description}
    \item[Usage :]  \rule{0pt}{1em}
      \begin{description}
      \item $computeCDF(scalar)$
      \item $computeCDF(scalar, flag)$
      \item $computeCDF(vector)$
      \item $computeCDF(vector, flag)$
      \item $computeCDF(sample)$
      \item $computeCDF(sample, flag)$
      \end{description}
    \item[Arguments :] \rule{0pt}{1em}
      \begin{description}
      \item $scalar$ : a NumericalScalar
      \item $vector$ : a NumericalPoint
      \item $sample$ : a NumericalSample
      \item $flag$ : a Bool
      \end{description}
    \item[Value :] \rule{0pt}{1em}
      \begin{description}
      \item if flag is false (default value), the method computes the CDF, else it computes the complementary CDF, i.e. computeCDF(x, False) = 1 - computeCDF(x).
      \item using the first usage, it gives the evaluation (a scalar) of the CDF (Cumulative Distribution Function) of a distribution of dimension 1 at the given scalar value $scalar$
      \item using the second usage, it gives the evaluation (a scalar) of the CDF (Cumulative Distribution Function) of a distribution of arbitrary dimension at the given point $vector$
      \item using the third usage, it gives the evaluation (a NumericalSample) of the CDF (Cumulative Distribution Function) of a distribution of arbitrary dimension over the given sample $sample$
      \end{description}
    \end{description}
    \bigskip

  \item $computeCDFGradient$
    \begin{description}
    \item[Usage :] $computeCDFGradient(vector)$
    \item[Arguments :] $vector$ : a NumericalPoint
    \item[Value :] a NumericalPoint object, the gradient of the distribution CDF, with respect to the parameters of the distribution, evaluated at point $vector$
    \end{description}
    \bigskip

  \item $computeCharacteristicFunction$
    \begin{description}
    \item[Usage :] $computeCharacteristicFunction(vector)$
    \item[Arguments :] $vector$ : a NumericalPoint
    \item[Value :] a NumericalComplex object, the value of the characteristic function at point $vector$. Open TURNS proposes an implementation of all its univariate distributions, continuous or discrete ones. But only some of the them have the implementation of a specific algorithm that evaluates the characteristic function : it is the case of all the discrete distributions and most of (but not all) the continuous ones. In that case, the evaluation is performant. For the remaining distributions, the generic implementation might be time consuming for high arguments.

    \end{description}
    \bigskip

  \item $computeDDF$
    \begin{description}
    \item[Usage :] \rule{0pt}{1em}
      \begin{description}
      \item $computeDDF(vector)$
      \item $computeDDF(sample)$
      \end{description}
    \item[Arguments :] \rule{0pt}{1em}
      \begin{description}
      \item $vector$ : a NumericalPoint
      \item $sample$ : a NumericalSample
      \end{description}
    \item[Value :] \rule{0pt}{1em}
      \begin{description}
      \item while using the first usage,  a NumericalPoint value, the gradient of the PDF (Probability Distribution Function) of the considered distribution at $vector$ (DDF = Derivative Density Function)
      \item while using the second usage,  a NumericalSample, the gradient of the PDF (Probability Distribution Function) of the considered distribution at $vector$ (DDF = Derivative Density Function)
      \end{description}
    \end{description}
    \bigskip

  \item $computeGeneratingFunction$
    \begin{description}
    \item[Usage :] \rule{0pt}{1em}
      \begin{description}
      \item $computeGeneratingFunction(value)$
      \item $computeGeneratingFunction(value, logScale)$
      \end{description}
    \item[Arguments :] \rule{0pt}{1em}
      \begin{description}
      \item $value$ : a NumericalComplex, a numerical complex value within which module is $<1$
      \item $logScale$ : a Boolean which indicates whether the generating function is computed on a logarithmic scale. By default, $logScale = False$.
      \end{description}
    \item[Value :]  a numerical complex value, the value of the generating function at $value$
    \end{description}
    \bigskip

  \item $computePDF$
    \begin{description}
    \item[Usage :] \rule{0pt}{1em}
      \begin{description}
      \item $computePDF(value)$
      \item $computePDF(vector)$
      \item $computePDF(sample)$
      \end{description}
    \item[Arguments :] \rule{0pt}{1em}
      \begin{description}
      \item $vector$ : a NumericalPoint
      \item $sample$ : a NumericalSample
      \end{description}
    \item[Value :] \rule{0pt}{1em}
      \begin{description}
      \item while using the first usage, a NumericalScalar, the PDF (Cumulative Distribution Function) of dimension 1 value of the considered distribution at $value$
      \item while using the second usage, a NumericalPoint, the PDF (Cumulative Distribution Function) value of the considered distribution at the vector $vector$
      \item while using the third usage, a NumericalSample, the PDF (Cumulative Distribution Function) values of the considered distribution at $sample$
      \end{description}
    \end{description}
    \bigskip

  \item $computePDFGradient$
    \begin{description}
    \item[Usage :] $computePDFGradient(vector)$
    \item[Arguments :] $vector$ : a NumericalPoint
    \item[Value :] a NumericalPoint object, the gradient of the distribution PDF, with respect to the parameters of the distribution, evaluated at point $vector$
    \end{description}
    \bigskip

  \item $computeQuantile$
    \begin{description}
    \item[Usage :] \rule{0pt}{1em}
      \begin{description}
      \item $computeQuantile(p)$
      \item $computeQuantile(p, flag)$
      \end{description}
    \item[Arguments :] \rule{0pt}{1em}
      \begin{description}
      \item $p$ : a real scalar $0\leq p \leq 1$
      \item $flag$ : a Bool
      \end{description}
    \item[Value :] a NumericalPoint, the value of the $p-$ quantile if flag = False, the value of the $(1-p)-$ quantile if flag = True. If the distribution if of dimension $n>1$, the $p-$ quantile is the hyper surface in $\mathbb{R}^n$ defined by  $\{\vect{x}\in \mathbb{R}^n, F(x_1, \dots, x_n) = p \}$ where $F$ is the CDF. Open TURNS makes the choice to return one particular point among these points : $(x_1^p, \dots, x_n^p)$ such that $\forall i, F_i(x_i^p) =  \tau$ where $F_i$ is the marginal of component $X_i$ and $F(x_1, \dots, x_n) = C(\tau, \dots, \tau)$ where $C$ is the distribution copula. Thus, Open TURNS resolves the equation $ C(\tau, \dots, \tau)=p$ then computes $F_i^{-1}(\tau) = x_i^p$.
    \end{description}
    \bigskip

  \item $drawCDF$
    \begin{description}
    \item[Usage :] \rule{0pt}{1em}
      \begin{description}
      \item $drawCDF()$
      \item $drawCDF(min,max)$
      \item $drawCDF(min,max,pointNumber)$
      \item $drawCDF(vectMin,vectMax)$
      \item $drawCDF(vectMin,vectMax,vectPointNumber)$
      \end{description}

    \item[Arguments :] \rule{0pt}{1em}
      \begin{description}
      \item $min$ and $max$ : real values with $min < max$, the range for the CDF curve of a distribution of dimension 1
      \item $pointNumber$ : an integer, the number of points to draw the CDF iso-curves of a distribution of dimension 1
      \item $vectMin$ and $vectMax$ : two NumericalPoint of dimension 2, respectively the left-bottom and ritgh-up corners of the square for the CDF iso-curves of a distribution of dimension 2
      \item $vectPointNumber$ : a NumericalPoint of dimension 2, the the number of points to draw the iso-curves of a distribution of dimension 2 on each direction
      \end{description}
    \item[Value :] a Graph, containing the elements of the curve or iso-curves of the CDF, depending on the dimension of the distribution (1 or 2)
    \end{description}

    \bigskip

  \item $drawPDF$
    \begin{description}
    \item[Usage :] \rule{0pt}{1em}
      \begin{description}
      \item $drawPDF()$
      \item $drawPDF(min,max)$
      \item $drawPDF(min,max,pointNumber)$
      \item $drawPDF(vectMin,vectMax)$
      \item $drawPDF(vectMin,vectMax,vectPointNumber)$
      \end{description}

    \item[Arguments :] \rule{0pt}{1em}
      \begin{description}
      \item $min$ and $max$ : real values with $min < max$, the range for the PDF curve of a distribution of dimension 1
      \item $pointNumber$ : an integer, the number of points to draw the PDF iso-curves of a distribution of dimension 1
      \item $vectMin$ and $vectMax$ : two NumericalPoint of dimension 2, respectively the left-bottom and ritgh-up corners of the square for the PDF iso-curves of a distribution of dimension 2
      \item $vectPointNumber$ : a NumericalPoint of dimension 2, the number of points to draw the iso-curves of a distribution of dimension 2 on each direction
      \end{description}
    \item[Value :] a Graph, containing the elements of the curve or iso-curves of the PDF, depending on the dimension of the distribution (1 or 2)
    \end{description}


    \bigskip

  \item $drawMarginal1DCDF$
    \begin{description}
    \item[Usage :] \rule{0pt}{1em}
      \begin{description}
      \item $drawMarginal1DCDF(i, min,max,pointNumber)$
      \end{description}

    \item[Arguments :] \rule{0pt}{1em}
      \begin{description}
      \item $i$ : an integer, the marginal we want to draw (Care : numerotation begins at 0)
      \item $min$ and $max$ : real values with $min < max$, the range for the CDF curve of a distribution of dimension >1
      \item $pointNumber$ : an integer, the number of points to draw the CDF iso-curves of a distribution of dimension >1
      \end{description}
    \item[Value :] a Graph, containing the elements of the curve of the CDF of the marginal i of the distribution of dimension >1
    \end{description}

    \bigskip

  \item $drawMarginal1DPDF$
    \begin{description}
    \item[Usage :] \rule{0pt}{1em}
      \begin{description}
      \item $drawMarginal1DPDF(i, min, max, pointNumber)$
      \end{description}

    \item[Arguments :] \rule{0pt}{1em}
      \begin{description}
      \item $i$ : an integer, the marginal we want to draw (Care : numerotation begins at 0)
      \item $min$ and $max$ : real values with $min < max$, the range for the PDF curve of a distribution of dimension >1
      \item $pointNumber$ : an integer, the number of points to draw the PDF iso-curves of a distribution of dimension >1
      \end{description}
    \item[Value :] a Graph, containing the elements of the curve of the PDF of the marginal i of the distribution of dimension >1
    \end{description}

    \bigskip

  \item $drawMarginal2DCDF$
    \begin{description}
    \item[Usage :] \rule{0pt}{1em}
      \begin{description}
      \item $drawMarginal2DCDF(i, j, vectMin,vectMax,vectPointNumber)$
      \end{description}

    \item[Arguments :] \rule{0pt}{1em}
      \begin{description}
      \item $i$ and $j$ : two integer, the marginal we want to draw (Care : numerotation begins at 0)
      \item $vectMin$ and $vectMax$ : two NumericalPoint of dimension n>2, respectively the left-bottom and ritgh-up corners of the square for the PDF iso-curves of a distribution of dimension n
      \item $vectPointNumber$ : a NumericalPoint of dimension n>2, the number of points to draw the iso-curves of a distribution of dimension n on each direction
      \end{description}
    \item[Value :] a Graph, containing the elements of the iso-curve of the CDF of the marginals (i,j) of distribution of dimension n>2
    \end{description}

    \bigskip

  \item $drawMarginal2DPDF$
    \begin{description}
    \item[Usage :] \rule{0pt}{1em}
      \begin{description}
      \item $drawMarginal2DPDF(i, j, vectMin,vectMax,vectPointNumber)$
      \end{description}

    \item[Arguments :] \rule{0pt}{1em}
      \begin{description}
      \item $i$ and $j$ : two integer, the marginal we want to draw (Care : numerotation begins at 0)
      \item $vectMin$ and $vectMax$ : two NumericalPoint of dimension n>2, respectively the left-bottom and ritgh-up corners of the square for the PDF iso-curves of a distribution of dimension n
      \item $vectPointNumber$ : a NumericalPoint of dimension n>2, the number of points to draw the iso-curves of a distribution of dimension n on each direction
      \end{description}
    \item[Value :] a Graph, containing the elements of the iso-curve of the PDF of the marginals (i,j) of distribution of dimension n>2
    \end{description}

    \bigskip

  \item $getCopula$
    \begin{description}
    \item[Usage :] $getCopula()$
    \item[Arguments :] no argument
    \item[Value :] a Copula, the copula of the considered distribution. If the distribution is of type ComposedDistribution, the copula is the one specified at the creation of the ComposedDistribution. If the distribution is not that sort (for example, a KernelMixture, a Mixture, a RandomMixture), the copula is computed from the Sklar theorem.
    \end{description}
    \bigskip

  \item $getCovariance$
    \begin{description}
    \item[Usage :] $getCovariance()$
    \item[Arguments :] no argument
    \item[Value :] a CovarianceMatrix of the considered distribution (if the distribution is unidimensional, it is the variance)
    \end{description}
    \bigskip

  \item $getMarginal$
    \begin{description}
    \item[Usage :] \rule{0pt}{1em}
      \begin{description}
      \item $getMarginal(i)$
      \item $getMarginal(indices)$
      \end{description}


    \item[Arguments :]  \rule{0pt}{1em}
      \begin{description}
      \item $i$ : an integer (i is lower or equal to the dimension of the considered distribution), with $0 \leq i$
      \item $indices$ : a Indices, which regroup all the indices considered
      \end{description}

    \item[Value :] a Distribution, the distribution of an extracted vector of the initial distribution
    \end{description}
    \bigskip

  \item $getKurtosis$
    \begin{description}
    \item[Usage :] $getKurtosis()$
    \item[Arguments :] no argument
    \item[Value :] a NumericalPoint, the value the kurtosis of each 1D marginal of the distribution
    \end{description}
    \bigskip

  \item $getMean$
    \begin{description}
    \item[Usage :] $getMean()$
    \item[Arguments :] no argument
    \item[Value :] a NumericalPoint, the value of the considered distribution mean
    \end{description}
    \bigskip

  \item $getNumericalSample$
    \begin{description}
    \item[Usage :] $getNumericalSample(n)$
    \item[Arguments :] $n$ : integer, the size of the sample
    \item[Value :] a NumericalSample representing $n$ realizations of the random variable with the considered distribution
    \end{description}
    \bigskip

  \item $getParametersCollection$
    \begin{description}
    \item[Usage :] $getParametersCollection()$
    \item[Arguments :] one
    \item[Value :] a NumericalPointWithDescriptionCollection, the list of the parameters of the distribution

    \end{description}
    \bigskip


  \item $getRealization$
    \begin{description}
    \item[Usage :] $getRealization()$
    \item[Arguments :] no argument
    \item[Value :] a NumericalPoint, one realization of random variable with the considered distribution
    \end{description}
    \bigskip

  \item $getRoughness$
    \begin{description}
    \item[Usage :] $getRoughness()$
    \item[Arguments :] no argument
    \item[Value :] a NumericalScalar, the value $roughness(\vect{X}) = ||p||_{\mathcal{L}^2} = \sqrt{\int_\vect{x} p^2(\vect{x})d\vect{x}}$
    \end{description}
    \bigskip

  \item $getSupport$
    \begin{description}
    \item[Usage :] \rule{0pt}{1em}
      \begin{description}
      \item $getSupport(interval)$
      \item $getSupport()$
      \end{description}
    \item[Arguments :] $interval$ : a Interval
    \item[Value :] \rule{0pt}{1em}
      \begin{description}
      \item in the first usage, a NumericalSample which gathers the different points of the discrete range. Care : this service is implemented only for discrete 1D distribution.
      \item in the second usage, a NumericalSample which gathers the different points of the discrete range which are inside  $interval$.
      \end{description}
    \end{description}
    \bigskip

  \item $getSkewness$
    \begin{description}
    \item[Usage :] $getSkewness()$
    \item[Arguments :] no argument
    \item[Value :] a NumericalPoint, the value the standard deviation of each 1D marginal of the distribution
    \end{description}
    \bigskip

  \item $getStandardDeviation$
    \begin{description}
    \item[Usage :] $getStandardDeviation()$
    \item[Arguments :] no argument
    \item[Value :] a NumericalPoint, the value the standard deviation of each 1D marginal of the distribution
    \end{description}
    \bigskip

  \item $getWeight$
    \begin{description}
    \item[Usage :] $getWeight()$
    \item[Arguments :] no argument
    \item[Value :] a NumericalScalar between 0 and 1, the weight of the considered distribution if used in a Mixture
    \end{description}
    \bigskip

  \item $hasEllipticalCopula$
    \begin{description}
    \item[Usage :] $hasEllipticalCopula()$
    \item[Arguments :] no argument
    \item[Value :] a boolean, it says if the considered distribution is elliptical
    \end{description}
    \bigskip

  \item $hasIndependentCopula$
    \begin{description}
    \item[Usage :] $hasIndependentCopula()$
    \item[Arguments :] no argument
    \item[Value :] a boolean which indicates wether the considered distribution is independent
    \end{description}
    \bigskip

  \item $isElliptical$
    \begin{description}
    \item[Usage :] $isElliptical()$
    \item[Arguments :] no argument
    \item[Value :] a boolean which indicates wether the considered distribution has an elliptical distribution
    \end{description}
    \bigskip

  \item $isIntegral$
    \begin{description}
    \item[Usage :] $isIntegral()$
    \item[Arguments :] no argument
    \item[Value :] a boolean which indicates wether the considered distribution has integer values.
    \end{description}
    \bigskip

  \item $str$
    \begin{description}
    \item[Usage :] $str()$
    \item[Arguments :] no argument
    \item[Value :] a string describing the object
    \end{description}
    \bigskip

  \end{description}

\end{description}

% -=============================================================
\newpage \subsection{Usual Distributions} \label{UsualDistributions}


\subsubsection{Arcsine}

This class inherits from the Distribution class.

\begin{description}

\item[Usage :] \rule{0pt}{1em}
  \begin{description}
  \item Main parameters set : $Arcsine(a,b)$
  \item Second parameters set : $Arcsine( \mu, \sigma,1)$
  \item Default construction : $Arcsine( )$
  \end{description}

\item[Arguments :]  \rule{0pt}{1em}
  \begin{description}
  \item $a$ : a real value, the lower bound
  \item $b$ : a real value, the upper bound, constraint: $a<b$
  \end{description}

\item[Value :]  Arcsine. In the default construction, we use the $Arcsine(a,b) = Arcsine(-1.0,1.0)$ definition.

\item[Some methods :] \rule{0pt}{1em}
  \begin{description}

  \item $getA$
    \begin{description}
    \item[Usage :] $getA()$
    \item[Arguments :] none
    \item[Value :]  a real value, the lower bound
    \end{description}
    \bigskip

  \item $getB$
    \begin{description}
    \item[Usage :] $getB()$
    \item[Arguments :] none
    \item[Value :]  a real value, the upper bound
    \end{description}

  \item $getMu$
    \begin{description}
    \item[Usage :] $getMu()$
    \item[Arguments :] none
    \item[Value :]  a real value, the mean
    \end{description}
    \bigskip

  \item $getSigma$
    \begin{description}
    \item[Usage :] $getSigma()$
    \item[Arguments :] none
    \item[Value :]  a real value, the standard deviation
    \end{description}



  \end{description}

\item[Details :]  \rule{0pt}{1em}
  \begin{description}
  \item density function :

    $$\frac{1}{\pi\frac{b-a}{2}\sqrt{1-\left(\frac{x-\frac{a+b}{2}}{\frac{b-a}{2}}\right)^{2}}}$$

  \item relation between parameter sets :
    \begin{eqnarray*}
      \mu                                       &       =       \frac{a+b}{2}   \\
      \sigma                            &  =    &       \frac{b-a}{2\sqrt{2}}
    \end{eqnarray*}
    \begin{align*}
      \mbox{where}
      &&
      \mu = \Expect{X}
      &&
      \sigma = \sqrt{\Var X }
    \end{align*}
  \end{description}
  \bigskip

\item[Links :]  \rule{0pt}{1em}
  \href{./Version/docref_B121_DistributionSelection.pdf}{see docref\_B121\_DistributionSelection}
\end{description}


Each  $getMethod$  is associated to a $setMethod$.
% =============================================================

\newpage \subsubsection{Bernoulli}

This class inherits from the Distribution class.

\begin{description}

\item[Usage :]\rule{0pt}{1em}
  \begin{description}
  \item Main parameters set : $Bernoulli(p)$
  \item  Default construction : $Bernoulli()$
  \end{description}

\item[Arguments :]  $p$ : a real value,
  constraint : $0\leq p\leq 1$

\item[Value :] a Bernoulli. In the default construction, we use the $Bernopulli() = Bernoulli(0.5)$ definition.

\item[Some methods :] \rule{0pt}{1em}
  \begin{description}

  \item $getP$
    \begin{description}
    \item[Usage :] $getP()$
    \item[Arguments :] none
    \item[Value :]  a real positive value $\leq 1$, the $p$ parameter of the  distribution.
    \end{description}
    \bigskip

  \item $getSupport$
    \begin{description}
    \item[Usage :] $getSupport(interval)$
    \item[Arguments :] $interval$ : a $Interval$, an interval in $\mathbb{R}$
    \item[Value :]  a $NumericalSample$, all the points (here of dimension 1) of the distribution range which are included in the interval $interval$.
    \end{description}

  \end{description}

\item[Details :]  \rule{0pt}{1em}
  \begin{description}
  \item probability distribution:
    $$
    \Prob 1  = p, \Prob 0 = 1-p
    $$
  \item relation between parameters set :
    \begin{eqnarray*}
      \mu  =   p                                              & \mbox{where}& \mu =\Expect{X} \\
      \sigma  = \sqrt{p(1-p)}  & \mbox{where}& \sigma =\sqrt{\Var X }
    \end{eqnarray*}

  \end{description}
  \bigskip

\item[Links :]  \rule{0pt}{1em}
  \href{OpenTURNS_ReferenceGuide.pdf}{see Reference Guide - Standard parametric models}
\end{description}


Each  $getMethod$  is associated to a $setMethod$.


% ==============================================
\newpage \subsubsection{Beta}

This class inherits from the Distribution class.

\begin{description}

\item[Usage :] \rule{0pt}{1em}
  \begin{description}
  \item Main parameters set : $Beta( r,  t,  a,  b)$
  \item Second parameters set : $Beta( \mu, \sigma, a, b, Beta.MUSIGMA)$
  \item Default construction : $Beta( )$
  \end{description}

\item[Arguments :]  \rule{0pt}{1em}
  \begin{description}
  \item $r$ : real value, first shape parameter, constraint : $r>0$
  \item $t$ :  real value, second shape parameter, constraint : $t>r$
  \item $a$ : real value, lower bound
  \item $b$ : real value, upper bound, constraint : $b>a$
  \item $\mu$ : real value, mean value
  \item $\sigma$ : real value, standard deviation, constraint : $\sigma >0$
  \end{description}
  \bigskip

\item[Value :] a Beta. In the default construction, we use the $Beta(r, t, a, b)= Beta(2, 4, -1, 1)$ definition.

\item[Some methods :]  \rule{0pt}{1em}
  \begin{description}

  \item $getA$
    \begin{description}
    \item[Usage :] $getA()$
    \item[Arguments :] none
    \item[Value :]  a real value, the $a$ parameter of the Beta distribution
    \end{description}
    \bigskip
  \item $getB$
    \begin{description}
    \item[Usage :] $getB()$
    \item[Arguments :] none
    \item[Value :]  a real value, the  $b$ parameter of the Beta distribution
    \end{description}
    \bigskip
  \item $getMu$
    \begin{description}
    \item[Usage :] $getMu()$
    \item[Arguments :] none
    \item[Value :]  a real value,  the $\mu$ parameter of the  distribution
    \end{description}
    \bigskip
  \item $getSigma$
    \begin{description}
    \item[Usage :] $getSigma()$
    \item[Arguments :] none
    \item[Value :]  a real value,  the $\sigma$ parameter of the  distribution
    \end{description}
    \bigskip
  \item $getR$
    \begin{description}
    \item[Usage :] $getR()$
    \item[Arguments :] none
    \item[Value :]  a real value,  the $r$ parameter of the  distribution
    \end{description}
    \bigskip
  \item $getT$
    \begin{description}
    \item[Usage :] $getT()$
    \item[Arguments :] none
    \item[Value :]  a real value,  the $t$ parameter of the  distribution
    \end{description}
    \bigskip

  \end{description}


\item[Details :]  \rule{0pt}{1em}
  \begin{description}
  \item density probability function :
    $$
    f(x) = \frac{(x-a)^{(r-1)}(b-x)^{(t-r-1)}}{(b-a)^{(t-1)}B(r,t-r)}\boldsymbol{1}_{[a,b]}(x)
    $$
  \item relation between parameters sets :
    $$
    \begin{array}{lcl}
      \mu & = & \displaystyle a + \frac{(b - a)  r}{ t} \\
      \sigma & = & \displaystyle  \frac{(b - a)}{ t}\sqrt{\frac{r (t - r)}{ (t + 1)}}
    \end{array}
    $$
  \end{description}

\item[Links :]  \rule{0pt}{1em}
  \href{OpenTURNS_ReferenceGuide.pdf}{see Reference Guide - Standard parametric models}
\end{description}

Each  $getMethod$  is associated to a $setMethod$.


% ==============================================
\newpage \subsubsection{Binomial}

This class inherits from the Distribution class.

\begin{description}

\item[Usage :]\rule{0pt}{1em}
  \begin{description}
  \item Main parameters set : $Binomial(n,p)$
  \item  Default construction : $Binomial()$
  \end{description}

\item[Arguments :]  \rule{0pt}{1em}
  \begin{description}
  \item $n$ : an integer $>0$,
  \item  $p$ : a real value such as $0\leq p\leq 1$.
  \end{description}

\item[Value :] a Binomial. In the default construction, we use the $Binomial() = Binomial(1,0.5)$ definition.

\item[Some methods :] \rule{0pt}{1em}
  \begin{description}

  \item $getP$
    \begin{description}
    \item[Usage :] $getP()$
    \item[Arguments :] none
    \item[Value :]  a real positive value $\leq 1$, the $p$ parameter of the  distribution.
    \end{description}
    \bigskip

  \item $getN$
    \begin{description}
    \item[Usage :] $getN()$
    \item[Arguments :] none
    \item[Value :]  an integer, the $n$ parameter of the  distribution.
    \end{description}
    \bigskip

  \item $getSupport$
    \begin{description}
    \item[Usage :] $getSupport(interval)$
    \item[Arguments :] $interval$ : a $Interval$, an interval in $\mathbb{R}$
    \item[Value :]  a $NumericalSample$, all the points (here of dimension 1) of the distribution range which are included in the interval $interval$.
    \end{description}

  \end{description}

\item[Details :]  \rule{0pt}{1em}
  \begin{description}
  \item probability distribution:
    $$
    \Prob k  = C_n^k p^k (1-p)^{n-k}, \, \forall k \in \{0, \hdots, n\}
    $$
  \item relation between parameters set :
    \begin{eqnarray*}
      \mu  =   p                                              & \mbox{where}& \mu =\Expect{X} \\
      \sigma  = \sqrt{p(1-p)}  & \mbox{where}& \sigma =\sqrt{\Var X }
    \end{eqnarray*}

  \end{description}
  \bigskip

\item[Links :]  \rule{0pt}{1em}
  \href{OpenTURNS_ReferenceGuide.pdf}{see Reference Guide - Standard parametric models}
\end{description}


Each  $getMethod$  is associated to a $setMethod$.




% ==============================================
\newpage \subsubsection{Burr}

This class inherits from the Distribution class.

\begin{description}

\item[Usage :]\rule{0pt}{1em}
  \begin{description}
  \item Main parameters set : $Burr(c,k)$
  \item  Default construction : $Burr()$
  \end{description}

\item[Arguments :]  \rule{0pt}{1em}
  \begin{description}
  \item $c$ : an real $>0$,
  \item  $k$ : a real value $>0$.
  \end{description}

\item[Value :] a Burr. In the default construction, we use the $Burr() = Burr(1,1.0)$ definition.

\item[Some methods :] \rule{0pt}{1em}
  \begin{description}

  \item $getC$
    \begin{description}
    \item[Usage :] $getC()$
    \item[Arguments :] none
    \item[Value :]  a real positive value, the $c$ parameter of the  distribution.
    \end{description}
    \bigskip

  \item $getK$
    \begin{description}
    \item[Usage :] $getK()$
    \item[Arguments :] none
    \item[Value :]  an real positive value, the $k$ parameter of the  distribution.
    \end{description}
    \bigskip

  \item $getSupport$
    \begin{description}
    \item[Usage :] $getSupport(interval)$
    \item[Arguments :] $interval$ : a $Interval$, an interval in $\mathbb{R}$
    \item[Value :]  a $NumericalSample$, all the points (here of dimension 1) of the distribution range which are included in the interval $interval$.
    \end{description}

  \end{description}

\item[Details :]  \rule{0pt}{1em}
  \begin{description}
  \item probability distribution:
    $$
    p(x;\vect{\theta}) = ck\frac{x^{(c-1)}}{(1+x^c)^{(k+1)}} \mathbf{1}_{x >0}
    $$
  \item relation between parameters set :
    \begin{eqnarray*}
      \mu  =   kBeta(k-1/c, 1+1/c)                                              & \mbox{where}& \mu =\Expect{X}
    \end{eqnarray*}

  \end{description}
  \bigskip

\item[Links :]  \rule{0pt}{1em}
  \href{OpenTURNS_ReferenceGuide.pdf}{see Reference Guide - Standard parametric models}
\end{description}


Each  $getMethod$  is associated to a $setMethod$.

% =============================================================
\newpage \subsubsection{Chi}

This class inherits from the Distribution class.

\begin{description}

\item[Usage :] \rule{0pt}{1em}
  \begin{description}
  \item Main parameters set : $Chi( \nu)$
  \item Default construction : $Chi( )$
  \end{description}

\item[Arguments :]  \rule{0pt}{1em}
  \begin{description}
  \item $\nu$ : real value, degrees of feedom, constraint : $\nu>0$
  \end{description}
  \bigskip

\item[Value :] a Chi. In the default construction, we use the $Chi(\nu)= Chi(1)$ definition.

\item[Some methods :]  \rule{0pt}{1em}
  \begin{description}

  \item $getNu$
    \begin{description}
    \item[Usage :] $getNu()$
    \item[Arguments :] none
    \item[Value :]  a real value, the $\nu$ parameter of the Chi distribution
    \end{description}
  \end{description}


\item[Details :]  \rule{0pt}{1em}
  \begin{description}
  \item density probability function :
    $$
    f(x) = \displaystyle x^{\nu-1}e^{-x^2/2}\frac{2^{1-\nu^{\strut}/2}}{\Gamma(\nu/2)_{\strut}} \boldsymbol{1}_{[0,+\infty[}(x)
    $$
  \end{description}

\item[Links :]  \rule{0pt}{1em}
  \href{OpenTURNS_ReferenceGuide.pdf}{see Reference Guide - Standard parametric models}
\end{description}

Each  $getMethod$  is associated to a $setMethod$.



% =============================================================
\newpage \subsubsection{ChiSquare}

This class inherits from the Distribution class.

\begin{description}

\item[Usage :] \rule{0pt}{1em}
  \begin{description}
  \item Main parameters set : $ChiSquare( \nu)$
  \item Default construction : $ChiSquare( )$
  \end{description}

\item[Arguments :]  \rule{0pt}{1em}
  \begin{description}
  \item $\nu$ : real value, degrees of feedom, constraint : $\nu>0$
  \end{description}
  \bigskip

\item[Value :] a ChiSquare. In the default construction, we use the $ChiSquare(\nu)= ChiSquare(1)$ definition.

\item[Some methods :]  \rule{0pt}{1em}
  \begin{description}

  \item $getNu$
    \begin{description}
    \item[Usage :] $getNu()$
    \item[Arguments :] none
    \item[Value :]  a real value, the $\nu$ parameter of the ChiSquare distribution
    \end{description}
  \end{description}


\item[Details :]  \rule{0pt}{1em}
  \begin{description}
  \item density probability function :
    $$
    f(x) = \frac{2^{-\nu/2}}{\Gamma(\nu/2)} x^{(\nu/2-1)}e^{-x/2}\boldsymbol{1}_{[0,+\infty[}(x)
    $$
  \end{description}

\item[Links :]  \rule{0pt}{1em}
  \href{OpenTURNS_ReferenceGuide.pdf}{see Reference Guide - Standard parametric models}
\end{description}

Each  $getMethod$  is associated to a $setMethod$.

% =============================================================
\newpage \subsubsection{Dirichlet}


This class inherits from the Distribution class.

\begin{description}

\item[Usage :]
  \begin{description}
  \item Main parameters set : $Dirichlet(vectTheta)$
  \item Default construction : $Dirichlet( )$
  \end{description}

\item[Arguments :]  \rule{0pt}{1em}
  \begin{description}
  \item $vectTheta$ : a NumericalPoint : $(\theta_1, \hdots, \theta_{d+1})$ for a multivariate $d$-dimensional distribution.
  \end{description}

\item[Value :] an Dirichlet. In the default construction, a univariate distribution is created with $(\theta_1, \theta-2) = (1,1)$.

\item[Some methods :]  \rule{0pt}{1em}
  \begin{description}

  \item $getTheta$
    \begin{description}
    \item[Usage :] $getTheta()$
    \item[Arguments :] none
    \item[Value :]  a NumericalPoint : the $(\theta_1, \hdots, \theta_{d+1})$  parameter of the  distribution
    \end{description}

  \end{description}



\item[Details :]  \rule{0pt}{1em}
  \begin{description}
  \item density probability function :
    $$
    f_X(\vect{x};\vect{\theta}) = \displaystyle \frac{\Gamma(\sum_{j=1}^{d+1}\theta_j)}{\prod_{j=1}^{d+1}\Gamma(\theta_j)} \left[ 1-\sum_{j=1}^{d} x_j\right]^{\theta_{d+1}-1}\prod_{j=1}^d x_j^{\theta_j-1}\mathbf{1}_{\Delta}(\vect{x})
    $$
    with $\Delta = \{ \vect{x} \in \mathbb{R}^d / \forall i, x_i \geq 0, \sum_{i=1}^{d} x_i \leq 1 \}$.

  \end{description}

\item[Links :]  \rule{0pt}{1em}
  \href{OpenTURNS_ReferenceGuide.pdf}{see Reference Guide - Standard parametric models}
\end{description}

Each  $getMethod$  is associated to a $setMethod$.



% =============================================================
\newpage \subsubsection{Epanechnikov}


This class inherits from the Distribution class.

\begin{description}

\item[Usage :]
  \begin{description}
  \item Default construction : $Epanechnikov( )$
  \end{description}

\item[Arguments :]  \rule{0pt}{1em}

\item[Value :] an Epanechnikov, which is a $Beta (a=-1, b=1, r=2, t=4)$ distribution.

\item[Some methods :]  \rule{0pt}{1em}



\item[Details :]  \rule{0pt}{1em}
  \begin{description}
  \item density probability function :
    $$
    f(x) = \displaystyle \frac{3^{\strut}}{4_{\strut}}(1 - x^2)\boldsymbol{1}_{[-1,1]}(x)
    $$

  \end{description}

\item[Links :]  \rule{0pt}{1em}
  \href{OpenTURNS_ReferenceGuide.pdf}{see Reference Guide - Standard parametric models}
\end{description}

Each  $getMethod$  is associated to a $setMethod$.


% =============================================================
\newpage \subsubsection{Exponential}


This class inherits from the Distribution class.

\begin{description}

\item[Usage :]
  \begin{description}
  \item Main parameters set : $Exponential(\lambda,\gamma)$
  \item Default construction : $Exponential( )$
  \end{description}

\item[Arguments :]  \rule{0pt}{1em}
  \begin{description}
  \item $\lambda$ : real value, scale parameter, constraint : $\lambda > 0 $
  \item $\gamma$ :  real value.
  \end{description}

\item[Value :] an Exponential. In the default construction, we use the $Exponential(lambda, gamma) = Exponential(1.0, 0.0)$ definition.

\item[Some methods :]  \rule{0pt}{1em}
  \begin{description}

  \item $getGamma$
    \begin{description}
    \item[Usage :] $getGamma()$
    \item[Arguments :] none
    \item[Value :]  a real value, the $\gamma$ parameter of the  distribution
    \end{description}
    \bigskip

  \item $getLambda$
    \begin{description}
    \item[Usage :] $getLambda()$
    \item[Arguments :] none
    \item[Value :]  a real value, the $\lambda$ parameter of the  distribution
    \end{description}
    \bigskip
  \end{description}



\item[Details :]  \rule{0pt}{1em}
  \begin{description}
  \item density probability function :
    $$
    f(x) = \displaystyle \lambda e^{-\lambda(x-\gamma)}\boldsymbol{1}_{[\gamma,+\infty[}(x)
    $$
  \item relation between parameters sets :
    \begin{eqnarray*}
      \mu = \gamma + \frac{1}{\lambda}                                   &  \mbox{where}& \mu =\Expect{X}\\
      \sigma = \frac{1}{\lambda} &  \mbox{where}& \sigma =\sqrt{\Var X }
    \end{eqnarray*}

  \end{description}

\item[Links :]  \rule{0pt}{1em}
  \href{OpenTURNS_ReferenceGuide.pdf}{see Reference Guide - Standard parametric models}
\end{description}

Each  $getMethod$  is associated to a $setMethod$.

% =============================================================
\newpage \subsubsection{FisherSnedecor}


This class inherits from the Distribution class.

\begin{description}

\item[Usage :]
  \begin{description}
  \item Main parameters set : $FisherSnedecor(d1,d2)$
  \item Default construction : $FisherSnedecor( )$
  \end{description}

\item[Arguments :]  \rule{0pt}{1em}
  \begin{description}
  \item $(d1,d2)$ : two  real value, scale parameter, constraint : $d_i > 0 $
  \end{description}

\item[Value :] an FisherSnedecor. In the default construction, we use the $FisherSnedecor(d1,d2) = FisherSnedecor(1.0, 1.0)$ definition.

\item[Some methods :]  \rule{0pt}{1em}
  \begin{description}

  \item $getD1$
    \begin{description}
    \item[Usage :] $getD1()$
    \item[Arguments :] none
    \item[Value :]  a real value, the $d_1$ parameter of the  distribution
    \end{description}
    \bigskip

  \item $getD2$
    \begin{description}
    \item[Usage :] $getD2()$
    \item[Arguments :] none
    \item[Value :]  a real value, the $d_2$ parameter of the  distribution
    \end{description}
    \bigskip

  \end{description}

\item[Details :]  \rule{0pt}{1em}
  \begin{description}
  \item density probability function :
    $$
    f(x) = \displaystyle \frac{1}{xB(d_1/2, d_2/2)}\left[\left(\frac{d_1x}{d_1x+d_2}\right)^{d_1/2} \left(1-\frac{d_1x}{d_1x+d_2}\right)^{d_2/2} \right]\mathbf{1}_{x \geq 0}
    $$
  \end{description}

\item[Links :]  \rule{0pt}{1em}
  \href{OpenTURNS_ReferenceGuide.pdf}{see Reference Guide - Standard parametric models}
\end{description}

Each  $getMethod$  is associated to a $setMethod$.

% =============================================================
\newpage \subsubsection{Gamma}

This class inherits from the Distribution class.

\begin{description}

\item[Usage :] \rule{0pt}{1em}
  \begin{description}
  \item Main parameters set : $Gamma(k ,\lambda ,\gamma)$
  \item Second parameters set : $Gamma( \mu, \sigma, \gamma, Gamma.MUSIGMA)$
  \item Default construction : $Gamma( )$
  \end{description}
  \bigskip

\item[Arguments :]  \rule{0pt}{1em}
  \begin{description}
  \item $k$ : integer value
    constraint : $k>0$
  \item $\lambda$ :  real value,
    constraint : $\lambda >0$
  \item $\gamma$ : real value,
  \item $\mu$ : real value, mean value
  \item $\sigma$ : real value, standard deviation, constraint : $\sigma > 0 $
  \end{description}

\item[Value :] a Gamma. In the default construction, we use the $Gamma(k, lambda, gamma) = Gamma(1.0, 1.0, 0.0)$ definition.

\item[Some methods :] \rule{0pt}{1em}
  \begin{description}

  \item $getGamma$
    \begin{description}
    \item[Usage :] $getGamma()$
    \item[Arguments :] none
    \item[Value :]  a real value, the $\gamma$ parameter of the  distribution
    \end{description}
    \bigskip

  \item $getK$
    \begin{description}
    \item[Usage :] $getK()$
    \item[Arguments :] none
    \item[Value :]  a real value, the $k$ parameter of the  distribution
    \end{description}
    \bigskip

  \item $getLambda$
    \begin{description}
    \item[Usage :] $getLambda()$
    \item[Arguments :] none
    \item[Value :]  a real value, the $\lambda$ parameter of the  distribution
    \end{description}
    \bigskip


  \item $setKLambda$
    \begin{description}
    \item[Usage :] $setKLambda(k, lambda)$
    \item[Arguments :]
      \begin{description}
      \item $k$ : a NumericalScalar, the $k$ parameter of the distribution
      \item $lambda$ : a NumericalScalar, the $\lambda$ parameter of the distribution
      \end{description}
    \end{description}

  \end{description}

\item[Details :]  \rule{0pt}{1em}
  \begin{description}
  \item density probability function :
    $$
    f(x) = \frac{\lambda}{\Gamma(k)}(\lambda(x-\gamma))^{(k-1)}
    e^{-\lambda(x-\gamma)}\boldsymbol{1}_{[\gamma,+\infty[}(x)
    $$
  \item relation between parameters sets :
    \begin{eqnarray*}
      \mu  =   \frac{k}{\lambda}+\gamma                                                 &  \mbox{where}& \mu =\Expect{X} \\
      \sigma  = \frac{\sqrt{k}}{\lambda}                &  \mbox{where}& \sigma =\sqrt{\Var X }
    \end{eqnarray*}

  \end{description}

\item[Links :]  \rule{0pt}{1em}
  \href{OpenTURNS_ReferenceGuide.pdf}{see Reference Guide - Standard parametric models}
\end{description}

Each  $getMethod$  is associated to a $setMethod$.

% =============================================================
\newpage \subsubsection{Geometric}

This class inherits from the Distribution class.

\begin{description}

\item[Usage :] Main parameters set : $Geometric(p)$

\item[Arguments :]  $p$ : a real value,
  constraint : $0<p<1$

\item[Value :] Geometric

\item[Some methods :] \rule{0pt}{1em}
  \begin{description}

  \item $getP$
    \begin{description}
    \item[Usage :] $getP()$
    \item[Arguments :] none
    \item[Value :]  a real positive value <1, the $p$ parameter of the  distribution
    \end{description}
    \bigskip

  \item $getSupport$
    \begin{description}
    \item[Usage :] $getSupport(interval)$
    \item[Arguments :] $interval$ : a $Interval$, an interval in $\mathbb{R}$
    \item[Value :]  a $NumericalSample$, all the points (here of dimension 1) of the distribution range which are included in the interval $interval$.
    \end{description}

  \end{description}

\item[Details :]  \rule{0pt}{1em}
  \begin{description}
  \item probability distribution:
    $$
    \Prob k  = (1-p)^{k-1} p, \, k\in \mathbb{N}^*
    $$
  \item relation between parameters set :
    \begin{eqnarray*}
      \mu  =   \frac{1}{p}                                              & \mbox{where}& \mu =\Expect{X} \\
      \sigma  = \sqrt{\frac{1-p}{p^2}}  & \mbox{where}& \sigma =\sqrt{\Var X }
    \end{eqnarray*}

  \end{description}
  \bigskip

\item[Links :]  \rule{0pt}{1em}
  \href{OpenTURNS_ReferenceGuide.pdf}{see Reference Guide - Standard parametric models}
\end{description}


Each  $getMethod$  is associated to a $setMethod$.


% =============================================================
\newpage \subsubsection{Gumbel}

This class inherits from the Distribution class.

\begin{description}

\item[Usage :] \rule{0pt}{1em}
  \begin{description}
  \item Main parameters set : $Gumbel(\alpha ,\beta)$
  \item Second parameters set : $Gumbel( \mu, \sigma,1)$
  \item Default construction : $Gumbel( )$
  \end{description}

\item[Arguments :]  \rule{0pt}{1em}
  \begin{description}
  \item $\alpha$ :  a real value, the scale parameter (the inverse), constraint : $\alpha >0$
  \item $\beta$ : a real value, location parameter
  \item $\mu$ : a real value, the mean value
  \item $\sigma$ : a real value, standard deviation, constraint : $\sigma > 0 $
  \end{description}

\item[Value :] a Gumbel. In the default construction, we use the $Gumbel(alpha, beta) = Gumbel(1.0, 1.0)$ definition.

\item[Some methods :]  \rule{0pt}{1em}
  \begin{description}

  \item $getAlpha$
    \begin{description}
    \item[Usage :] $getAlpha()$
    \item[Arguments :] none
    \item[Value :]  a real value, the  $\alpha$ of the considered distribution
    \end{description}
    \bigskip

  \item $getBeta$
    \begin{description}
    \item[Usage :] $getBeta()$
    \item[Arguments :] none
    \item[Value :]  a real value, the  $\beta$ of the considered distribution
    \end{description}
    \bigskip
  \item $getMu$
    \begin{description}
    \item[Usage :] $getMu()$
    \item[Arguments :] none
    \item[Value :]  a real value,  the $\mu$ parameter of the  distribution
    \end{description}
    \bigskip
  \item $getSigma$
    \begin{description}
    \item[Usage :] $getSigma()$
    \item[Arguments :] none
    \item[Value :]  a real value,  the $\sigma$ parameter of the  distribution
    \end{description}
    \bigskip
  \end{description}

\item[Details :]  \rule{0pt}{1em}
  \begin{description}
  \item density probability function :
    $$
    f(x) = \alpha e^{-\alpha(x-\beta)-e^{-\alpha(x-\beta)}}
    $$
  \item relation between parameters set :
    \begin{align*}
      \mu \;            =       &   \beta + \frac{c}{\alpha}
      & \mbox{where $c$ is the Euler-Mascheroni constant}
      && (c \approx 0.5772156649)
      \\
      \sigma \;  =      &       \frac{1}{\sqrt{6}}\frac{\pi}{\alpha}
    \end{align*}
    \begin{align*}
      \mbox{where}
      &&
      \mu = \Expect{X}
      &&
      \sigma = \sqrt{\Var X }
    \end{align*}
  \end{description}

\item[Links :]  \rule{0pt}{1em}
  \href{OpenTURNS_ReferenceGuide.pdf}{see Reference Guide - Standard parametric models}
\end{description}

Each  $getMethod$  is associated to a $setMethod$.

% ================================================================================

\newpage \subsubsection{Histogram}

This class inherits from the Distribution class.

\begin{description}

\item[Usage :] \rule{0pt}{1em}
  \begin{description}
  \item Main parameters set :   $Histogram(first,Coll)$
  \end{description}
  \bigskip

\item[Arguments :]  \rule{0pt}{1em}
  \begin{description}
  \item $first$ : a  real value, the lower bound of the distribution
  \item $Coll$ : an HistogramPairCollection, the collection of $(h_i, l_i)$ where $h_i$ is the height and $l_i$ the width of each barplot of the Histogram
  \end{description}
  \bigskip

\item[Value :] an Histogram with normalized heights

\item[Some methods :]  \rule{0pt}{1em}
  \begin{description}

  \item $getFirst$
    \begin{description}
    \item[Usage :] $getFirst()$
    \item[Arguments :] none
    \item[Value :]  a real value, the  $first$ parameter of the considered distribution
    \end{description}
    \bigskip

  \item $getPairCollection$
    \begin{description}
    \item[Usage :] $getPairCollection()$
    \item[Arguments :] none
    \item[Value :]  a HistogramPairCollection, the  $Coll$ parameter of the considered distribution
    \end{description}
    \bigskip
  \end{description}

\item[Details :] \rule{0pt}{1em}
  \begin{description}
  \item density probability function :
    $$
    f(x) = \sum_{i=1}^{n}H_i\;\boldsymbol{1}_{[x_i,x_{i+1}]}(x)
    $$
    where
    \begin{itemize}
    \item[] $H_i=h_i/S$ is the normalized heights, with $S=\sum_{i=1}^nh_i\,l_i$ being the initial surface of the histogram.
    \item[] $l_i = x_{i+1} - x_i$, $1\leq i \leq n$
    \item[] $n$ is the size of the HistogramPairCollection
    \end{itemize}
  \end{description}
\end{description}

Each  $getMethod$  is associated to a $setMethod$.

% =============================================================
\newpage \subsubsection{InverseNormal}


This class inherits from the Distribution class.

\begin{description}

\item[Usage :]
  \begin{description}
  \item Main parameters set : $InverseNormal(lambda, mu)$
  \item Default construction : $InverseNormal( )$
  \end{description}

\item[Arguments :]  \rule{0pt}{1em}
  \begin{description}
  \item $(lambda, mu)$ : two  real value, scale parameter, constraint : $\lambda > 0 $ and $\mu>0$
  \end{description}

\item[Value :] an InverseNormal. In the default construction, we use the $InverseNormal(lambda, mu) = InverseNormal(1.0, 1.0)$ definition.

\item[Some methods :]  \rule{0pt}{1em}
  \begin{description}

  \item $getLambda$
    \begin{description}
    \item[Usage :] $getLambda()$
    \item[Arguments :] none
    \item[Value :]  a real value, the $\lambda$ parameter of the  distribution
    \end{description}
    \bigskip

  \item $getMu$
    \begin{description}
    \item[Usage :] $getMu()$
    \item[Arguments :] none
    \item[Value :]  a real value, the $\mu$ parameter of the  distribution
    \end{description}
    \bigskip

  \end{description}

\item[Details :]  \rule{0pt}{1em}
  \begin{description}
  \item density probability function :
    $$
    f(x) = \displaystyle \left(\frac{\lambda}{2\pi x^3} \right)^{1/2}e^{-\lambda(x-\mu)^2/(2\mu^2x)} \mathbf{1}_{x>0}
    $$
  \end{description}

\item[Links :]  \rule{0pt}{1em}
  \href{OpenTURNS_ReferenceGuide.pdf}{see Reference Guide - Standard parametric models}
\end{description}

Each  $getMethod$  is associated to a $setMethod$.




% =============================================================
\newpage \subsubsection{Laplace}

This class inherits from the Distribution class.

\begin{description}

\item[Usage :]  \rule{0pt}{1em}
  \begin{description}
  \item Main parameters set : $Laplace(\lambda ,\mu)$
  \item Default construction : $Laplace( )$
  \end{description}

\item[Arguments :]  \rule{0pt}{1em}
  \begin{description}
  \item $\lambda$ :  a real value, scale parameter, constraint $\lambda>0$,
  \item $\mu$ : a real value, mean value.
  \end{description}

\item[Value :] Laplace. In the default construction, we use the $Laplace(\lambda, \mu) = Laplace(1.0, 0.0)$ definition.

\item[Some methods :] \rule{0pt}{1em}
  \begin{description}

  \item $getLambda$
    \begin{description}
    \item[Usage :] $getLambda()$
    \item[Arguments :] none
    \item[Value :]  a real value, the  $\lambda$ parameter of the considered distribution
    \end{description}
    \bigskip

  \item $getMu$
    \begin{description}
    \item[Usage :] $getMu()$
    \item[Arguments :] none
    \item[Value :]  a real value, the  $\mu$ parameter of the considered distribution
    \end{description}
    \bigskip
  \end{description}

\item[Details :]  \rule{0pt}{1em}
  \begin{description}
  \item density function :
    $$
    f(x) = \frac{\lambda}{2}e^{-\lambda|x-\mu|}
    $$
  \end{description}

\item[Links :]  \rule{0pt}{1em}
  \href{OpenTURNS_ReferenceGuide.pdf}{see Reference Guide - Standard parametric models}
\end{description}


Each  $getMethod$  is associated to a $setMethod$.

% =============================================================
\newpage \subsubsection{Logistic}

This class inherits from the Distribution class.

\begin{description}

\item[Usage :]  \rule{0pt}{1em}
  \begin{description}
  \item Main parameters set : $Logistic(\alpha ,\beta)$
  \item Default construction : $Logistic( )$
  \end{description}

\item[Arguments :]  \rule{0pt}{1em}
  \begin{description}
  \item $\alpha$ :  a real value, mean value
  \item $\beta$ :a  real value, scale parameter,
    constraint : $\beta \geq 0$
  \end{description}

\item[Value :] Logistic. In the default construction, we use the $Logistic(\alpha, \beta) = Logistic(0.0, 1.0)$ definition.

\item[Some methods :] \rule{0pt}{1em}
  \begin{description}

  \item $getAlpha$
    \begin{description}
    \item[Usage :] $getAlpha()$
    \item[Arguments :] none
    \item[Value :]  a real value, the  $\alpha$ parameter of the considered distribution
    \end{description}
    \bigskip

  \item $getBeta$
    \begin{description}
    \item[Usage :] $getBeta()$
    \item[Arguments :] none
    \item[Value :]  a real value, the  $\beta$ parameter of the considered distribution
    \end{description}
    \bigskip
  \end{description}

\item[Details :]  \rule{0pt}{1em}
  \begin{description}
  \item density function :
    $$
    f(x) = \frac{e^{\left(\frac{x-\alpha}{\beta}\right)}}
    {\beta\left(1+ e^{\left(\frac{x-\alpha}{\beta}\right)}\right)^2}\boldsymbol{1}_{[\alpha,+\infty[}(x)
    $$
  \item relation between parameters set :
    \begin{eqnarray*}
      \mu                                        &      =       &   \alpha              \\
      \sigma     &  =   &       \sqrt{\frac{1}{3}\pi^2\beta^2}
    \end{eqnarray*}
    \begin{align*}
      \mbox{where}
      &&
      \mu = \Expect{X}
      &&
      \sigma = \sqrt{\Var X }
    \end{align*}
  \end{description}

\item[Links :]  \rule{0pt}{1em}
  \href{OpenTURNS_ReferenceGuide.pdf}{see Reference Guide - Standard parametric models}
\end{description}


Each  $getMethod$  is associated to a $setMethod$.

% =============================================================
\newpage \subsubsection{LogNormal}

This class inherits from the Distribution class.

\begin{description}

\item[Usage :] \rule{0pt}{1em}
  \begin{description}
  \item Main parameters set : $LogNormal(\mu_\ell ,\sigma_\ell,\gamma)$
  \item Second parameters set : $LogNormal( \mu, \sigma,\gamma, LogNormal.MUSIGMA)$
  \item Third parameters set : $LogNormal( \mu, \sigma/\mu ,\gamma, LogNormal.MU_SIGMAOVERMU)$
  \item Default construction : $LogNormal( )$
  \end{description}

\item[Arguments :]  \rule{0pt}{1em}
  \begin{description}
  \item $\mu_\ell$ :  a real value, mean value of $\log(X)$,
  \item $\sigma_\ell$ : a real value, standard deviation  of $\log(X)$, constraint : $\sigma_\ell>0$
  \item $\gamma$ : a real value
  \item $\mu$ : a real value, mean value, constraint : $\mu > \gamma$
  \item $\sigma$ : a real value, standard deviation, constraint : $\sigma > 0 $
  \end{description}

\item[Value :] a LogNormal . In the default construction, we use the $LogNormal(mu_\ell, sigma_\ell,gamma) = LogNormal(0.0, 1.0, 0.0)$ definition.

\item[Some methods :] \rule{0pt}{1em}
  \begin{description}

  \item $getGamma$
    \begin{description}
    \item[Usage :] $getGamma()$
    \item[Arguments :] none
    \item[Value :]  a real value, the $\gamma$ parameter of the LogNormal distribution
    \end{description}
    \bigskip
  \item $getMu$
    \begin{description}
    \item[Usage :] $getMu()$
    \item[Arguments :] none
    \item[Value :]  a real value,  the $\mu$ parameter of the  LogNormaldistribution
    \end{description}
    \bigskip
  \item $getMuLog$
    \begin{description}
    \item[Usage :] $getMuLog()$
    \item[Arguments :] none
    \item[Value :]  a real value,  the $\mu_\ell$ parameter of the LogNormal distribution
    \end{description}
    \bigskip
  \item $getSigma$
    \begin{description}
    \item[Usage :] $getSigma()$
    \item[Arguments :] none
    \item[Value :]  a real value,  the $\sigma$ parameter of the LogNormal distribution
    \end{description}
    \bigskip
  \item $getSigmaLog$
    \begin{description}
    \item[Usage :] $getSigmaLog()$
    \item[Arguments :] none
    \item[Value :]  a real value,  the $\sigma_\ell$ parameter of the  LogNormal distribution
    \end{description}
    \bigskip

  \item $getSigmaOverMu$
    \begin{description}
    \item[Usage :] $getSigmaOverMu()$
    \item[Arguments :] none
    \item[Value :]  a real value, the  $\sigma/\mu$ parameter of the considered distribution
    \end{description}
  \end{description}

\item[Details :]  \rule{0pt}{1em}
  \begin{description}
  \item density probability function :
    $$
    f(x) = \frac{1}{\sqrt{2\pi}\sigma_\ell(x-\gamma)}\;
    e^{-\frac{1}{2}\left(\frac{\log(x-\gamma)-\mu_\ell}{\sigma_\ell}\right)^2}
    \boldsymbol{1}_{[\gamma,+\infty[}(x)
    $$
  \item relation between parameters set :
    \begin{eqnarray*}
      \mu &     =       &   e^{\mu_\ell + \sigma_\ell^2/2} + \gamma             \\
      \sigma     &  =   &  e^{\mu_\ell + \sigma_\ell^2/2} \sqrt{ \left( e^{\sigma_\ell^2} -1 \right)}
    \end{eqnarray*}
    \begin{align*}
      \mbox{where}
      &&
      \mu = \Expect{X}
      &&
      \sigma = \sqrt{\Var X }
    \end{align*}
  \end{description}

\item[Links :]  \rule{0pt}{1em}
  \href{OpenTURNS_ReferenceGuide.pdf}{see Reference Guide - Standard parametric models}
\end{description}

Each  $getMethod$  is associated to a $setMethod$.

% =============================================================
\newpage \subsubsection{LogUniform}

This class inherits from the Distribution class.

\begin{description}

\item[Usage :] \rule{0pt}{1em}
  \begin{description}
  \item Main parameters set : $LogUniform(a_\ell ,b_\ell)$
  \item Default construction : $LogUniform( )$
  \end{description}

\item[Arguments :]  \rule{0pt}{1em}
  \begin{description}
  \item $a_\ell$ : a real value, lower bound of $\log(X)$,
  \item $b_\ell$ : a real value, upper bound of $\log(X)$, constraint : $b_\ell>a_\ell$
  \end{description}

\item[Value :] a LogUniform : the $\log$ of the variate is $Uniform(a_\ell ,b_\ell)$ distributed. In the default construction, we use the $LogUniform(a_\ell, b_\ell) = LogUniform(-1.0, 1.0)$ definition.

\item[Some methods :] \rule{0pt}{1em}
  \begin{description}
  \item $getALog$
    \begin{description}
    \item[Usage :] $getALog()$
    \item[Arguments :] none
    \item[Value :]  a real value,  the $a_\ell$ parameter of the LogUniform distribution
    \end{description}
    \bigskip
  \item $getBLog$
    \begin{description}
    \item[Usage :] $getBLog()$
    \item[Arguments :] none
    \item[Value :]  a real value,  the $b_\ell$ parameter of the LogUniform distribution
    \end{description}
    \bigskip

  \item[Details :]  \rule{0pt}{1em}
    \begin{description}
    \item density probability function :
      $$
      f(x) = \frac{1}{x(b_\ell - a_\ell)}\boldsymbol{1}_{[a_\ell, b_\ell]}(\log(x))
      $$
  \item relation between parameters set :
    \begin{eqnarray*}
      \mu &     =       &  \displaystyle  \frac{e^{b_\ell} - e^{a_\ell}}{b_\ell - a_\ell}       \\
      \sigma^2     &  =   &  \displaystyle \frac{1}{2} \frac{(e^{b_\ell} - e^{a_\ell}) \left[ e^{b_\ell}(b_\ell - a_\ell -2) +  e^{a_\ell}(b_\ell - a_\ell +2)\right]}{(b_\ell - a_\ell)^2}
    \end{eqnarray*}
    \begin{align*}
      \mbox{where}
      &&
      \mu = \Expect{X}
      &&
      \sigma = \sqrt{\Var X }
    \end{align*}
    \item[Links :]  \rule{0pt}{1em}
      \href{OpenTURNS_ReferenceGuide.pdf}{see Reference Guide - Standard parametric models}
    \end{description}
  \end{description}
\end{description}

Each  $getMethod$  is associated to a $setMethod$.

% =============================================================
\newpage \subsubsection{Multinomial}

This class inherits from the Distribution class.

\begin{description}

\item[Usage :] Main parameters set : $Multinomial(p, N)$

\item[Arguments :]  \rule{0pt}{1em}
  \begin{description}
  \item $p$ :  NumericalPoint of dimension $n$,
    constraint : $0\leq p_i \leq 1$, $\displaystyle q = \sum_{i=1}^n p_i \leq 1$
  \item $N$ :  an integer,
  \end{description}

\item[Value :] a Multinomial

\item[Some methods :] \rule{0pt}{1em}
  \begin{description}

  \item $getN$
    \begin{description}
    \item[Usage :] $getN()$
    \item[Arguments :] none
    \item[Value :]  a integer, the  $N$ parameter of the considered distribution
    \end{description}
    \bigskip

  \item $getP$
    \begin{description}
    \item[Usage :] $getP()$
    \item[Arguments :] none
    \item[Value :]  a NumericalPoint, the  $p$ parameter of the considered distribution
    \end{description}
    \bigskip

  \item $getSupport$
    \begin{description}
    \item[Usage :] $getSupport(interval)$
    \item[Arguments :] $interval$ : a $Interval$, an interval in $\mathbb{R}$
    \item[Value :]  a $NumericalSample$, all the points (here of dimension $n$) of the distribution range which are included in the interval $interval$.
    \end{description}
  \end{description}

\item[Details :]  \rule{0pt}{1em}
  \begin{description}
  \item probability function :
    $$
    \displaystyle P(\vect{X} = \vect{x}) = \frac{N!}{x_1!\dots x_n! (N-s)!}p_1^{x_1}\dots p_n^{x_n}(1-q)^{N-s}
    $$ with $0\leq p_i \leq 1$, $x_i\in \mathbb{N}$, $\displaystyle q = \sum_{i=1}^n p_i \leq 1$, $s=  \sum_{i=1}^n x_i \leq N$
  \item relation between parameters set :
    \begin{eqnarray*}
      \mu_i                                     & =     &   n \, p_i            \\
      \sigma_i                          & =     &               \sqrt{n\, p_i \,(1-p_i) }\\
      \sigma_{i,j}              & =     &       -n \,p_i \, p_j
    \end{eqnarray*}
    \begin{align*}
      \mbox{where}
      &&
      \mu_i = \Expect{X_i}
      &&
      \sigma_i = \sqrt{\Var X_i }
      &&
      \sigma_{i,j} = \Cov(X_i,X_j)
    \end{align*}
  \end{description}

\item[Links :]  \rule{0pt}{1em}
  \href{OpenTURNS_ReferenceGuide.pdf}{see Reference Guide - Standard parametric models}
\end{description}

Each  $getMethod$  is associated to a $setMethod$.

% ==============================================
\newpage \subsubsection{NegativeBinomial}

This class inherits from the Distribution class.

\begin{description}

\item[Usage :]\rule{0pt}{1em}
  \begin{description}
  \item Main parameters set : $NegativeBinomial(r,p)$
  \item Default construction : $NegativeBinomial()$
  \end{description}

\item[Arguments :]  \rule{0pt}{1em}
  \begin{description}
  \item $r$ : a real value $>0$,
  \item  $p$ : a real value such as $0 < p < 1$.
  \end{description}

\item[Value :] a NegativeBinomial. In the default construction, we use the $NegativeBinomial() = NegativeBinomial(1,0.5)$ definition.

\item[Some methods :] \rule{0pt}{1em}
  \begin{description}

  \item $getP$
    \begin{description}
    \item[Usage :] $getP()$
    \item[Arguments :] none
    \item[Value :]  a real value in $(0, 1)$, the $p$ parameter of the distribution.
    \end{description}
    \bigskip

  \item $getR$
    \begin{description}
    \item[Usage :] $getR()$
    \item[Arguments :] none
    \item[Value :]  a positive real value, the $r$ parameter of the distribution.
    \end{description}
    \bigskip

  \item $getSupport$
    \begin{description}
    \item[Usage :] $getSupport(interval)$
    \item[Arguments :] $interval$ : a $Interval$, an interval in $\mathbb{R}$
    \item[Value :]  a $NumericalSample$, all the points (here of dimension 1) of the distribution range which are included in the interval $interval$.
    \end{description}

  \end{description}

\item[Details :]  \rule{0pt}{1em}
  \begin{description}
  \item probability distribution:
    $$
    \Prob k  = \frac{\Gamma(k + r)}{\Gamma(r)\Gamma(k+1)}p^k(1-p)^r, \, \forall k \in \mathbb{N}
    $$
  \item relation between parameters set :
    \begin{eqnarray*}
      \mu  =  \frac{rp}{1-p}               & \mbox{where}& \mu =\Expect{X} \\
      \sigma  = \frac{\sqrt{rp}}{1-p}  & \mbox{where}& \sigma =\sqrt{\Var X }
    \end{eqnarray*}

  \end{description}
  \bigskip

\item[Links :]  \rule{0pt}{1em}
  \href{OpenTURNS_ReferenceGuide.pdf}{see Reference Guide - Standard parametric models}
\end{description}


Each  $getMethod$  is associated to a $setMethod$.




% =============================================================
\newpage \subsubsection{NonCentralChiSquare}

This class inherits from the Distribution class.

\begin{description}

\item[Usage :] \rule{0pt}{1em}
  \begin{description}
  \item Main parameters set : $NonCentralChiSquare(\nu, \lambda)$
  \item Default construction : $NonCentralChiSquare( )$
  \end{description}

\item[Arguments :]  \rule{0pt}{1em}
  \begin{description}
  \item $\nu$ :  a real positive value,  constraint : $\nu > 0$
  \item $\lambda$ :  a real value, constraint : $\lambda \geq 0$
  \end{description}

\item[Value :] a NonCentralChiSquare distribution. In the default construction, we use the $NonCentralChiSquare(\nu, \lambda) = NonCentralChiSquare(5, 0)$ definition.

\item[Some methods :] \rule{0pt}{1em}
  \begin{description}

  \item $getNu$
    \begin{description}
    \item[Usage :] $getNu()$
    \item[Arguments :] none
    \item[Value :]  a real value,  the $\mu$ parameter of the NonCentralChiSquare distribution
    \end{description}
    \bigskip

  \item $getLambda$
    \begin{description}
    \item[Usage :] $getLambda()$
    \item[Arguments :] none
    \item[Value :]  a real value,  the $\lambda$ parameter of the NonCentralChiSquare distribution
    \end{description}
    \bigskip
  \end{description}

\item[Details :]  \rule{0pt}{1em}
  \begin{description}
  \item density function :
    $$
    f(x) = \displaystyle \sum_{j=0}^{\infty} e^{-\lambda}\frac{\lambda^j}{j!}p_{\chi^2(\nu+2j)}(x)
    $$
    where $p_{\chi^2(q)}$ is the probability density function of a $\chi^2(q)$ random variate.
  \end{description}

\item[Links :]  \rule{0pt}{1em}
  \href{OpenTURNS_ReferenceGuide.pdf}{see Reference Guide - Standard parametric models}
\end{description}


Each  $getMethod$  is associated to a $setMethod$.

% =============================================================
\newpage \subsubsection{NonCentralStudent}

This class inherits from the Distribution class.

\begin{description}

\item[Usage :] \rule{0pt}{1em}
  \begin{description}
  \item Main parameters set : $NonCentralStudent(\nu,\delta, \gamma)$
  \item Main parameters set : $NonCentralStudent(\nu,\delta)$
  \item Main parameters set : $NonCentralStudent(\nu)$
  \item Default construction : $NonCentralStudent( )$
  \end{description}

\item[Arguments :]  \rule{0pt}{1em}
  \begin{description}
  \item $\nu$ :  a real positive value, generalised number degree of freedom, constraint : $\nu > 0$
  \item $\delta$ :  a real value, the non-centrality parameter
  \item $\gamma$ :  a real value, the shift from the origin
  \end{description}

\item[Value :] a NonCentralStudent distribution. In the default construction, we use the $NonCentralStudent(\nu, \delta, \gamma) = NonCentralStudent(0.5, 0.0, 0.0)$ definition, and in the alternative constructions we use $NonCentralStudent(\nu, \delta) = NonCentralStudent(\nu, \delta, 0.0)$ which is the classical non-central Student distribution and $NonCentralStudent(\nu) = NonCentralStudent(\nu, 0.0, 0.0)$ which is the classical Student distribution.

\item[Some methods :] \rule{0pt}{1em}
  \begin{description}

  \item $getNu$
    \begin{description}
    \item[Usage :] $getNu()$
    \item[Arguments :] none
    \item[Value :]  a real value,  the $\mu$ parameter of the NonCentralStudent distribution
    \end{description}
    \bigskip
  \item $getDelta$
    \begin{description}
    \item[Usage :] $getDelta()$
    \item[Arguments :] none
    \item[Value :]  a real value,  the $\delta$ parameter of the NonCentralStudent distribution
    \end{description}
    \bigskip
  \item $getGamma$
    \begin{description}
    \item[Usage :] $getGamma()$
    \item[Arguments :] none
    \item[Value :]  a real value,  the $\gamma$ parameter of the NonCentralStudent distribution
    \end{description}
    \bigskip
  \end{description}

\item[Details :]  \rule{0pt}{1em}
  \begin{description}
  \item density function :
    $$
    f(x) =\frac{e^{(-\delta^2 / 2)}}{\sqrt{\nu\pi} \Gamma(\nu / 2)}\left(\frac{\nu}{\nu + (x-\gamma)^2}\right) ^ {(\nu + 1) / 2} \sum_{j=0}^{\infty} \frac{\Gamma\left(\frac{\nu + j + 1}{2}\right)}{\Gamma(j + 1)}\left(\delta(x-\gamma)\sqrt{\frac{2}{\nu + (x-\gamma)^2}}\right) ^ j
    $$
  \item relation between parameters set :
    \begin{eqnarray*}
      \sigma            &  =    &       \sqrt{\frac{\nu}{\nu-2} }
    \end{eqnarray*}
    \begin{align*}
      \mbox{where}
      &&
      \mu = \Expect{X}
      &&
      \sigma = \sqrt{\Var X }
    \end{align*}
  \end{description}

\item[Links :]  \rule{0pt}{1em}
  \href{OpenTURNS_ReferenceGuide.pdf}{see Reference Guide - Standard parametric models}
\end{description}


Each  $getMethod$  is associated to a $setMethod$.



% ===================================================
\newpage \subsubsection{Normal}


\begin{description}
\item[Usage :] \rule{0pt}{1em}
  \begin{description}
  \item $Normal(mean,standardDeviation)$
  \item $Normal(dim)$
  \item $Normal(\mu,\sigma,R)$
  \item $Normal(\mu,\Sigma)$
  \item Default construction : $Normal( )$
  \end{description}

\item[Arguments :]  \rule{0pt}{1em}
  \begin{description}
  \item $mean$ : a scalar, the mean value of the 1D normal distribution
  \item $standardDeviation$ : a scalar, the standard deviation value of the 1D normal distribution
  \item $dim$, an integer : the dimension of the Normal distribution
  \item $\mu$ : a NumericalPoint, the mean of the Distribution
  \item $\sigma$ : a NumericalPoint, the standard deviation of each component, constraint : $\sigma[i]>0, \forall i $
  \item $R$ : a CorrelationMatrix, the linear correlation matrix of the Normal distribution
  \item $\Sigma$ : a CovarianceMatrix, the covariance matrix of the Normal distribution
  \end{description}

\item[value :]  \rule{0pt}{1em}
  \begin{description}
  \item while using the first usage, a 1D normal distribution with $mean$ as mean value, $standardDeviation$ as standard deviation
  \item while using the second  usage, a normal distribution of dimension $dim$, with $\vect{0}$ mean value vector, $\vect{1}$-standard deviation vector and identity correlation matrix
  \item  while using the third  usage, a nD normal distribution with $\mu$ as mean vector, $\sigma$ as standard deviation vector and $R$ as linear correlation matrix
  \item  while using the fourth  usage, a nD normal distribution with $\mu$ as mean vector, $\Sigma$ as covariance matrix
  \item  while using the defalt  usage, a 1D normal distribution with O mean and unit variance.
  \end{description}

\item[Some methods :] \rule{0pt}{1em}
  \begin{description}
  \item $getMu$
    \begin{description}
    \item[Usage :] $getMu()$
    \item[Arguments :] none
    \item[Value :]  a NumericalPoint,  the $\mu$ parameter of the  distribution
    \end{description}
    \bigskip
  \item $getSigma$
    \begin{description}
    \item[Usage :] $getSigma()$
    \item[Arguments :] none
    \item[Value :]  a NumericalPoint,  the $\sigma$ parameter of the  distribution
    \end{description}
    \bigskip
  \item $getCorrelationMatrix$
    \begin{description}
    \item[Usage :] $getCorrelationMatrix()$
    \item[Arguments :] none
    \item[Value :]  a CorrelationMatrix,  the $R$ parameter of the  distribution
    \end{description}
  \end{description}

\item[Details :]
  \begin{description}
  \item probability density function :
    $$\displaystyle
    \frac{1}
    {
      \displaystyle (2\pi)^{\frac{n}{2}}(\mathrm{det}\mathbf{\Sigma})^{\frac{1}{2}}
    }
    \displaystyle e^{-\frac{1}{2}\Tr{(x-\mu)}\mathbf{\Sigma}^{-1}(x-\mu)}
    $$
    with $\Sigma = \Lambda(\sigma) R \Lambda(\sigma)$, $\Lambda(\sigma) = diag(\sigma)$, $R$ symmetric, definite and positive, $\sigma_i >0$.

  \end{description}

\item[Links :]  \rule{0pt}{1em}
  \href{OpenTURNS_ReferenceGuide.pdf}{see Reference Guide - Standard parametric models}

\end{description}


Each  $getMethod$  is associated to a $setMethod$.

% =============================================================
\newpage \subsubsection{Poisson}

This class inherits from the Distribution class.

\begin{description}

\item[Usage :] Main parameters set : $Poisson(\lambda)$

\item[Arguments :]  $\lambda$ :  real value, mean and variance value, constraint : $\lambda>0$

\item[Value :] Poisson

\item[Some methods :]  \rule{0pt}{1em}
  \begin{description}

  \item $getLambda$
    \begin{description}
    \item[Usage :] $getLambda()$
    \item[Arguments :] none
    \item[Value :]  a real positive value, the  $\lambda$ parameter of the considered distribution
    \end{description}
    \bigskip
  \end{description}

\item[Details :]  \rule{0pt}{1em}
  \begin{description}
  \item probability function :
    $$
    \Prob{k} =
    \frac{\lambda^k}{k!}\;e^{-\lambda}, \,  k \in \mathbb{N}
    $$
  \item relation between parameters set :
    \begin{eqnarray*}
      \mu                                       &       =       &   \lambda             \\
      \sigma                            &  =    &       \sqrt{\lambda }
    \end{eqnarray*}
    \begin{align*}
      \mbox{where}
      &&
      \mu = \Expect{X}
      &&
      \sigma = \sqrt{\Var X }
    \end{align*}
  \end{description}

\item[Links :]  \rule{0pt}{1em}
  \href{OpenTURNS_ReferenceGuide.pdf}{see Reference Guide - Standard parametric models}
\end{description}

Each  $getMethod$  is associated to a $setMethod$.

% =============================================================
\newpage \subsubsection{Rayleigh}

This class inherits from the Distribution class.

\begin{description}

\item[Usage :] Main parameters set : $Rayleigh(\sigma, \gamma)$

\item[Arguments :]  \rule{0pt}{1em}
  \begin{description}
  \item $\sigma$ :  a real positive value, constraint : $\sigma > 0$
  \item $\gamma$ :  a real value
  \end{description}

\item[Value :] Rayleigh. In the default construction, we use the $Rayleigh(\sigma, \gamma) = Rayleigh(1.0, 0.0)$ definition.

\item[Some methods :]  \rule{0pt}{1em}
  \begin{description}

  \item $getSigma$
    \begin{description}
    \item[Usage :] $getSigma()$
    \item[Arguments :] none
    \item[Value :]  a real positive value, the  $\sigma$ parameter of the considered distribution
    \end{description}
    \bigskip
  \item $getGamma$
    \begin{description}
    \item[Usage :] $getGamma()$
    \item[Arguments :] none
    \item[Value :]  a real value, the  $\gamma$ parameter of the considered distribution
    \end{description}
    \bigskip
  \end{description}

\item[Details :]  \rule{0pt}{1em}
  \begin{description}
  \item probability function :
    $$
    f(x) = \frac{(x - \gamma)}{\sigma^2}e^{-\frac{(x-\gamma)^2}{2\sigma^2}}\boldsymbol{1}_{[\gamma,+\infty[}(x)
    $$
  \end{description}

\item[Links :]  \rule{0pt}{1em}
  \href{OpenTURNS_ReferenceGuide.pdf}{see Reference Guide - Standard parametric models}
\end{description}

Each  $getMethod$  is associated to a $setMethod$.

% =============================================================
\newpage \subsubsection{Rice}

This class inherits from the Distribution class.

\begin{description}

\item[Usage :] Main parameters set : $Rice(\sigma, \nu)$

\item[Arguments :]  \rule{0pt}{1em}
  \begin{description}
  \item $\sigma$ :  a real positive value, constraint : $\sigma > 0$
  \item $\nu$ :  a real value, constraint : $\nu \geq 0$
  \end{description}

\item[Value :] Rice. In the default construction, we use the $Rice( \sigma, \nu) = Rice(1, 0)$ definition.

\item[Some methods :]  \rule{0pt}{1em}
  \begin{description}

  \item $getSigma$
    \begin{description}
    \item[Usage :] $getSigma()$
    \item[Arguments :] none
    \item[Value :]  a real positive value, the  $\sigma$ parameter of the considered distribution
    \end{description}
    \bigskip
  \item $getNu$
    \begin{description}
    \item[Usage :] $getNu()$
    \item[Arguments :] none
    \item[Value :]  a real value, the  $\nu$ parameter of the considered distribution
    \end{description}
    \bigskip
  \end{description}

\item[Details :]  \rule{0pt}{1em}
  \begin{description}
  \item probability function :
    $$
    f(x) =  \displaystyle 2\frac{x}{\sigma^2}p_{\chi^2(2,\frac{\nu^2}{\sigma^2})}(\frac{x^2}{\sigma^2})
    $$
    where $p_{\chi^2(\nu, \lambda)}$ is the probability density function of a Non Central Chi Square distribution.

  \end{description}

\item[Links :]  \rule{0pt}{1em}
  \href{OpenTURNS_ReferenceGuide.pdf}{see Reference Guide - Standard parametric models}
\end{description}

Each  $getMethod$  is associated to a $setMethod$.

% =============================================================
\newpage \subsubsection{Student}

This class inherits from the Distribution class.

\begin{description}

\item[Usage :] \rule{0pt}{1em}
  \begin{description}
  \item $Student(\nu, \vect{\mu}, \vect{\sigma}, \mat{R})$
  \item $Student(\nu)$
  \item $Student(\nu, \mu)$
  \item $Student(\nu, \mu, \sigma)$
  \item $Student(\nu, d)$
  \item Default construction : $Student()$
  \end{description}

\item[Arguments :]  \rule{0pt}{1em}
  \begin{description}
  \item $\nu$ : a real positive value, generalised number degree of freedom, constraint : $\nu > 0$.
  \item $\vect{\mu}$ ($\mu$) : a NumericalPoint (a real value), the mean value of the distribution if $\nu>1$, its location parameter for $\nu\leq 1$.
  \item $\vect{\sigma}$ ($\sigma$) : a NumericalPoint (a real value), the scale parameter of the distribution, constraint : $\sigma_i > 0$.
  \item $\mat{R}$ : a CorrelationMatrix, the correlation matrix of the distribution if $\nu>2$, its generalized correlation matrix for $\nu\leq 2$.
  \item $d$ : an integer value, the dimension of the distribution, constraint: $d \geq 1$.
  \end{description}

\item[Value :] a Student distribution. In the simplified constructions where the dimension $d$ is not specified (usages number 2 to 5) and in the default construction, we use $d=1$, $\nu=3$, $\mu=0$, $\sigma=1$. In the last construction where the dimension $d$d is specified, we use $\vect{\mu}= (1, \dots, 1) \in \mathbb{R}^d$, $\vect{\sigma}= (1, \dots, 1) \in \mathbb{R}^d$ and $\mat{R} = \mat{Id}(d) \in \mathbb{R}^d \times \mathbb{R}^d$.

\item[Some methods :] \rule{0pt}{1em}
  \begin{description}

  \item $getMu$
    \begin{description}
    \item[Usage :] $getMu()$
    \item[Arguments :] none
    \item[Value :]  a real value, the $\mu$ parameter of the Student distribution. Only defined when the dimension is 1 (else, use the getMEan() method inherited from the EllipticalDistribution class).
    \end{description}
    \bigskip
  \item $getNu$
    \begin{description}
    \item[Usage :] $getNu()$
    \item[Arguments :] none
    \item[Value :]  a real value, the generalized number of degrees of freedom of the Student distribution;
    \end{description}
    \bigskip
  \end{description}

\item[Details :]  \rule{0pt}{1em}
  \begin{description}
  \item density function :
    $$
    f(\vect{x}) = \frac{\Gamma\left(\frac{\nu+d}{2}\right)}
    {(\pi d)^{\frac{d}{2}}\Gamma\left(\frac{\nu}{2}\right)}\frac{\left|\mathrm{det}(\mat{R})\right|^{-1/2}}{\prod_{k=1}^d\sigma_k}\left(1+\frac{\vect{z}^t\mat{R}^{-1}\vect{z}}{\nu}\right)^{-\frac{\nu+d}{2}}
    $$
    where $\vect{z}=\mat{\Delta}^{-1}\left(\vect{x}-\vect{\mu}\right)$ with $\mat{\Delta}=\mat{\mathrm{diag}}(\vect{\sigma})$.
  \item relation between parameters and moments :
    \begin{eqnarray*}
      \vect{\Expect{X}} & = & \vect{\mu}\mbox{ if }\nu>1\\
      \mat{\Cov{X}} & = & \mat{\Delta}^t\,\mat{R}\,\mat{\Delta}\mbox{ if }\nu>2
    \end{eqnarray*}
  \end{description}

\item[Links :]  \rule{0pt}{1em}
  \href{OpenTURNS_ReferenceGuide.pdf}{see Reference Guide - Standard parametric models}
\end{description}


Each  $getMethod$  is associated to a $setMethod$.

% =============================================================
\newpage \subsubsection{Trapezoidal}

This class inherits from the Distribution class.

\begin{description}

\item[Usage :] \rule{0pt}{1em}
  \begin{description}
  \item Main parameters set : $Trapezoidal(a,b,c,d)$
  \item Default construction : $Trapezoidal( )$
  \end{description}

\item[Arguments :]  \rule{0pt}{1em}
  \begin{description}
  \item $a$ : a real value, the lower bound
  \item $b$ : a real value, the level start
  \item $c$ : a real value, the level end
  \item $d$ : a real value, the upper bound, constraints: $a\leq b < c\leq d$
  \end{description}

\item[Value :]  Trapezoidal. In the default construction, we use the $Trapezoidal(a,b,c,d) = Trapezoidal(-2.0,-1.0,1.0, 2.0)$ definition.

\item[Some methods :] \rule{0pt}{1em}
  \begin{description}

  \item $getA$
    \begin{description}
    \item[Usage :] $getA()$
    \item[Arguments :] none
    \item[Value :]  a real value,  the $a$ parameter of the Trapezoidal distribution
    \end{description}
    \bigskip

  \item $getB$
    \begin{description}
    \item[Usage :] $getB()$
    \item[Arguments :] none
    \item[Value :]  a real value,  the $b$ parameter of the Trapezoidal distribution
    \end{description}

  \item $getC$
    \begin{description}
    \item[Usage :] $getC()$
    \item[Arguments :] none
    \item[Value :]  a real value,  the $c$ parameter of the Trapezoidal distribution
    \end{description}

  \item $getD$
    \begin{description}
    \item[Usage :] $getD()$
    \item[Arguments :] none
    \item[Value :]  a real value,  the $d$ parameter of the Trapezoidal distribution
    \end{description}

  \end{description}

\item[Details :]  \rule{0pt}{1em}
  \begin{description}
  \item density function :
    $$
    f_X(x;\vect{\theta}) = \left\{
      \begin{array}{ll}
        h \frac {x-a}{b-a} & \textrm{if}\ a\leq x < b \\
        h & \textrm{if}\ b\leq x < c \\
        h \frac{d-x}{d-c}& \textrm{if}\ c\leq x < d \\
        0 & \textrm{otherwise}
      \end{array}
    \right.
    $$
    with:
    $h=\frac{2}{d+c-a-b}$

  \item relation between parameter sets :
    \begin{eqnarray*}
      \mu           &   = &   \frac{h}{6}(d^2 + cd + c^2 - b^2 - ab - a^2)  \\
      \sigma^2        &  =  &  \frac{h^2}{72}(d^4 + 2cd^3 - 3bd^3 - 3ad^3 - 3bcd^2 - 3acd^2 + ...\\
         & & ... + 4b^2d^2 + 4abd^2 + 4a^2d^2 + 2c^3d - 3bc^2d - 3ac^2d   \\
      & & + 4b^2cd + 4abcd + 4a^2cd - 3b^3d - 3ab^2d - ...\\
         & & ... 3a^2bd - 3a^3d + c^4 - 3bc^3 - 3ac^3 +4b^2c^2 + 4abc^2   \\
      & & + 4a^2c^2 - 3b^3c - 3ab^2c - 3a^2bc - ...\\
         & & ... 3a^3c + b^4 + 2ab^3 + 2a^3b + a^4)
    \end{eqnarray*}
    \begin{align*}
      \mbox{where}
      &&
      \mu = \Expect{X}
      &&
      \sigma^2 = \Var X
    \end{align*}
  \end{description}
  \bigskip

\item[Links :]  \rule{0pt}{1em}
  \href{./Version/docref_B121_DistributionSelection.pdf}{see docref\_B121\_DistributionSelection}
\end{description}


Each  $getMethod$  is associated to a $setMethod$.




% =============================================================
\newpage \subsubsection{Triangular}

This class inherits from the Distribution class.

\begin{description}

\item[Usage :] \rule{0pt}{1em}
  \begin{description}
  \item Main parameters set : $Triangular(a,m,b)$
  \item Default construction : $Triangular( )$
  \end{description}

\item[Arguments :]  \rule{0pt}{1em}
  \begin{description}
  \item $a$ : a  real value, the lower bound
  \item $b$ : a real value, the upper bound, constraint : $b\geq a$
  \item $m$ : a real value, the mode, constant, $b\geq m \geq a$
  \end{description}

\item[Value :] Triangular. In the default construction, we use the $Triangular(a, m, b) = Triangular(-1.0, 0.0, 1.0)$ definition.

\item[Some methods :] \rule{0pt}{1em}
  \begin{description}

  \item $getA$
    \begin{description}
    \item[Usage :] $getA()$
    \item[Arguments :] none
    \item[Value :]  a real value,  the $a$ parameter of the Triangular distribution
    \end{description}
    \bigskip

  \item $getB$
    \begin{description}
    \item[Usage :] $getB()$
    \item[Arguments :] none
    \item[Value :]  a real value,  the $b$ parameter of the Triangular distribution
    \end{description}
    \bigskip

  \item $getM$
    \begin{description}
    \item[Usage :] $getM()$
    \item[Arguments :] none
    \item[Value :]  a real value,  the $m$ parameter of the Triangular distribution
    \end{description}
    \bigskip
  \end{description}

\item[Details :]  \rule{0pt}{1em}
  \begin{description}
  \item density function :
    $$
    f(x) =
    \left\{
      \begin{array}{ll}
        \displaystyle \frac{2(x-a)}{(m-a)(b-a)} & a \leq x \leq m \\
        \displaystyle \frac{2(b-x)}{(b-m)(b-a)} & m \leq x \leq b \\
        0 & \mbox{elsewhere}
      \end{array}
    \right.
    $$

  \item relation between parameters set :
    \begin{eqnarray*}
      \mu                                       &       =       &   \frac{1}{3}\,(a+m+b)        \\
      \sigma                            &  =    &       \sqrt{ \frac{1}{18} (a^2+b^2+m^2-ab-am-bm)}
    \end{eqnarray*}
    \begin{align*}
      \mbox{where}
      &&
      \mu = \Expect{X}
      &&
      \sigma = \sqrt{\Var X }
    \end{align*}
  \end{description}
  \bigskip

\item[Links :]  \rule{0pt}{1em}
  \href{OpenTURNS_ReferenceGuide.pdf}{see Reference Guide - Standard parametric models}
\end{description}


Each  $getMethod$  is associated to a $setMethod$.

% =============================================================
\newpage \subsubsection{TruncatedNormal}

This class inherits from the Distribution class.

\begin{description}

\item[Usage :] \rule{0pt}{1em}
  \begin{description}
  \item Main parameters set : $TruncatedNormal(\mu_n,\sigma_n,a,b)$
  \item Default construction : $TruncatedNormal( )$
  \end{description}

\item[Arguments :]  \rule{0pt}{1em}
  \begin{description}
  \item $\mu_n$ :  a real value which corresponds to the mean of the associated non truncated normal
  \item $\sigma_n$ :  a real value which corresponds to the standard deviation of the associated non truncated normal
  \item $a$ : a real value, the lower bound
  \item $b$ : a real value, the upper bound, constraint : $b\geq a$
  \end{description}

\item[Value :] TruncatedNormal . In the default construction, we use the \\
$TruncatedNormal(mu_n, sigma_n, a, b) = TruncatedNormal(0.0, 1.0, -1.0, 1.0)$ definition.

\item[Some methods :]   \rule{0pt}{1em}
  \begin{description}

  \item $getA$
    \begin{description}
    \item[Usage :] $getA()$
    \item[Arguments :] none
    \item[Value :]  a real value, the $a$ parameter of the TruncatedNormal distribution
    \end{description}
    \bigskip
  \item $getB$
    \begin{description}
    \item[Usage :] $getB()$
    \item[Arguments :] none
    \item[Value :]  a real value, the  $b$ parameter of the TruncatedNormal distribution
    \end{description}
    \bigskip
  \item $getMu$
    \begin{description}
    \item[Usage :] $getMu()$
    \item[Arguments :] none
    \item[Value :]  a real value, the $\mu_n$ parameter of the TruncatedNormal distribution
    \end{description}
    \bigskip
  \item $getSigma$
    \begin{description}
    \item[Usage :] $getSigma()$
    \item[Arguments :] none
    \item[Value :]  a real value, the $\sigma_n$ parameter of the TruncatedNormal distribution
    \end{description}
    \bigskip
  \end{description}

\item[Details :]  \rule{0pt}{1em}
  \begin{description}
  \item probability density function :
    $$
    f(x) =
    \frac{\frac{1}{\sigma_n}\;\phi(\frac{x-\mu_n}{\sigma_n})}
    {\Phi(\frac{b-\mu_n}{\sigma_n}) - \Phi(\frac{a-\mu_n}{\sigma_n})}
    \boldsymbol{1}_{[a, b]}(x)
    $$
    (where $\phi$ and $\Phi$ are, respectively, the probability density distribution function and the cumulative
    distribution function of a standard normal distribution)

  \item relation between parameters set :
    \begin{eqnarray*}
      \mu                                       &       =       &   \mu_n -
      \sigma_n \;
      \frac{\phi(b_{red}) - \phi(a_{red})}
      {\Phi(b_{red}) - \Phi(a_{red})}   \\
      \sigma            &  =    &       \sigma_n
      \left\{
        1
        -
        \frac{b_{red}\,\phi(b_{red}) - a_{red}\,\phi(a_{red})}
        {\Phi(b_{red}) - \Phi(a_{red})}
        -
        \left[
          \frac{\phi(b_{red}) - \phi(a_{red})}
          {\Phi(b_{red}) - \Phi(a_{red})}
        \right]^2
      \right\}^{1/2}
    \end{eqnarray*}
    where
    \begin{eqnarray*}
      a_{red} = \frac{a - \mu_n}{\sigma_n} &&
      b_{red} = \frac{b - \mu_n}{\sigma_n}
    \end{eqnarray*}
    and
    \begin{align*}
      \mu = \Expect{X}
      &&
      \sigma = \sqrt{\Var X }
    \end{align*}
  \end{description}

\item[Links :]  \rule{0pt}{1em}
  \href{OpenTURNS_ReferenceGuide.pdf}{see Reference Guide - Standard parametric models}
\end{description}




Each  $getMethod$  is associated to a $setMethod$.

% =============================================================
\newpage \subsubsection{Uniform}

This class inherits from the Distribution class.

\begin{description}

\item[Usage :] \rule{0pt}{1em}
  \begin{description}
  \item Main parameters set : $Uniform(a,b)$
  \item Default construction : $Uniform( )$
  \end{description}

\item[Arguments :]  \rule{0pt}{1em}
  \begin{description}
  \item $a$ : a  real value, the lower bound
  \item $b$ : a real value, the upper bound, constraint : $b\geq a$
  \end{description}

\item[Value :]  Uniform. In the default construction, we use the $Uniform(a,b) = Uniform(-1.0, 1.0)$ definition.

\item[Some methods :] \rule{0pt}{1em}
  \begin{description}

  \item $getA$
    \begin{description}
    \item[Usage :] $getA()$
    \item[Arguments :] none
    \item[Value :]  a real value,  the $a$ parameter of the Uniform distribution
    \end{description}
    \bigskip

  \item $getB$
    \begin{description}
    \item[Usage :] $getB()$
    \item[Arguments :] none
    \item[Value :]  a real value,  the $b$ parameter of the Uniform distribution
    \end{description}

  \end{description}

\item[Details :]  \rule{0pt}{1em}
  \begin{description}
  \item density function :
    $$
    f(x) =
    \left\{
      \begin{array}{ll}
        \displaystyle \frac{1}{(b-a)} & a \leq x \leq b \\
        0 & \mbox{elsewhere}
      \end{array}
    \right.
    $$

  \item relation between parameters set :
    \begin{eqnarray*}
      \mu                                       &       =       &   \frac{a+b}{2}       \\
      \sigma                            &  =    &       \frac{b-a}{2\sqrt{3}}
    \end{eqnarray*}
    \begin{align*}
      \mbox{where}
      &&
      \mu = \Expect{X}
      &&
      \sigma = \sqrt{\Var X }
    \end{align*}
  \end{description}
  \bigskip

\item[Links :]  \rule{0pt}{1em}
  \href{OpenTURNS_ReferenceGuide.pdf}{see Reference Guide - Standard parametric models}
\end{description}


Each  $getMethod$  is associated to a $setMethod$.



% =================================================

\newpage \subsubsection{UserDefined}

This class inherits from the Distribution class.

\begin{description}

\item[Usage :] $UserDefined(Coll)$

\item[Arguments :]  $Coll$ : a UserDefinedPairCollection. The collection of UserDefinedPair of the UserDefinedPairCollection does not need such that $\sum_1^n  p_i = 1.0$. If not the case, the weights are normalized.

\item[Value :] a UserDefined

\item[Some methods :]  \rule{0pt}{1em}
  \begin{description}


  \item $getPairCollection$
    \begin{description}
    \item[Usage :] $getPairCollection()$
    \item[Arguments :] none
    \item[Value :]  a UserDefinedPairCollection, the  $Coll$ parameter of the considered distribution where the weights are normalized.
    \end{description}
    \bigskip

  \item $getSupport$
    \begin{description}
    \item[Usage :] $getSupport(interval)$
    \item[Arguments :] $interval$ : a $Interval$, an interval in $\mathbb{R}$
    \item[Value :]  a $NumericalSample$, all the points (here of dimension $n$) of the distribution range which are included in the interval $interval$.
    \end{description}
    \bigskip

  \item $isIntegral$
    \begin{description}
    \item[Usage :] $isIntegral()$
    \item[Arguments :] no argument
    \item[Value :] a boolean which indicates wether the considered distribution has integer values.
    \end{description}

  \end{description}

\item[Details :] \rule{0pt}{1em}
  \begin{description}
  \item probability function :
    $$
    \Prob{x_i} = p_i,\quad i = 1,\ldots,n
    $$
    where
    \begin{itemize}
    \item[] $(x_i,p_i)$ and $i=1,\ldots,n$ are respectively a NumericalPoint and its associated probability
    \item[] $n$ is the size of the UserDefinedPairCollection
    \end{itemize}

  \item One must have
    $$
    \sum_{i=1}^{n} p_i = 1
    $$

  \end{description}
\end{description}

Each  $getMethod$  is associated to a $setMethod$.

% =============================================================
\newpage \subsubsection{Weibull}

This class inherits from the Distribution class.

\begin{description}

\item[Usage :] \rule{0pt}{1em}
  \begin{description}
  \item Main parameters set : $Weibull(\alpha,\beta,\gamma)$
  \item Second parameter set : $Weibull(\mu,\sigma,\gamma,1)$
  \item Default construction : $Weibull( )$
  \end{description}
  \bigskip

\item[Arguments :]  \rule{0pt}{1em}
  \begin{description}
  \item $\alpha$        : a real value, the shape parameter, constraint : $\alpha > 0$
  \item $\beta$         : a real value, the scale parameter, constraint : $\beta > 0$
  \item $\gamma$        : a real value, the location parameter
  \item $\mu$   : a real value, the mean value,
  \item $\sigma$        : a real value, the standard deviation value, constraint : $\sigma > 0$
  \end{description}

\item[Value :] a Weibull. In the default construction, we use the $Weibull(\alpha, \beta, \gamma) = Weibull(1.0, 1.0, 0.0)$ definition.

\item[Some methods :]  \rule{0pt}{1em}
  \begin{description}

  \item $getAlpha$
    \begin{description}
    \item[Usage :] $getAlpha()$
    \item[Arguments :] none
    \item[Value :]  a real value, the  $\alpha$ of the considered distribution
    \end{description}
    \bigskip

  \item $getBeta$
    \begin{description}
    \item[Usage :] $getBeta()$
    \item[Arguments :] none
    \item[Value :]  a real value, the  $\beta$ of the considered distribution
    \end{description}
    \bigskip

  \item $getGamma$
    \begin{description}
    \item[Usage :] $getGamma()$
    \item[Arguments :] none
    \item[Value :]  a real value, the $\gamma$ parameter of the considered distribution
    \end{description}
    \bigskip

  \item $getMu$
    \begin{description}
    \item[Usage :] $getMu()$
    \item[Arguments :] none
    \item[Value :]  a real value,  the $\mu$ parameter of the considered distribution
    \end{description}
    \bigskip
  \item $getSigma$
    \begin{description}
    \item[Usage :] $getSigma()$
    \item[Arguments :] none
    \item[Value :]  a real value,  the $\sigma$ parameter of the considered distribution
    \end{description}
    \bigskip
  \end{description}

\item[Details :]  \rule{0pt}{1em}
  \begin{description}
  \item density function :
    $$
    f(x) =
    \frac{\beta}{\alpha}
    \left(
      \frac{x-\gamma}{\alpha}
    \right)^{\beta-1}
    e^{
      \left(
        - \left(
          \frac{x-\gamma}{\alpha}
        \right)^{\beta}
      \right)}
    \boldsymbol{1}_{[\gamma,+\infty[}(x)
    $$


  \item relation between parameters set :
    \begin{eqnarray*}
      \mu                       &       =       & \alpha \,\Gamma\left(1+\frac{1}{\beta}\right) + \gamma        \\
      \sigma            &  =    &        \alpha \sqrt{\Gamma\left(1+\frac{2}{\beta}\right) -  \Gamma^2 \left(1+\frac{2}{\beta}\right)}
    \end{eqnarray*}
    where $\Gamma$ is the $\Gamma$-function and
    \begin{align*}
      \mu = \Expect{X}
      &&
      \sigma = \sqrt{\Var X }
    \end{align*}
  \end{description}

\item[Links :]  \rule{0pt}{1em}
  \href{OpenTURNS_ReferenceGuide.pdf}{see Reference Guide - Standard parametric models}
\end{description}

Each  $getMethod$  is associated to a $setMethod$.



% =============================================================
\newpage \subsubsection{ZipfMandelbrot}

This class inherits from the Distribution class.

\begin{description}

\item[Usage :] \rule{0pt}{1em}
  \begin{description}
  \item Main parameters set : $ZipfMandelbrot(N,q,s)$
  \item Default construction : $ZipfMandelbrot( )$
  \end{description}

\item[Arguments :]  \rule{0pt}{1em}
  \begin{description}
  \item $N$ : a   integer, $N \geq 1$
  \item $q$ : a real value, $q \geq 0$
  \item $s$ : a real value, $s > 0$
  \end{description}

\item[Value :]  ZipfMandelbrot. In the default construction, we use the $ZipfMandelbrot(N,q,s) = ZipfMandelbrot(1,0,1)$ definition.

\item[Some methods :] \rule{0pt}{1em}
  \begin{description}

  \item $getN$
    \begin{description}
    \item[Usage :] $getN()$
    \item[Arguments :] none
    \item[Value :]  an integer,  the $N$ parameter of the ZipfMandelbrot distribution
    \end{description}
    \bigskip

  \item $getQ$
    \begin{description}
    \item[Usage :] $getQ()$
    \item[Arguments :] none
    \item[Value :]  a real value $\geq 0$,  the $q$ parameter of the ZipfMandelbrot distribution
    \end{description}
    \bigskip

  \item $getS$
    \begin{description}
    \item[Usage :] $getS()$
    \item[Arguments :] none
    \item[Value :]  a real value $>0$ ,  the $s$ parameter of the ZipfMandelbrot distribution
    \end{description}

  \end{description}

\item[Details :]  \rule{0pt}{1em}
  \begin{description}
  \item distribution :
    $$
    \forall k\in [1,N], P(X=k) = \frac{1}{(k+q)^s} \frac{1}{H(N,q,s)}
    $$
    where $H(N,q,s)$ is the Generalized Harmonic Number : $H(N,q,s) = \sum_{i=1}^{N} \displaystyle \frac{1}{(i+q)^s}$.

  \end{description}
  \bigskip

\item[Links :]  \rule{0pt}{1em}
  \href{OpenTURNS_ReferenceGuide.pdf}{see Reference Guide - Standard parametric models}
\end{description}


Each  $getMethod$  is associated to a $setMethod$.




% ===========================================================

\newpage \subsection{TruncatedDistribution}



This class enables to truncate any distribution within a specified range which one of the bounds may be infinite. It offers the methods of the Distribution class.

\begin{description}

\item[Usage :] \rule{0pt}{1em}
  \begin{description}
  \item $TruncatedDistribution(distribution, lowerBound, upperBound)$
  \item $TruncatedDistribution(distribution, bound, TruncatedDistribution.UPPER)$
  \item $TruncatedDistribution(distribution, bound, TruncatedDistribution.LOWER)$
  \end{description}
  \bigskip


\item[Arguments :]  \rule{0pt}{1em}
  \begin{description}
  \item $distribution$ : a Distribution
  \item $lowerBound$ : a real, the new lower bound of the distribution  : the distribution range is $[lowerBound, \infty[$ or $[lowerBound, max[$ if the distribution is already bounded by $max$
  \item $upperBound$ : a real, the new upper bound of the distribution : the distribution range is $[- \infty,upperBound [$ or $[min, upperBound[$ if the distribution is already bounded by $min$
  \end{description}

\item[Value :] a Distribution



\item[Some methods :]  \rule{0pt}{1em}
  \begin{description}

  \item $isContinuous$
    \begin{description}
    \item[Usage :] $isContinuous()$
    \item[Arguments :] no argument
    \item[Value :] a boolean which indicates wether the considered distribution is continuous.
    \end{description}
    \bigskip

  \item $isIntegral$
    \begin{description}
    \item[Usage :] $isIntegral()$
    \item[Arguments :] no argument
    \item[Value :] a boolean which indicates wether the considered distribution has integer values.
    \end{description}
    \bigskip


  \item $getSupport$
    \begin{description}
    \item[Usage :] $getSupport()$
    \item[Arguments :] none
    \item[Value :] a NumericalSample which gathers the different points of the discrete range. Care : this service is implemented only for discrete 1D distribution.
    \end{description}


  \end{description}





\item[Links :]  \rule{0pt}{1em}
  \href{./Version/docref_B122_Copules_en.pdf}{see docref\_B122\_Copules\_en}
\end{description}

Each  $getMethod$  is associated to a $setMethod$.

% ===========================================================

\newpage \subsection{Copulas}

\subsubsection{Copula}


This class inherits from the Distribution class.

\begin{description}

\item[Usage :] $Copula(copulaImplementation)$

\item[Arguments :]  $copulaImplementation$        : a CopulaImplementation, i.e a distribution which must verify the properties of a copula. This distribution can be any of the IndependentCopula, NormalCopula, ClaytonCopula, GumbelCopula, FrankCopula, MinCopula or SklarCopula built upon a multivariate distribution such as the Normal, Student, Dirichlet, Multinomial, KernelMixture, Mixture ones.

\item[Value :] a Copula

\item[Links :]  \rule{0pt}{1em}
  \href{./Version/docref_B122_Copules_en.pdf}{see docref\_B122\_Copules\_en}
\end{description}

Each  $getMethod$  is associated to a $setMethod$.

% =========================================================================
\newpage \subsubsection{ClaytonCopula}

This class inherits from the CopulaImplementation class.

\begin{description}

\item[Usage :] \rule{0pt}{1em}
  \begin{description}
  \item $ClaytonCopula()$
  \item $ClaytonCopula(theta)$
  \end{description}


\item[Arguments :]  $theta$     : a real, the only parameter of the Clayton copula, which PDF is : $\displaystyle \left(u_1^{-\theta}+u_2^{-\theta}-1\right)^{-1/\theta}$, for $u_i \in [0,1]$

\item[Value :]  \rule{0pt}{1em}
  \begin{description}
  \item In the first usage, a  ClaytonCopula of dimension 2 with $\theta=2.0$,
  \item In the second usage, a ClaytonCopula of dimension 2 with the $\theta$ specified.
  \end{description}

\item[Links :]
  \href{./Version/docref_B122_Copules_en.pdf}{see docref\_B122\_Copules\_en}
\end{description}

% =========================================================================
\newpage \subsubsection{FrankCopula}

This class inherits from the CopulaImplementation class.

\begin{description}

\item[Usage :] \rule{0pt}{1em}
  \begin{description}
  \item $FrankCopula()$
  \item $FrankCopula(theta)$
  \end{description}


\item[Arguments :]  $theta$     : a real, the only parameter of the Gumbel copula, which PDF is :
  $$
  \displaystyle -\frac{1}{\theta}\log\left(1+\frac{(e^{-\theta u_1}-1)(e^{-\theta u_2}-1}{e^{-\theta}-1}\right)
  $$, for $u_i \in [0,1]$

\item[Value :]  \rule{0pt}{1em}
  \begin{description}
  \item In the first usage, a  FrankCopula of dimension 2 with $\theta=2.0$,
  \item In the second usage, a FrankCopula of dimension 2 with the $\theta$ specified.
  \end{description}

\item[Links :]
  \href{./Version/docref_B122_Copules_en.pdf}{see docref\_B122\_Copules\_en}
\end{description}

% =========================================================================
\newpage \subsubsection{GumbelCopula}

This class inherits from the CopulaImplementation class.


\begin{description}

\item[Usage :] \rule{0pt}{1em}
  \begin{description}
  \item $GumbelCopula()$
  \item $GumbelCopula(theta)$
  \end{description}


\item[Arguments :]  $theta$     : a real, the only parameter of the Gumbel copula, which PDF is :
  $$
  \displaystyle e^{\left(-\left((-\log(u_1))^{\theta}+(-\log(u_2))^{\theta}\right)^{1/\theta}\right)}
  $$, for $u_i \in [0,1]$

\item[Value :]  \rule{0pt}{1em}
  \begin{description}
  \item In the first usage, a  GumbelCopula of dimension 2 with $\theta=2.0$,
  \item In the second usage, a  GumbelCopula of dimension 2 with the $\theta$ specified.
  \end{description}

\item[Links :]
  \href{./Version/docref_B122_Copules_en.pdf}{see docref\_B122\_Copules\_en}
\end{description}

% =============================================================
\newpage \subsubsection{IndependentCopula}

This class inherits from the CopulaImplementation class.

\begin{description}

\item[Usage :] \rule{0pt}{1em}
  \begin{description}
  \item $IndependentCopula()$
  \item $IndependentCopula(dim)$
  \end{description}

\item[Arguments :]  $dim$       : an integer, the dimension of the copula

\item[Value :] \rule{0pt}{1em}
  \begin{description}
  \item In the first usage, a  IndependentCopula of dimension 1,
  \item In the second usage, a IndependentCopula  of the dimension $dim$  specified.
  \end{description}

\item[Links :]
  \href{./Version/docref_B122_Copules_en.pdf}{see docref\_B122\_Copules\_en}
\end{description}


% =============================================================
\newpage \subsubsection{MinCopula}

This class inherits from the CopulaImplementation class.

\begin{description}

\item[Usage :] \rule{0pt}{1em}
  \begin{description}
  \item $IndependentCopula()$
  \item $IndependentCopula(dim)$
  \end{description}

\item[Arguments :]  $dim$       : an integer, the dimension of the copula

\item[Value :] \rule{0pt}{1em}
  \begin{description}
  \item In the first usage, a MinCopula of dimension 1,
  \item In the second usage, a MinCopula  of the dimension $dim$  specified.
  \end{description}

\item[Links :]
  \href{./Version/docref_B122_Copules_en.pdf}{see docref\_B122\_Copules\_en}
\end{description}


% =========================================================================
\newpage \subsubsection{NormalCopula}

This class inherits from the CopulaImplementation class.

\begin{description}

\item[Usage :] \rule{0pt}{1em}
  \begin{description}
  \item $NormalCopula()$
  \item $NormalCopula(R)$
  \end{description}

\item[Arguments :] $R$  : a CorrelationMatrix which is not the Kendall nor the Spearman rank correlation matrix of the distribution. The $R$ matrix  can be evaluated from the Spearman or Kendall correlation matrix.

\item[Value :] \rule{0pt}{1em}
  \begin{description}
  \item In the first usage, a NormalCopula  of dimension 1
  \item In the second usage, a NormalCopula  with the correlation matrix $R$ specified.
  \end{description}

\item[Some methods :]  \rule{0pt}{1em}
  \begin{description}
  \item $GetCorrelationFromKendallCorrelation$
    \begin{description}
    \item[Usage :] $NormaCopula.GetCorrelationFromKendallCorrelation(K)$
    \item[Arguments :] $K$ : a CorrelationMatrix, it must be the Kendall correlation matrix of the considered random vector
    \item[Value :] a CorrelationMatrix, the correlation matrix of the normal copula evaluated from the Kendall correlation matrix $K$
    \end{description}
    \bigskip

  \item $GetCorrelationFromSpearmanCorrelation$
    \begin{description}
    \item[Usage :] $NormalCopula.GetCorrelationFromSpearmanCorrelation(S)$
    \item[Arguments :]$S$ : a CorrelationMatrix, it must be the Spearman correlation matrix of the considered random vector
    \item[Value :] a CorrelationMatrix, the correlation matrix of the normal copula evaluated from the Spearman correlation matrix $S$
    \end{description}
  \end{description}
\item[Links :]
  \href{./Version/docref_B122_Copules_en.pdf}{see docref\_B122\_Copules\_en}
\end{description}


% =========================================================================
\newpage \subsubsection{SklarCopula}

This class inherits from the CopulaImplementation class.

\begin{description}

\item[Usage :] $SklarCopula(distribution)$

\item[Arguments :] $distribution$ : a Distribution, whatever its type (UsualDistribution, ComposedDistribution, KernelMixture, Mixture, RandomMixture, Copula, ...)

\item[Value :]  a  SklarCopula with the same dimension as the $distribution$

\item[Some methods :]  no specific method.
\item[Links :]
  \href{./Version/docref_B122_Copules_en.pdf}{see docref\_B122\_Copules\_en}
\end{description}


% =============================================================


\newpage        \subsubsection{ComposedCopula}


This class inherits from the CopulaImplementation class.


\begin{description}

\item[Usage :] $ComposedCopula(copulaCollection)$

\item[Arguments :] $copulaCollection$ : a CopulaCollection

\item[Value :] a ComposedCopula, defined as the product of the initial copulas. For example, if $C_1$ and $C_2$ are two copulas respectively of $\mathcal{R}^{n_1}$ and $\mathcal{R}^{n_2}$, we can create the copula of a random vector of $\mathcal{R}^{n_1+n_2}$, noted $C$ as follows :
  $$
  C(u_1, \cdots, u_n) = C_1(u_1, \cdots, u_{n_1}) C_2(u_{n_1+1}, \cdots, u_{n_1+n_2})
  $$
  It means that both subvectors $(u_1, \cdots, u_{n_1}$ and $(u_{n_1+1}, \cdots, u_{n_1+n_2})$ of $\mathcal{R}^{n_1}$ and $\mathcal{R}^{n_2}$ are independent.

\item[Some methods :]   \rule{0pt}{1em}
  \begin{description}
  \item $getCopulaCollection$
    \begin{description}
    \item[Usage :] $getCopulaCollection()$
    \item[Arguments :] none
    \item[Value :] a CopulaCollection, the collection of copulas from which the ComposedCopula is built
    \end{description}
    \bigskip
  \end{description}


\item[Links :]
  \href{./Version/docref_B_JoinedCDF_en.pdf}{see docref\_B\_JoinedCDF\_en}
\end{description}



% =============================================================


\newpage        \subsection{ComposedDistribution}


This class inherits from the Distribution class.


\begin{description}

\item[Usage :]  \rule{0pt}{1em}
  \begin{description}
  \item $ComposedDistribution(distributionCollection, copula)$
  \item $ComposedDistribution(distributionCollection)$
  \end{description}

\item[Arguments :]  \rule{0pt}{1em}
  \begin{description}
  \item $distributionCollection$ : a DistributionCollection, the collection of the marginals of the distribution
  \item $copula$ : a Copula, the copula of the distribution.
  \end{description}

\item[Value :] a ComposedDistribution,
  \begin{description}
  \item in the first usage : which marginals and copula are specified,
  \item in the second usage : which marginals are specified, and which copula is the independent one.
  \end{description}

\item[Some methods :]   \rule{0pt}{1em}
  \begin{description}
  \item $getDistributionCollection$
    \begin{description}
    \item[Usage :] $getDistributionCollection()$
    \item[Arguments :] none
    \item[Value :] a DistributionCollection, the collection of distributions from which the ComposedDistribution is built
    \end{description}
    \bigskip
  \end{description}


\item[Links :]
  \href{./Version/docref_B_JoinedCDF_en.pdf}{see docref\_B\_JoinedCDF\_en}
\end{description}

Each  $getMethod$  is associated to a $setMethod$.



% =============================================================

\newpage \subsection{Linear combination of probability density functions}

\subsubsection{Mixture}

A Mixture is a distribution such that its probability density function is a linear combination of probability density functions, with the linear combination coefficients greater or equal to zero such that their sum is equal to 1. \\
It is important to note that the linear combination coefficients are given through the {\itshape weight} attribute of each component of the DistributionCollection, thanks to the command {\itshape DistributionCollection[i].setWeight(coefficient)}.

\begin{description}

\item[Usage :] \rule{0pt}{1em}
  \begin{description}
  \item $Mixture(distributionCollection)$
  \item $Mixture(distributionCollection, weights)$
  \end{description}

\item[Arguments :] \rule{0pt}{1em}
  \begin{description}
  \item $distributionCollection$     : a DistributionCollection, the collection of the  distributions which compose the linear combination
  \item $weights$ : a NumericalPoint, which contains the weights of the distributions in the mixture. These weights will be used instead of the individual weights of each distribution.
  \end{description}
\item[Value :] a Mixture

\item[Some methods :]  \rule{0pt}{1em}

  \begin{description}
  \item $getDistributionCollection$
    \begin{description}
    \item[Usage :] $getDistributionCollection()$
    \item[Arguments :] none
    \item[Value :] a DistributionCollection the collection of distribution from which the Mixture is built
    \end{description}
  \end{description}


\item[Details :]  \rule{0pt}{1em}
  \begin{description}
  \item probability density function :
    $$
    f(x) =  \sum \alpha_i p_i(x)
    $$
    with $\alpha_i\geq 0$. The null weights are automatically removed from the sum, and the coefficients are automatically normalized such that $\sum \alpha_i=1$.
  \end{description}

\item[Links :]  \rule{0pt}{1em}
  \href{OpenTURNS_ReferenceGuide.pdf}{see Reference Guide - Standard parametric models}
\end{description}

Each  $getMethod$  is associated to a $setMethod$.


% =============================================================


\newpage \subsubsection{KernelMixture}


A KernelMixture is a distribution built from a NumericalSample, such that its probability density function is a linear combination of the kernel specified by the User, centered on each point of the NumericalSample, which standard deviation is the bandwidth specified by the User.
It is important to note that the linear combination coefficients are all equal.



\begin{description}

\item[Usage :] $KernelMixture(kernel, bandwidth, sample)$

\item[Arguments :] $distributionCollection$     : a DistributionCollection, the collection of the  distributions which compose the linear combination


\item[Value :] a  KernelMixture

\item[Some methods :]  \rule{0pt}{1em}
  \begin{description}
  \item $getBandwidth$
    \begin{description}
    \item[Usage :] $getBandwidth()$
    \item[Arguments :] none
    \item[Value :] a NumericalPoint, the bandwidth of the kernel mixture, (see equation below for dimension 1). The bandwidth is the same at each point of the NumericalSample
    \end{description}
    \bigskip

  \item $getKernel$
    \begin{description}
    \item[Usage :] $getKernel()$
    \item[Arguments :] none
    \item[Value :] a Distribution, the kernel $K$ of the mixture, (see equation below for dimension 1)
    \end{description}
    \bigskip

  \item $getSample$
    \begin{description}
    \item[Usage :] $getSample()$
    \item[Arguments :] none
    \item[Value :] a NumericalSample, the NumericalSample of the mixture, (see equation below for dimension 1)
    \end{description}


  \end{description}

\item[Details :]  \rule{0pt}{1em}
  \begin{description}
  \item Probability density function in dimension 1 :
    $$
    f(x) =  \sum \frac{1}{nh}K(\frac{X_i-x}{h}), x \in \mathbb{R}
    $$
    where $(X_1, \dots, X_n)$ is  a NumericalSample
  \end{description}

\item[Links :]  \rule{0pt}{1em}
  \href{OpenTURNS_ReferenceGuide.pdf}{see Reference Guide - Standard parametric models}
\end{description}

Each  $getMethod$  is associated to a $setMethod$.


% =============================================================


\newpage   \subsubsection{KernelSmoothing}


The class KernelSmoothing enables to build some kernels used to fit a distribution to a numerical sample.


\begin{description}

\item[Usage :] \rule{0pt}{1em}
  \begin{description}
  \item $KernelSmoothing()$
  \item $KernelSmoothing(Distribution(myDistribution))$
  \item $KernelSmoothing(DistributionImplentation())$
  \end{description}


\item[Arguments :] \rule{0pt}{1em}
  \begin{description}
  \item $myDistribution$ : a 1D Distribution of any kind
  \item $DistributionImplentation()$ : default constructor of the 1D UsualDistribution. For example, $Uniform()$, $Triangular()$, ...
  \end{description}



\item[Value :]  a Distribution
  \begin{description}
  \item In the first usage, the kernel is the  kernel product of 1D Normal(1.0, 0.0). The dimension of the product is detected from the numerical sample.
  \item In the second usage, the kernel is the  kernel product of 1D distributions specified by $myDistribution$. Care : the kernel smoothing method is all the more efficient than the kernel is symmetric with respect to 0.0. The dimension of the product is detected from the numerical sample.
  \item In the third usage, the kernel is the  kernel product of the default constructions of the 1D UsualDistributions. Note that the default constructor of a UsualDistribution builds a distribution which is symmetric with respect to 0.0 when it is possible. The dimension of the product is detected from the numerical sample.
  \end{description}

\item[Some methods :]  \rule{0pt}{1em}
  \begin{description}
  \item $build$
    \begin{description}
    \item[Usage :] \rule{0pt}{1em}
      \begin{description}
      \item  $build(sample)$
      \item  $build(sample, boundaryCorrection)$
      \item  $build(sample, bandwidth)$
      \item  $build(sample, bandwidth, boundaryCorrection)$
      \end{description}
    \item[Arguments :] \rule{0pt}{1em}
      \begin{description}
      \item  $sample$ : a NumericalSample, the numerical sample from which the kernel mixture is built
      \item  $boundaryCorrection$ : a Bool which indicates if it is necessary to make a boundary treatment (according to the mirroring technique)
      \item  $bandwidth$ : a NumericalPoint, the bandwidth of the kernel product. The dimension is detected from the numerical sample $sample$ and evaluated according to the Scott rule.
      \end{description}
    \item[Value :]  a Distribution. When the bandwidth is not specified, Open TURNS proceeds as follows : the plug-in method on the entire numerical sample if its size is inferior to 250; the mixted method in the other case. Refer to the Reference Guide in order to have details on these methods.
    \end{description}
    \bigskip

  \item $computeSilvermanBandwidth$
    \begin{description}
    \item[Usage :] $computeSilvermanBandwidth(sample)$
    \item[Arguments :] $sample$ : a NumericalSample, the numerical sample from which the kernel mixture is built
    \item[Value :] a NumericalPoint, the bandwidth automatically evaluated by Open TURNS from the numerical sample according to the Silverman rule (see Reference Guide)
    \end{description}
    \bigskip

  \item $computePluginBandwidth$
    \begin{description}
    \item[Usage :] $computePluginBandwidth(sample)$
    \item[Arguments :] $sample$ : a NumericalSample, the numerical sample from which the kernel mixture is built
    \item[Value :] a NumericalPoint, the bandwidth automatically evaluated by Open TURNS from the numerical sample according to the plug-in method (see Reference Guide)
    \end{description}
    \bigskip

  \item $computeMixedBandwidth$
    \begin{description}
    \item[Usage :] $computeMixedBandwidth(sample)$
    \item[Arguments :] $sample$ : a NumericalSample, the numerical sample from which the kernel mixture is built
    \item[Value :] a NumericalPoint, the bandwidth automatically evaluated by Open TURNS from the numerical sample according to the mixted method (see Reference Guide)
    \end{description}
    \bigskip

  \item $getBandwidth$
    \begin{description}
    \item[Usage :] $getBandwidth()$
    \item[Arguments :] none
    \item[Value :] a NumericalPoint, the bandwidth of the kernel mixture, (see equation below for dimension 1). The bandwidth is the same at each point of the NumericalSample
    \end{description}
    \bigskip

  \item $getKernel$
    \begin{description}
    \item[Usage :] $getKernel()$
    \item[Arguments :] none
    \item[Value :] a Distribution, the kernel adopted for the kernel smoothing
    \end{description}


  \end{description}

\item[Details :]  \rule{0pt}{1em}
  \begin{description}
  \item Probability density function in dimension 1 :
    $$
    p_n(x) =  \sum \frac{1}{nh}K(\frac{x-X^i}{h}), x \in \mathbb{R}
    $$
    where $(X^1, \dots, X^n)$ is  a NumericalSample and $K$ the kernel PDF,
  \item Probability density function in dimension $N$ :
    $$
    p_n(\vec{x}) = \displaystyle \frac{1}{n}\sum_{i=1}^{i=N} \prod_{j=1}^{j=N} \frac{1}{h_j} K(\frac{x^j-X_i^j}{h^j})
    $$
    where $\prod_{j=1}^{j=N} K(x^j)$ is the kernel product and $\vect{h} = (h^1, \cdots, h^N)$ the vector of bandwidth.
  \end{description}

\item[Links :]  \rule{0pt}{1em}
  \href{OpenTURNS_ReferenceGuide.pdf}{see Reference Guide - Standard parametric models}
\end{description}


% =============================================================

\newpage \subsection{Affine combinations of independent univariate random variables}

\subsubsection{RandomMixture}

A RandomMixture $Y$ is a univariate random variable defined as an affine combination of independent univariate random variable, as follows :
$$
\displaystyle Y = a_0 + \sum_{k=1}^{k=n} a_k X_k
$$
where $(a_i)_{ 0 \leq k \leq n} \in \mathbb{R}$ and $(X_k)_{ 1 \leq k \leq n}$ are some independent univariate distributions.\\


\begin{description}

\item[Usage :] \rule{0pt}{1em}
  \begin{description}
  \item $RandomMixture(collDist)$
  \item $RandomMixture(collDist, constant)$
  \item $RandomMixture(collDist, weights)$
  \item $RandomMixture(collDist, weights, constant)$
  \end{description}


\item[Arguments :] \rule{0pt}{1em}
  \begin{description}
  \item $collDist$ : a DistributionCollection, the collection of the  univariate independent distributions distributions which compose the affine combination,
  \item $constant$ : a scalar, the constant term $a_0$ of the affine combination,
  \item $weights$ : a NumericalPoint, which contains the weights of the affine combination : $(a_1, \dots, a_n)$.
  \end{description}




\item[Value :] a  RandomMixture such that :
  \begin{description}
  \item in the first usage : the weights $a_i$ have been affected in the  distributions of the univariate random variables $X_i$, thanks to the method $setWeight(a_i)$, before the creation of $collDist$. If not specified, the weight by default is $1.0$. The constant term $a_0 = 0$.
  \item in the second usage : the weights are defined as in the first usage and the constant term $a_0$ is equal to $constant$.
  \item in the third usage : the weights are directly specified at the creation of the RandomMixture. The distribution weights are modified to be equal to those specified values. The constant term $a_0 = 0$.
  \item in the fourth usage : the weights are specified as in the third usage and the constant term $a_0$ is equal to $constant$.
  \end{description}

\item[Some methods :]  As a {\itshape RandomMixture} is a {\itshape Distribution}, it is possible to ask it any request compatible with a {\itshape Distribution} object : moments, quantiles, PDF and CDF evaluations, ...

\item $project$
  \begin{description}
  \item[Usage :] \rule{0pt}{1em}
    \begin{description}
    \item $project(factoryCollection, testResult)$
    \item $project(factoryCollection, testResult, size)$
    \end{description}
  \item[Arguments :] \rule{0pt}{1em}
    \begin{description}
    \item $factoryCollection$ : a FactoryCollection, the collection of models from which one want to find the best approximation of the RandomMixture.
    \item $testResult$ : a TestResult, the test result associated with the best model found within the collection of DistributionFactory upon which one projects the RandomMixture.
    \item $size$ : an integer. The RandomMixture is regularly sampled over a grid of $2\,size+1$ points that correspond to quantiles of regularly spaced levels, then one uses the Kolmogorov test to identify the best candidate amongst the several factories. If not given, size defaults to 50, which means a regular sample of size 101.
    \end{description}
  \item[Value :] a Distribution, the best candidates found according to the Kolmogorov test within the given DistributionFactoryCollection using the regular sample of size $2\,size+1$.
  \end{description}



\item[Links :]  \rule{0pt}{1em}
  \href{OpenTURNS_ReferenceGuide.pdf}{see Reference Guide - Standard parametric models}
\end{description}






% =============================================================

\newpage \subsection{Random vector}

\subsubsection{RandomVector}

\begin{description}

\item[Usage :] \rule{0pt}{1em}
  \begin{description}
  \item $RandomVector(distribution)$
  \item $RandomVector(function, antecedent)$
  \item $RandomVector(functionalChaosResult)$
  \item $RandomVector(constant)$
  \end{description}

\item[Arguments :]  \rule{0pt}{1em}
  \begin{description}
  \item $distribution$ : a Distribution
  \item $function$ : a NumericalMathFunction
  \item $antecedent$ : a RandomVector of type Usual (see farther)
  \item $functionalChaosResult$ : a FunctionalChaosResult which contains the results of a Chaos Polynomial approximation
  \item $constant$ : a NumericalPoint
  \end{description}

\item[Value :] a RandomVector, which is of type : \rule{0pt}{1em}
  \begin{description}
  \item $Usual$ : if created thanks to the first usage. In that case, the RandomVector has for distribution the one specified through $distribution$.
  \item $Composite$ : if created thanks to the second usage. In that case, the RandomVector is defined as the image through the function $function$ of the Usual RandomVector $antecedent$ :  $Y=fuction(antecedent)$.
  \item $FunctionalChaosRandomVector$ : if created thanks to the third usage. In that case, the RandomVector is the image through a functionial chaos approximation model of the assoiated Usual RandomVector.
  \item $Constant$ : if created thanks to the fourth usage. In that case, the RandomVector is a constant vector equal to the NumericalPoint specified  $constant$.
  \end{description}

\item[Some methods :]  \rule{0pt}{1em}
  \begin{description}
  \item $getAntecedent$
    \begin{description}
    \item[Usage :] $getAntecedent()$
    \item[Arguments :] no argument
    \item[Value :] a RandomVector, only in the case of Composite RandomVector : the  RandomVector $X$ such that $Y=function(X)$.
    \end{description}
    \bigskip
  \item $getCovariance$
    \begin{description}
    \item[Usage :] $getCovariance()$
    \item[Arguments :] no argument
    \item[Value :] a CovarianceMatrix, only in the case of Usual or FunctionalChaos RandomVector : the covariance of the  considered RandomVector
    \end{description}
    \bigskip

  \item $getDistribution$
    \begin{description}
    \item[Usage :] $getDistribution()$
    \item[Arguments :] no argument
    \item[Value :] a Distribution, only in the case of Usual RandomVector : the distribution of the RandomVector
    \end{description}
    \bigskip

  \item $getDescription$
    \begin{description}
    \item[Usage :] $getDescription()$
    \item[Arguments :] no argument
    \item[Value :] a Description, the description of the Randomvector
    \end{description}
    \bigskip

  \item $getDimension$
    \begin{description}
    \item[Usage :] $getDimension()$
    \item[Arguments :] no argument
    \item[Value :] an integer, the dimension of the RandomeVector
    \end{description}
    \bigskip

  \item $getFunctionalChaosResult$
    \begin{description}
    \item[Usage :] $getFunctionalChaosResult()$
    \item[Arguments :] no argument
    \item[Value :] a FunctionalChaosResult, only in the case of FunctionalChaos RandomVector : the  result structure of the Chaos Polynomial study.
    \end{description}
    \bigskip


  \item $getMarginal$
    \begin{description}
    \item[Usage :] \rule{0pt}{1em}
      \begin{description}
      \item $getMarginal(i)$
      \item $getMarginal(indices)$
      \end{description}
    \item[Arguments :] \rule{0pt}{1em}
      \begin{description}
      \item $i$ : an integer which indicates the component concerned
      \item $indices$ : a Indices which regroups all the components concerned
      \end{description}
    \item[Value :] a RandomVector restricted to the concerned components.
    \end{description}
  \item[Details :] Let's note $\vect{Y} = \Tr{(Y_1, \cdots, Y_n)}$ a random vector and $I \in [1, n]$ a set of indices. If $\vect{Y}$ is a UsualRandomvector, the subvector is defined by $\vect{\tilde{Y}} = \Tr{(Y_i)_{i \in I}}$. If $\vect{Y}$ is a CompositeRandomVector, defined by $\vect{Y} = f(\vect{X})$ with $f = (f_1, \cdots, f_n)$, $f_i$ some scalar functions, the sub vector is $\vect{\tilde{Y}} = (f_i(\vect{X}))_{i \in I}$.
    \bigskip
  \item $getMean$
    \begin{description}
    \item[Usage :] $()$
    \item[Arguments :] no argument
    \item[Value :] a NumericalPoint, only in the case of Usual or FunctionalChaos RandomVector : the mean vector of the associated distribution
    \end{description}
    \bigskip

  \item $getName$
    \begin{description}
    \item[Usage :] $getName()$
    \item[Arguments :] no argument
    \item[Value :] a string, the name of the RandomVector
    \end{description}
    \bigskip

  \item $getNumericalSample$
    \begin{description}
    \item[Usage :] $getNumericalSample()$
    \item[Arguments :] no argument
    \item[Value :] a NumericalSample
      a sample of the random vector
    \end{description}
    \bigskip

  \item $isComposite$
    \begin{description}
    \item[Usage :] $isComposite()$
    \item[Arguments :] no argument
    \item[Value :] a boolean which indicates if the RandomVector is of type Composite or Usual.
    \end{description}
  \end{description}

\end{description}

Each  $getMethod$  is associated to a $setMethod$.



% =============================================================

\newpage \subsubsection{ConditionalRandomVector}

\begin{description}

\item[Usage :] $ConditionalRandomVector(conditionalDist, randomParameters)$

\item[Arguments :]  \rule{0pt}{1em}
  \begin{description}
  \item $conditionalDist$ : a Distribution
  \item $randomParameters$ : a RandomVector
  \end{description}

\item[Value :] a ConditionalRandomVector,  $\vect{Y}$ which distribution $\mathcal{L}(\vect{\Theta})$ has random parameters $\vect{\Theta}$ distributed according to the distribution $\mathcal{D}_{\vect{\Theta}}$. The random vector  $\vect{\Theta}$ can be :  
\begin{itemize}
  \item a {\itshape UsualRandomVector} which means described by a given distribution $\mathcal{D}_{\vect{\Theta}}$,  
  \item or a {\itshape CompositeRandomVector} which means the output vector of a fonction $f$ evaluated on the random vector $\vect{X}$  : $\vect{\Theta} = f(\vect{X})$. In that case, the distribution $\mathcal{D}_{\vect{\Theta}}$ is not explicitely known.
\end{itemize}

\item[Some methods :]  \rule{0pt}{1em}
  \begin{description}

  \item $getAntecedent$
    \begin{description}
    \item[Usage :] $getRealization()$
    \item[Arguments :] no argument
    \item[Value :] a NumericalPoint, a realization of the final distribution of $\vect{Y}$. Open TURNS proceeds as follows : it first generates a realization $\vect{\theta}$  of the random vector  $\vect{\Theta}$ according to  $\mathcal{D}_{\vect{\Theta}}$ then a realization of the distribution $\mathcal{L}(\vect{\theta})$ where the random vector $\vect{\Theta}$  is fixed to the previous realization $\vect{\theta}$ .\\
    \end{description}
    \bigskip

  \item $getRandomParameters$
    \begin{description}
    \item[Usage :] $getRandomParameters()$
    \item[Arguments :] no argument
    \item[Value :] a RandomVector, the random vector $\vect{\Theta}$.
    \end{description}
    \bigskip

  \item $getDistribution$
    \begin{description}
    \item[Usage :] $getDistribution()$
    \item[Arguments :] no argument
    \item[Value :] a Distribution, the conditional distribution $\mathcal{L}(\vect{\Theta})$ .
    \end{description}
    \bigskip

  \item $getDimension$
    \begin{description}
    \item[Usage :] $getDimension()$
    \item[Arguments :] no argument
    \item[Value :] an integrer, the dimension of the distribution $\mathcal{L}(\vect{\Theta})$.
    \end{description}


    \end{description}
 
\end{description}
