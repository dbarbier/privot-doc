% Copyright (c)  2005-2010 EDF-EADS-PHIMECA.
% Permission is granted to copy, distribute and/or modify this document
% under the terms of the GNU Free Documentation License, Version 1.2
% or any later version published by the Free Software Foundation;
% with no Invariant Sections, no Front-Cover Texts, and no Back-Cover
% Texts.  A copy of the license is included in the section entitled "GNU
% Free Documentation License".



\newpage \section{Polynomials}



% ===================================================================

\subsection{UniVariatePolynomial}

\begin{description}

\item[Usage :] $UniVariatePolynomial(coefficients)$

\item[Arguments :]  $coefficients$ : a NumericalPoint, the list of the coefficients of each term $x^k$ (no sparse repesentation)

\item[Value :] a UniVariatePolynomial

\item[Some methods :]  \rule{0pt}{1em}

  \begin{description}

  \item $*$
    \begin{description}
    \item[Usage :]  \rule{0pt}{1em}
      \begin{description}
      \item $Pol1*Pol2$
      \item $lambda*Pol1$
      \end{description}
    \item[Arguments :] \rule{0pt}{1em}
      \begin{description}
      \item $(Pol1, Pol2)$ : two UniVariatePolynomial,
      \item $lambda$ : a NumericalScalar
      \end{description}
    \item[Value :] \rule{0pt}{1em}
      \begin{description}
      \item usage 1 : a UniVariatePolynomial, the result of the multiplication  $Pol1*Pol2$
      \item usage 2 : a UniVariatePolynomial, the result of the multiplication  $lambda*Pol1$
      \end{description}
    \end{description}
    \bigskip

  \item $+$
    \begin{description}
    \item[Usage :]  $Pol1+Pol2$
    \item[Arguments :] $(Pol1, Pol2)$ : two UniVariatePolynomial
    \item[Value :] a UniVariatePolynomial, the result of the addition  $Pol1+Pol2$
    \end{description}
    \bigskip

  \item $-$
    \begin{description}
    \item[Usage :]  $Pol1-Pol2$
    \item[Arguments :] $(Pol1, Pol2)$ : two UniVariatePolynomial
    \item[Value :] a UniVariatePolynomial, the result of the substraction  $Pol1-Pol2$
    \end{description}
    \bigskip

  \item $derivate$
    \begin{description}
    \item[Usage :] $derivate()$
    \item[Arguments :] none
    \item[Value :] a UniVariatePolynomial, the derivated univariate polynomials
    \end{description}
    \bigskip

  \item $derivative$
    \begin{description}
    \item[Usage :] $derivative(point)$
    \item[Arguments :] $point$ : a NumericalScalar
    \item[Value :] a NumericalScalar, the value of the derivated polynomials at point $point$
    \end{description}
    \bigskip

  \item $draw$
    \begin{description}
    \item[Usage :] $draw(min, max, pointNumber)$
    \item[Arguments :] \rule{0pt}{1em}
      \begin{description}
      \item $min, max$ : a NumericalScalar
      \item $pointNumber$ : an integer, the number of points used for the grah
      \end{description}
    \item[Value :] a Graph, the polynomials curve on the range $[min, max]$.
    \end{description}

  \item $getCoefficients$
    \begin{description}
    \item[Usage :] $getCoefficients()$
    \item[Arguments :] none
    \item[Value :] a Coefficients, the coefficients of each $x^k$ for $k \leq$ to the degree of the univariate polynomials (no sparse repesentation)
    \end{description}
    \bigskip

  \item $getDegree$
    \begin{description}
    \item[Usage :] $getDegree()$
    \item[Arguments :] none
    \item[Value :] an integer, the degree of the univariate polynomials
    \end{description}
    \bigskip

  \item $getRoots$
    \begin{description}
    \item[Usage :] $getRoots()$
    \item[Arguments :] none
    \item[Value :] a NumericalComplexCollection, the collection of complex roots of the univariate polynomials
    \end{description}
    \bigskip

  \item $incrementDegree$
    \begin{description}
    \item[Usage :] $incrementDegree(deg)$
    \item[Arguments :] $deg$ : an integer
    \item[Value :] a UniVariatePolynomial obtained by multiplying the polynomial by $x^{deg}$
    \end{description}

  \end{description}

\end{description}




% ===================================================================

\newpage \subsection{PolynomialCollection}

\begin{description}

\item[Usage :] $PolynomialCollection(size, univariatePol)$

\item[Arguments :]  \rule{0pt}{1em}
  \begin{description}
  \item $size$ : an integer
  \item $univariatePol$ : a UniVariatePolynomial
  \end{description}

\item[Value :] a PolynomialCollection, which contains $size$ polynomials each equal to  $univariatePol$

\item[Some methods :]  \rule{0pt}{1em}

\item $add$
  \begin{description}
  \item[Usage :]  $add(univariatePol)$
  \item[Arguments :] $univariatePol$ : a UniVariatePolynomial,
  \item[Value :] a PolynomialCollection which size has been increased of 1 and to which the polynomials $univariatePol$ has been added
  \end{description}
  \bigskip

\item $at$
  \begin{description}
  \item[Usage :]  $at(i)$
  \item[Arguments :] $i$ : an integer
  \item[Value :] a UniVariatePolynomial, the polynomials at position $i$ in the collection
  \end{description}
  \bigskip

\item $resize$
  \begin{description}
  \item[Usage :]  $resize(newSize)$
  \item[Arguments :] $i$ : an integer
  \item[Value :] a PolynomialCollection which size has been modified into $newSize$ as follows : if $inewSize \leq getSize()$ then, the collection is truncated to the first $newSize$ polynomials. Otherwise, the collection is increased until the size $newSize$ : the added polynomials are the nul ones.
  \end{description}


\end{description}



% ===================================================================

\newpage \subsection{ProductPolynomialEvaluationImplementation}

\begin{description}

\item[Usage :] $ProductPolynomialEvaluationImplementation(polCollection)$

\item[Arguments :]  $polCollection$ : a PolynomialCollection, a collection of UniVariatePolynomial

\item[Value :] a ProductPolynomialEvaluationImplementation, the product of the polynomials of $polCollection$. The result polynomials is of input dimension $n$ where $n$ is the number of polynomials in  $polCollection$.

\item[Some methods :]  \rule{0pt}{1em}

\item $Operator()$
  \begin{description}
  \item[Usage :]  $Operator(point)$
  \item[Arguments :] $point$ : a NumericalPoint, which dimension is $n$ where $n$ is the number of polynomials in  $polCollection$
  \item[Value :] a NumericalPoint of dimension 1,
  \end{description}
  \bigskip

\item $+$
  \begin{description}
  \item[Usage :]  $Pol1+Pol2$
  \item[Arguments :] $(Pol1, Pol2)$ : two UniVariatePolynomial
  \item[Value :] a UniVariatePolynomial, the result of the addition  $Pol1+Pol2$
  \end{description}


\item[Details :]  The exact gradient and hessian evaluations have been implemented for the products of polynomials.

\end{description}

