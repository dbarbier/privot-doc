\newpage

\section{Taylor decomposition of the limit state function}

\subsection{QuadraticCumul}
\begin{description}

\item[Usage :]  $QuadraticCumul(randVect)$

\item[Arguments :]  $randVect$ : a RandomVector, constraint : this RandomVector must be of type Composite, which means it must have been defined with the second usage of declaration of a RandomVector (from a NumericalMathFunction and an antecedent Distribution)

\item[Value :] a QuadraticCumul

\item[Some methods :]  \rule{0pt}{1em}

  \begin{description}

  \item $drawImportanceFactors$
    \begin{description}
    \item[Usage :] $drawImportanceFactors()$
    \item[Arguments :] none
    \item[Value :] a Graph, the structure containing the pie corresponding to the importance factors of the probabilistic variables
    \end{description}
    \bigskip

  \item $getCovariance$
    \begin{description}
    \item[Usage :] $getCovariance()$
    \item[Arguments :] none
    \item[Value :] a CovarianceMatrix, approximation of first order of the covariance matrix of the random vector
    \end{description}
    \bigskip

  \item $getImportanceFactors$
    \begin{description}
    \item[Usage :] $getImportanceFactors()$
    \item[Arguments :] none
    \item[Value :] a NumericalPoint, the importance factors of the inputs : only when $randVect$ is of dimension 1
    \end{description}

    \bigskip
  \item $getMeanFirstOrder$
    \begin{description}
    \item[Usage :] $getMeanFirstOrder()$
    \item[Arguments :] none
    \item[Value :] a NumericalPoint, approximation at the first order of the mean of the random vector
    \end{description}
    \bigskip


  \item $getMeanSecondOrder$
    \begin{description}
    \item[Usage :] $getMeanSecondOrder()$
    \item[Arguments :] none
    \item[Value :] a NumericalPoint, approximation at the second order of the mean of the random vector (it requires that the hessian of the NumericalMathFunction has been defined)
    \end{description}
    \bigskip

  \item $getValueAtMean$
    \begin{description}
    \item[Usage :] $getValueAtMean()$
    \item[Arguments :] none
    \item[Value :] a NumericalPoint, the value of the NumericalMathFunction which defines the random vector at the mean point of the input random vector
    \end{description}
    \bigskip

  \item $getGradientAtMean$
    \begin{description}
    \item[Usage :] $getGradientAtMean()$
    \item[Arguments :] none
    \item[Value :] a Matrix, the gradient of the NumericalMathFunction which defines the random vector at the mean point of the input random vector
    \end{description}
    \bigskip

  \item $getHessianAtMean$
    \begin{description}
    \item[Usage :] $getHessianAtMean()$
    \item[Arguments :] none
    \item[Value :] a SymmetricTensor, the hessian of the NumericalMathFunction which defines the random vector at the mean point of the input random vector
    \end{description}

  \end{description}

\end{description}


