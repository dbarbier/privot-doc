% Copyright (c)  2005-2010 EDF-EADS-PHIMECA.
% Permission is granted to copy, distribute and/or modify this document
% under the terms of the GNU Free Documentation License, Version 1.2
% or any later version published by the Free Software Foundation;
% with no Invariant Sections, no Front-Cover Texts, and no Back-Cover
% Texts.  A copy of the license is included in the section entitled "GNU
% Free Documentation License".


\newpage\section{Response Surface : Parametric Approximation}




\subsection{Taylor approximation}

\subsubsection{LinearTaylor}

\begin{description}

\item[Usage :] $LinearTaylor(center, function)$
  \bigskip

\item[Arguments :]  \rule{0pt}{1em}
  \begin{description}
  \item $center$ : a NumericalPoint, the point where the Taylor expansion of the function $function$ is performed
  \item $function$ : a NumericalMathFunction, the function to be approximated.
  \end{description}

\item[Value :] a LinearTaylor

\item[Some methods :]  \rule{0pt}{1em}
  \begin{description}

  \item $getInputFunction$
    \begin{description}
    \item[Usage :] $getInputFunction$
    \item[Arguments :] none
    \item[Value :] a NumericalMathFunction, the function $function$
    \end{description}
    \bigskip

  \item $getName$
    \begin{description}
    \item[Usage :] $getName()$
    \item[Arguments :] none
    \item[Value :] a string, the name of the LinearTaylor
    \end{description}
    \bigskip

  \item $getCenter$
    \begin{description}
    \item[Usage :] $getCenter()$
    \item[Arguments :] none
    \item[Value :] a NumericalPoint, around which the approximation has been made : $center$
    \end{description}
    \bigskip

  \item $run$
    \begin{description}
    \item[Usage :] $run()$
    \item[Arguments :] none
    \item[Value :] it performs the linear Taylor expansion around $center$
      (while this method has not been executed, only
      $getInputFunction$, $getName$ and $setName$ methods can be used)
    \end{description}
    \bigskip

  \item $getConstant$
    \begin{description}
    \item[Usage :] $getConstant()$
    \item[Arguments :] none
    \item[Value :] a NumericalPoint, the constant vector of the approximation, equal to $function(center)$
    \end{description}
    \bigskip

  \item $getLinear$
    \begin{description}
    \item[Usage :] $getLinear()$
    \item[Arguments :] none
    \item[Value :] a Matrix, the gradient of the function $function$ at the point $center$ (the transposition of the jacobian matrix)
    \end{description}
    \bigskip


  \item $getResponseSurface$
    \begin{description}
    \item[Usage :] $getResponseSurface()$ %
    \item[Arguments :] none
    \item[Value :] a NumericalMathFunction, an approximation of the function $function$ by a linear Taylor expansion at the $center$
    \end{description}
    \bigskip

  \end{description}

\item[Links :] \rule{0pt}{1em}
  \href{./Version/docref_SurfRep_Taylor_en.pdf}{see docref\_SurfRep\_Taylor}
\end{description}

The methods $getInputFunction$, $getName$, $getCenter$ have their associated $setMethod$.

% -=============================================================


\newpage \subsubsection{QuadraticTaylor}

\begin{description}

\item[Usage :] $QuadraticTaylor(center, function)$

\item[Arguments :]  \rule{0pt}{1em}
  \begin{description}
  \item $center$ : a NumericalPoint, the point where the quadratic Taylor expansion of
    the function $function$ is performed
  \item $function$ : a NumericalMathFunction, the function to be approximated : the gradient and  hessian of the NumericalMathFunction must be defined.
  \end{description}

\item[Value :] a QuadraticTaylor
  \bigskip

\item[Some methods :]  \rule{0pt}{1em}
  \begin{description}

  \item $getInputFunction$
    \begin{description}
    \item[Usage :] $getInputFunction$
    \item[Arguments :] none
    \item[Value :] a NumericalMathFunction, the function $function$
    \end{description}
    \bigskip

  \item $getName$
    \begin{description}
    \item[Usage :] $getName()$
    \item[Arguments :] none
    \item[Value :] a string, the name of the QuadraticTaylor
    \end{description}
    \bigskip

  \item $getCenter$
    \begin{description}
    \item[Usage :] $getCenter()$
    \item[Arguments :] none
    \item[Value :] a NumericalPoint, around which the approximation has been made : $center$
    \end{description}
    \bigskip

  \item $run$
    \begin{description}
    \item[Usage :] $run()$
    \item[Arguments :] none
    \item[Value :] it performs the Quadratic Taylor expansion around $center$
      (while this method has not been executed, only
      $getInputFunction$, $getName$ and $setName$ methods can be used)
    \end{description}
    \bigskip

  \item $getConstant$
    \begin{description}
    \item[Usage :] $getConstant()$
    \item[Arguments :] none
    \item[Value :] a NumericalPoint, the constant vector of the approximation, equal to $function(center)$
    \end{description}
    \bigskip

  \item $getLinear$
    \begin{description}
    \item[Usage :] $getLinear()$
    \item[Arguments :] none
    \item[Value :] a Matrix, the gradient of the function $function$ at the point $center$ (the transposition of the jacobian matrix)
    \end{description}
    \bigskip

  \item $getQuadratic$
    \begin{description}
    \item[Usage :] $getQuadratic()$
    \item[Arguments :] none
    \item[Value :] a SymmetricTensor which contains the 0.5 *  transposition of the hessian values of $function$ at $center$
    \end{description}
    \bigskip

  \item $getResponseSurface$
    \begin{description}
    \item[Usage :] $getResponseSurface()$
    \item[Arguments :] none
    \item[Value :] a NumericalMathFunction, an approximation of the function $function$ by a Quadratic Taylor expansion at  $center$
    \end{description}
  \end{description}

\item[Links :] \rule{0pt}{1em}
  \href{./Version/docref_SurfRep_Taylor_en.pdf}{see docref\_SurfRep\_Taylor}

  The methods $getInputFunction$, $getName$, $getCenter$ have their associated $setMethod$.

\end{description}



% ===================================================================


\newpage \subsection{Least squares approximation}


\subsubsection{LinearLeastSquares}

\begin{description}

\item[Usage :] \rule{0pt}{1em}
  \begin{description}
  \item $LinearLeastSquares(dataIn, function)$
  \item $LinearLeastSquares(dataIn,dataOut)$
  \end{description}

\item[Arguments :]  \rule{0pt}{1em}
  \begin{description}
  \item $dataIn$ : a NumericalSample, the input variables
  \item $function$ : a NumericalMathFunction, the function to be approximated
  \item $dataOut$ : a NumericalSample, the output variables
  \end{description}

\item[Value :] a LinearLeastSquares, the linear least squares approximation between :
  \begin{description}
  \item the two samples  $dataIn$ and $dataOut$ in the case of the second usage
  \item the two samples  $dataIn$ and $function(dataIn)$ in the case of the first usage
  \end{description}

\item[Some methods :]  \rule{0pt}{1em}
  \begin{description}

  \item $getInputFunction$
    \begin{description}
    \item[Usage :] $getInputFunction()$
    \item[Arguments :] none
    \item[Value :] a NumericalMathfunctiontion the $function$ parameter in the case of the first usage
    \end{description}
    \bigskip

  \item $getDataIn$
    \begin{description}
    \item[Usage :] $getDataIn()$
    \item[Arguments :] none
    \item[Value :] a NumericalSample, the $dataIn$ parameter
    \end{description}
    \bigskip

  \item $getName$
    \begin{description}
    \item[Usage :] $getName()$
    \item[Arguments :] none
    \item[Value :] a string, the name of the LinearLeastSquares
    \end{description}
    \bigskip


  \item $run$
    \begin{description}
    \item[Usage :] $run()$
    \item[Arguments :] none
    \item[Value :] it performs the linear least squares approximation (while this method has not been executed, only $getInputfunctiontion$, $getDataIn$, $getName$ and $setName$ methods can be used)
    \end{description}
    \bigskip


  \item $getDataOut$
    \begin{description}
    \item[Usage :] $getDataIn()$
    \item[Arguments :] none
    \item[Value :] a NumericalSample, it returns the ouput variable :
      \begin{description}
      \item in the case of the first usage, it corresponds to the values of the function $function$ at the input variables $dataIn$ : $function(dataIn)$
      \item in the case of the second usage, it corresponds to  $dataOut$
      \end{description}
    \end{description}
    \bigskip

  \item $getLinear$
    \begin{description}
    \item[Usage :] $getLinear()$ %
    \item[Arguments :] none
    \item[Value :] a Matrix, the gradient of the function $function$ at the point $center$ (the transposition of the jacobian matrix)
    \end{description}
    \bigskip

  \item $getResponseSurface$
    \begin{description}
    \item[Usage :] $getResponseSurface()$ %
    \item[Arguments :] none
    \item[Value :] a NumericalMathFunction, an approximation of the function $function$ by Linear Least Squares
    \end{description}

  \end{description}

\item[Links :]
  \href{./Version/docref_SurfRep_LeastSquare_en.pdf}{see docref\_SurfRep\_LeastSquare}
\end{description}

The methods $getInputFunction$, $getName$, $getDataIn$ have their associated $setMethod$.

% ===================================================================


\newpage \subsubsection{QuadraticLeastSquares}

\begin{description}

\item[Usage :]
  \begin{description}
  \item $QuadraticLeastSquares(dataIn, function)$
  \item $QuadraticLeastSquares(dataIn, dataOut)$
  \end{description}

\item[Arguments :]  \rule{0pt}{1em}
  \begin{description}
  \item $dataIn$ : a NumericalSample, the input variables
  \item $function$ : a NumericalMathFunction, the function to be approximated
  \item $dataOut$ : a NumericalSample, the output variables
  \end{description}

\item[Value :] a QuadraticLeastSquares, the quadratic least squares approximation between :
  \begin{description}
  \item the two samples  $dataIn$ and $dataOut$ in the case of the second usage
  \item the two samples  $dataIn$ and $function(dataIn)$ in the case of the first usage
  \end{description}

\item[Some methods :]  \rule{0pt}{1em}
  \begin{description}

  \item $getInputFunction$
    \begin{description}
    \item[Usage :] $getInputFunction$
    \item[Arguments :] none
    \item[Value :] a NumericalMathFunction, the function $function$
    \end{description}
    \bigskip

  \item $getName$
    \begin{description}
    \item[Usage :] $getName()$
    \item[Arguments :] none
    \item[Value :] a string, the name of the QuadraticLeastSquares
    \end{description}
    \bigskip

  \item $run$
    \begin{description}
    \item[Usage :] $run()$
    \item[Arguments :] none
    \item[Value :] it performs the quadratic least squares approximation
      (while this method has not been executed, only
      $getInputFunction$, $getName$ and $setName$ methods can be used)
    \end{description}
    \bigskip

  \item $getDataOut$
    \begin{description}
    \item[Usage :] $getDataIn()$
    \item[Arguments :] none
    \item[Value :] a NumericalSample, it returns the ouput variable :
      \begin{description}
      \item in the case of the first usage, it corresponds to the values of the function $function$ at the input variables $dataIn$ : $function(dataIn)$
      \item in the case of the second usage,  it corresponds to  $dataOut$
      \end{description}
    \end{description}
    \bigskip

  \item $getConstant$
    \begin{description}
    \item[Usage :] $getConstant()$
    \item[Arguments :] none
    \item[Value :] a NumericalPoint, the constant vector of the approximation, equal to $function(center)$
    \end{description}
    \bigskip

  \item $getLinear$
    \begin{description}
    \item[Usage :] $getLinear()$
    \item[Arguments :] none
    \item[Value :] a Matrix, the linear matrix of the approximation
    \end{description}
    \bigskip


  \item $getQuadratic$
    \begin{description}
    \item[Usage :] $getQuadratic()$
    \item[Arguments :] none
    \item[Value :] a SymmetricTensor, the quadratic term of the approximation
    \end{description}

  \item $getResponseSurface$
    \begin{description}
    \item[Usage :] $getResponseSurface()$ %
    \item[Arguments :] none
    \item[Value :] a NumericalMathFunction, an approximation of the function $function$ by Quadratic Least Squares
    \end{description}

  \end{description}

\item[Links :]\rule{0pt}{1em}
  \href{./Version/docref_SurfRep_LeastSquare_en.pdf}{see docref\_SurfRep\_LeastSquare}
\end{description}

The methods $getInputFunction$, $getName$, $getDataIn$ have their associated $setMethod$.

% ===================================================================

