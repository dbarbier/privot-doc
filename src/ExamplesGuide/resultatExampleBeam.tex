###################################################
Min/Max study with deterministic experiment plane
###################################################
From a composite  experiment plane of size =  73
Levels =  0.5 ,  1.0 ,  3.0
Min Value =  0.649717975365
Max Value =  55.3605185131

#################################
Min/Max study by random sampling
#################################
From random sampling =  10000
Min Value =  5.38682360207
Max Value =  52.7553885642


###########################################
Random Study : central tendance of
the output variable of interest
###########################################

##############################
Taylor variance decomposition
##############################

First order mean= 12.3369023123
Evaluation calls number =  1
Second order mean= 12.4198129769
Evaluation calls number =  33
Standard deviation= 4.18703072295
Evaluation calls number =  8
Importance factors=
E  =  0.149096880954
F  =  0.781344650861
L  =  0.0145457110918
I  =  0.0550127570932

#######################
Random sampling
#######################
Sample size =  10000
Mean from sample =  12.609042535
Standard deviation from sample =  4.35926575014

##########################
# Kernel Smoothing Fitting
##########################
Sample size =  10000
Kernel bandwidth= 0.688228112153
Mean from kernel smoothing =  12.619538853


############################################################
Probabilistic Study : threshold exceedance: deviation <-1cm
############################################################

#####
FORM
#####
FORM event probability= 0.00670980421088
Number of evaluations of the limit state function =  176
Generalized reliability index= 2.47243508163
Standard space design point=
E  =  -0.602386403812
F  =  2.31055515463
L  =  0.355793665542
I  =  -0.533677429099
Physical space design point=
E  =  30327158.555
F  =  61318.4694411
L  =  256.390024534
I  =  378.634729684
Importance factors=
E  =  0.058682003115
F  =  0.863350794277
L  =  0.0204715646194
I  =  0.0574956379882
Hasofer reliability index= 2.47243508163

############
Monte Carlo
############

Number of evaluations of the limit state function =  18300
Monte Carlo probability estimation =  0.00551912568306
Variance of the Monte Carlo probability estimator =  3.02926673715e-07
0.95 Confidence Interval = [ 0.00444038551844 ,  0.00659786584768 ]

#######################
Directional Sampling
#######################
Number of evaluations of the limit state function =  18258
Directional Sampling probability estimation =  0.0048902194515
Variance of the Directional Sampling probability estimator =  2.39120163922e-07
0.95 Confidence Interval = [ 0.0039317987386 ,  0.00584864016439 ]

###########################
Latin HyperCube Sampling
###########################
Number of evaluations of the limit state function =  20600
LHS probability estimation =  0.00490291262136
Variance of the LHS probability estimator =  2.39206700255e-07
0.95 Confidence Interval = [ 0.00394431850044 ,  0.00586150674228 ]

####################
Importance Sampling
####################
Number of evaluations of the limit state function =  306
Importance Sampling probability estimation =  0.00658159419011
Variance of the Importance Sampling probability estimator =  4.29506441412e-07
0.95 Confidence Interval = [ 0.00529709767088 ,  0.00786609070933 ]


##########################
Polynomial expansion chaos
##########################

Polynomial number  0  in truncated basis <-> polynomial number  0  =  [0,0,0,0]  in complete basis
Polynomial number  1  in truncated basis <-> polynomial number  1  =  [1,0,0,0]  in complete basis
Polynomial number  2  in truncated basis <-> polynomial number  2  =  [0,1,0,0]  in complete basis
Polynomial number  3  in truncated basis <-> polynomial number  3  =  [0,0,1,0]  in complete basis
Polynomial number  4  in truncated basis <-> polynomial number  4  =  [0,0,0,1]  in complete basis
Polynomial number  5  in truncated basis <-> polynomial number  5  =  [2,0,0,0]  in complete basis
Polynomial number  6  in truncated basis <-> polynomial number  6  =  [1,1,0,0]  in complete basis
Polynomial number  7  in truncated basis <-> polynomial number  7  =  [1,0,1,0]  in complete basis
Polynomial number  8  in truncated basis <-> polynomial number  8  =  [1,0,0,1]  in complete basis
Polynomial number  9  in truncated basis <-> polynomial number  9  =  [0,2,0,0]  in complete basis
Polynomial number  10  in truncated basis <-> polynomial number  10  =  [0,1,1,0]  in complete basis
Polynomial number  11  in truncated basis <-> polynomial number  11  =  [0,1,0,1]  in complete basis
Polynomial number  12  in truncated basis <-> polynomial number  12  =  [0,0,2,0]  in complete basis
Polynomial number  13  in truncated basis <-> polynomial number  13  =  [0,0,1,1]  in complete basis
Polynomial number  14  in truncated basis <-> polynomial number  14  =  [0,0,0,2]  in complete basis
Polynomial number  15  in truncated basis <-> polynomial number  15  =  [3,0,0,0]  in complete basis
Polynomial number  16  in truncated basis <-> polynomial number  16  =  [2,1,0,0]  in complete basis
Polynomial number  17  in truncated basis <-> polynomial number  18  =  [2,0,0,1]  in complete basis
Polynomial number  18  in truncated basis <-> polynomial number  19  =  [1,2,0,0]  in complete basis
Polynomial number  19  in truncated basis <-> polynomial number  20  =  [1,1,1,0]  in complete basis
Polynomial number  20  in truncated basis <-> polynomial number  26  =  [0,2,1,0]  in complete basis
Polynomial number  21  in truncated basis <-> polynomial number  29  =  [0,1,1,1]  in complete basis
Polynomial number  22  in truncated basis <-> polynomial number  30  =  [0,1,0,2]  in complete basis
Polynomial number  23  in truncated basis <-> polynomial number  31  =  [0,0,3,0]  in complete basis
Polynomial number  24  in truncated basis <-> polynomial number  33  =  [0,0,1,2]  in complete basis
Polynomial number  25  in truncated basis <-> polynomial number  50  =  [1,1,0,2]  in complete basis
Polynomial number  26  in truncated basis <-> polynomial number  56  =  [0,3,1,0]  in complete basis
Polynomial number  27  in truncated basis <-> polynomial number  62  =  [0,1,2,1]  in complete basis
Polynomial number  28  in truncated basis <-> polynomial number  64  =  [0,1,0,3]  in complete basis
Polynomial number  29  in truncated basis <-> polynomial number  65  =  [0,0,4,0]  in complete basis
Polynomial number  30  in truncated basis <-> polynomial number  105  =  [0,5,0,0]  in complete basis
Polynomial number  31  in truncated basis <-> polynomial number  111  =  [0,2,3,0]  in complete basis
Polynomial number  32  in truncated basis <-> polynomial number  120  =  [0,0,5,0]  in complete basis
Polynomial number  33  in truncated basis <-> polynomial number  122  =  [0,0,3,2]  in complete basis
Polynomial number  34  in truncated basis <-> polynomial number  131  =  [4,1,1,0]  in complete basis
Polynomial number  35  in truncated basis <-> polynomial number  147  =  [2,3,1,0]  in complete basis
Polynomial number  36  in truncated basis <-> polynomial number  150  =  [2,2,1,1]  in complete basis
Polynomial number  37  in truncated basis <-> polynomial number  163  =  [1,4,0,1]  in complete basis
Polynomial number  38  in truncated basis <-> polynomial number  238  =  [3,1,1,2]  in complete basis
Polynomial number  39  in truncated basis <-> polynomial number  248  =  [2,3,2,0]  in complete basis
Polynomial number  40  in truncated basis <-> polynomial number  249  =  [2,3,1,1]  in complete basis
Polynomial number  41  in truncated basis <-> polynomial number  266  =  [1,6,0,0]  in complete basis
Polynomial number  42  in truncated basis <-> polynomial number  278  =  [1,2,2,2]  in complete basis
Polynomial number  43  in truncated basis <-> polynomial number  294  =  [0,7,0,0]  in complete basis
Polynomial number  44  in truncated basis <-> polynomial number  345  =  [5,1,0,2]  in complete basis
Polynomial number  45  in truncated basis <-> polynomial number  367  =  [3,4,0,1]  in complete basis
Polynomial number  46  in truncated basis <-> polynomial number  399  =  [2,2,1,3]  in complete basis
Polynomial number  47  in truncated basis <-> polynomial number  437  =  [1,1,4,2]  in complete basis
Polynomial number  48  in truncated basis <-> polynomial number  450  =  [0,8,0,0]  in complete basis
Polynomial number  49  in truncated basis <-> polynomial number  481  =  [0,1,4,3]  in complete basis

Distribution in the tansformed variables =  class=ComposedDistribution name=ComposedDistribution dimension=4 copula=class=IndependentCopula name=IndependentCopula dimension=4 marginal[0]=class=Beta name=Beta dimension=1 r=0.93 t=3.2 a=-1 b=1 marginal[1]=class=Gamma name=Gamma dimension=1 k=2.78 lambda=1 gamma=0 marginal[2]=class=Uniform name=Uniform dimension=1 a=-1 b=1 marginal[3]=class=Beta name=Beta dimension=1 r=2.5 t=4 a=-1 b=1

Mean = class=NumericalPoint name=Unnamed dimension=1 implementation=class=NumericalPointImplementation name=Unnamed dimension=1 values=[12.6138]
Standard deviation = 4.23422289923

