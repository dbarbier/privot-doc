% Copyright (c)  2005-2009  EDF-EADS-PHIMECA.
% Permission is granted to copy, distribute and/or modify this document
% under the terms of the GNU Free Documentation License, Version 1.2
% or any later version published by the Free Software Foundation;
% with no Invariant Sections, no Front-Cover Texts, and no Back-Cover
% Texts.  A copy of the license is included in the section entitled "GNU
% Free Documentation License".

Among the requirements expressed by the users during the initial design phase of the \OT\ platform, the wish to undertake uncertainty treatment studies with any code, whether it be as simple as an analytical mathematical formula or a coupling involving several computational codes dedicated to the resolution of a very complex physical problem, was crucial.

Therefore, it was necessary to design a system that can be very flexible, efficient and scalable, in order to answer the users' needs while preserving some fundamental principles of software engineering, in particular the independence of the \OT\ platform from the code with which it collaborates, and the separation of roles between developers and users.

What can be seen, at first, as a "squaring of the circle" type problem was achieved through this mechanism coupling \OT\ with computational codes. This "wrapping" process relies on the use of many techniques that are sometimes complex: dynamic library loading, interface definition, delegation of processing, genericity, parallelism, etc.

However, much attention has been paid to simplifying or even hiding all of this, so as to make \OT\ accessible to anyone wishing to design a \index{wrapper}wrapper for a computational code. The documentation reflects this care for simplification. However, some prerequisites are essential to fully understand it. This document assumes that the reader is familiar with programming, at least with traditional procedural languages such as \index{FORTRAN}FORTRAN and C. Although the \OT\ platform is mainly written in C++, no specific knowledge of this language is required to design a wrapper: this is the first element of the above-mentioned simplification.

The technique suggested in \OT\ tends to simplify itself as the product evolves. Some features are replaced, as the project goes along, by simplifications that hide many elements unnecessary to the developer. We invite the reader to always take advantage of these innovations, not least because they bring clarity and maintainability to the source code. However, nothing has been fundamentally modified since the first \OT\ versions: what has been developed in the past continues to operate as long as all is recompiled with the new version of the platform.