% Copyright (c)  2005-2009  EDF-EADS-PHIMECA.
% Permission is granted to copy, distribute and/or modify this document
% under the terms of the GNU Free Documentation License, Version 1.2
% or any later version published by the Free Software Foundation;
% with no Invariant Sections, no Front-Cover Texts, and no Back-Cover
% Texts.  A copy of the license is included in the section entitled "GNU
% Free Documentation License".

There are three directories containing the templates used to interface \OT\ with an external code written in C, in \index{FORTRAN}FORTRAN 77 or available as a command that can be called from the shell.

These directories share the same structure described here. The specifics of each directory are outlined in the relevant sections.

The developer interested in any of these directories must start by copying it entirely in a personal directory. We will assume that the copy is made in the developer's home directory ({\bf \$HOME}) that the produced wrapper will be called {\bf myWrapper}.
\lstset{language=Bash, basicstyle=\normalsize}
\begin{lstlisting}[frame=TBRL]
  > cp -r <template_dir> $HOME/myWrapper
\end{lstlisting}

Then, after ensuring that they have the {\bf autoconf}, {\bf automake} and {\bf libtool} tools, the developer enters the newly copied directory and initializes the development environment. This step is performed only once.

\lstset{language=Bash, basicstyle=\normalsize}
\begin{lstlisting}[frame=TBRL]
  > cd myWrapper
  > sh customize myWrapper
  > sh bootstrap
\end{lstlisting}

\small
\begin{quote}
  \textit{Note: The {\bf customize} script modifies a number of files in which the name of the wrapper ({\bf myWrapper} in our case) must appear.}
\end{quote}
\normalsize

At this stage, a new {\bf configure} file has been created in the directory. This file will allow to continue with the procedure. This executable requires, however, two pieces of information:
\begin{itemize}
\item the name of the \index{installation directory}installation directory for the produced wrapper. By default, this wrapper will be written in the directory {\bf \$HOME/openturns}, which is a standard directory for \OT\ to search for wrappers. However, if you choose to install the wrapper elsewhere, you will have to declare this path in the \index{wrapper search}{\bf OPENTURNS\_WRAPPER\_PATH} environment variable.
\item the \OT\ installation directory, i.e the directory providing access to {\bf lib/openturns/libOT.so}. We will call this directory {\bf openturns\_dir}.
\end{itemize}

\lstset{language=Bash, basicstyle=\normalsize}
\begin{lstlisting}[frame=TBRL]
  > ./configure --with-openturns=<openturns_dir>
\end{lstlisting}

This command produces the {\bf Makefile} files needed to \index{compilation}compile the wrapper.
\lstset{language=Bash, basicstyle=\normalsize}
\begin{lstlisting}[frame=TBRL]
  > make
  > make install
\end{lstlisting}

All of the directories contain a file called \index{wrapper source code}{\bf wrapper.c} defining symbols for functions expected by the \OT\ library. In some cases, these functions are written as macros or as a traditional C code.

Finally, a Python test file allows to control the minimal functioning of the produced wrapper.

\lstset{language=Bash, basicstyle=\normalsize}
\begin{lstlisting}[frame=TBRL]
  > python test.py
\end{lstlisting}

\subsection{wrapper\_linked\_with\_C\_function directory}

This directory contains two additional files that encode a hypothetical function written in C and serving as the external code:
\begin{description}
\item[myCFunction.h] the header file declaring the function;
\item[myCFunction.c] the source file defining the function.
\end{description}

The function here is called {\bf myCFunction} and shows how to pass the input vector ({\bf inPoint}) as an argument and how to retrieve the result in an output vector ({\bf outpoint}). We remind here that no memory allocation is needed.

The {\bf wrapper.c} file imports the definition of {\bf myCFunction} using the directive:
\lstset{language=C++, basicstyle=\normalsize}
\begin{lstlisting}[frame=TBRL]
  #include "myCFunction.h"
\end{lstlisting}
It explicitely calls this function in the {\bf FUNC\_EXEC} macro.

\subsection{wrapper\_linked\_with\_F77\_function directory}

This directory contains only one additional file encoding a hypothetical function written in FORTRAN, which acts as the external code:
\begin{description}
\item[code.f] the source file defining the function.
\end{description}

The function here is called {\bf COMPUTE} and shows how to pass the input vector ({\bf inPoint}) as an argument and how to retrieve the result in an output vector ({\bf outpoint}). We remind here that no memory allocation is needed.

Since there is no header file allowing to import the {\bf wrapper.c} file where the {\bf COMPUTE} function is declared, and since it is necessary to adapt the symbols between both languages C and FORTRAN, we must call the {\bf F77\_FUNC} macro defined by {\bf autoconf}. This macro produces a new symbol to be called {\bf COMPUTE\_F77}, which will be called directly in the {\bf FUNC\_EXEC} macro.

\subsection{wrapper\_calling\_shell\_command directory}

This directory contains two files used to simulate an external code called from the shell command line:
\begin{description}
\item[code\_C1.c] the self-standing source code processing the numerical input data;
\item[code\_C1.data] the file containing the numerical input data expected by {\bf code\_C1}.
\end{description}

At compile time, the {\bf code\_C1.c} file will be used to produce the {\bf code\_C1} executable, which takes two files as arguments on its command line:
\begin{description}
\item[code\_C1.data] the name of the input file containing the numerical data;
\item[code\_C1.res] the name of the output file containing the numerical results.
\end{description}

\lstset{language=Bash, basicstyle=\normalsize}
\begin{lstlisting}[frame=TBRL]
  > ./code_C1 code_C1.data code_C1.res
\end{lstlisting}

The macros listed in the {\bf wrapper.c} file, and the file and variable description in the {\bf myWrapper.xml.in} file, induce that the call to the {\bf code\_C1} external code is correctly performed, as well as the data transfer for input and output.
