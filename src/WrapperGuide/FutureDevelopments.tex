% Copyright (c)  2005-2009  EDF-EADS-PHIMECA.
% Permission is granted to copy, distribute and/or modify this document
% under the terms of the GNU Free Documentation License, Version 1.2
% or any later version published by the Free Software Foundation;
% with no Invariant Sections, no Front-Cover Texts, and no Back-Cover
% Texts.  A copy of the license is included in the section entitled "GNU
% Free Documentation License".

If the interfacing technique developed for \OT\ has not significantly evolved in its principles since the inception of the platform, it has experienced many improvements that always strive for a simplification of the writing of wrappers, in order to make it accessible to as many users as possible and to enrich the library with new features.

The introduction of the \index{wrapper}wrapper library, of \index{regular expressions}regular expressions and their shortcuts, or of macros, are part of the most emblematic recent developments.

Other developments, however, are in the pipeline and will be introduced over time. These include (the order being no indication of priority):
\begin{itemize}
\item variable search and \index{substitution}substitution withinin binary files, especially MED-format files%\footnote{See {\bf http://www.code-aster.org/outils/med/html/introduction.html}.}
  . These files often containing mesh elements, it would then be possible to turn geometric variables into probabilistic ones.
\item batch execution of external codes on computational servers of any kind. In doing so, \OT\ would be able to launch a large number of simulations on far more powerful machines, thereby enabling much better quality probabilistic studies.
\end{itemize}
