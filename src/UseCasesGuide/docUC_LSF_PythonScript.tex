% Copyright (c)  2005-2010 EDF-EADS-PHIMECA.
% Permission is granted to copy, distribute and/or modify this document
% under the terms of the GNU Free Documentation License, Version 1.2
% or any later version published by the Free Software Foundation;
% with no Invariant Sections, no Front-Cover Texts, and no Back-Cover
% Texts.  A copy of the license is included in the section entitled "GNU
% Free Documentation License".
\renewcommand{\filename}{docUCLSF_PythonScript.tex}
\renewcommand{\filetitle}{UC : From a fonction defined in the script python}

% \HeaderNNIILevel
% \HeaderIILevel
\HeaderIIILevel

\label{OpenTURNSPythonFunction}

\index{Limit State Function!Function declarde in the script python}


The objective of this Use Case is to create the limit state function, from a function defined in the script python. Open TURNS automatically gives to the analytical formula an implementation for the gradient and the hessian : by default,
\begin{itemize}
\item the gradient evaluation method is the  centered finite difference method, with the differential increment $h=1e-5$ for each direction,
\item the hessian evaluation method is the  centered finite difference method, with the differential increment $h=1e-4$ for each direction.
\end{itemize}
It is possible to change the evaluation method for the gradient or the hessian. The following Use Case shows how to proceed.\\


Details on each object may be found in the User Manual  (\href{OpenTURNS_UserManual_TUI.pdf}{see User Manual - Base Objects / NumericalMathFunction}).\\


In order to be able to use the function with the {\itshape openturns} library, it is necessary to define a class which derives from {\itshape OpenTURNSPythonFunction} as indicated belows. The example here is the functions {\itshape modelePYTHON} and {\itshape modelePYTHON2}:
\begin{equation}
  \begin{array}{l|lcl}
    modelePYTHON : & \mathbb{R}^4 & \rightarrow & \mathbb{R} \\
    & (E,F,L,I)    & \mapsto     & \displaystyle \frac{FL^3}{3EI}
  \end{array}
\end{equation}

\begin{equation}
  \begin{array}{l|lcl}
    modelePYTHON2 : & \mathbb{R}^4 & \rightarrow & \mathbb{R}^2 \\
    & (a,b,c)    & \mapsto     & \displaystyle (a^2, abc)
  \end{array}
\end{equation}


\requirements{
  \begin{description}
  \item[$\bullet$] none
  \end{description}
}
{
  \begin{description}
  \item[$\bullet$] the python function assoiated to the model: {\itshape modelePYTHON}
  \item[type:] a OpenTURNSPythonFunction
  \item[$\bullet$] the limit state function : {\itshape modeleOpenTURNS}
  \item[type:] a NumericalMathFunction
  \end{description}
}

\textspace\\
Python script for this UseCase :

\begin{lstlisting}
  # CASE 1 : function : R^4 --> R

  # Create here the python lines to define the implementation of the function

  # In order to be able to use that function with the openturns library,
  # it is necessary to define a class which derives from OpenTURNSPythonFunction

  class modelePYTHON(OpenTURNSPythonFunction) :
  # that following method defines the input size (4) and the output size (1)
  def __init__(self) :
  OpenTURNSPythonFunction.__init__(self,4,1)

  # that following method gives the implementation of modelePYTHON
  def _exec(self,x) :
  E=x[0]
  F=x[1]
  L=x[2]
  I=x[3]
  return [-(F*L*L*L)/(3.*E*I)]

  # Use that function defined in the script python
  # with the openturns library
  # Create a NumericalMathFunction from modelePYTHON
  modeleOpenTURNS = NumericalMathFunction(modelePYTHON())


  # CASE 2 : function : R^3 --> R^2

  # Create here the python lines to define the implementation of the function

  # In order to be able to use that function with the openturns library,
  # it is necessary to define a class which derives from OpenTURNSPythonFunction

  class modelePYTHON2(OpenTURNSPythonFunction) :
  # that following method defines the input size (3) and the output size (2)
  def __init__(self) :
  OpenTURNSPythonFunction.__init__(self,3,2)

  # that following method gives the implementation of modelePYTHON
  def _exec(self,x) :
  a=x[0]
  b=x[1]
  c=x[2]
  return [-a*a,a*b*c]

  # Use that function defined in the script python
  # with the openturns library
  # Create a NumericalMathFunction from modelePYTHON2
  modeleOpenTURNS2 = NumericalMathFunction(modelePYTHON2())
\end{lstlisting}

