% Copyright (c)  2005-2010 EDF-EADS-PHIMECA.
% Permission is granted to copy, distribute and/or modify this document
% under the terms of the GNU Free Documentation License, Version 1.2
% or any later version published by the Free Software Foundation;
% with no Invariant Sections, no Front-Cover Texts, and no Back-Cover
% Texts.  A copy of the license is included in the section entitled "GNU
% Free Documentation License".
\renewcommand{\filename}{docUC_InputWithData_StatisticalManipulation.tex}
\renewcommand{\filetitle}{UC :  Statistical manipulations on data : min, max, covariance, skewness, kurtosis, quantile, empirical CDF, Pearson, Kendall and Spearman correlation matrixes and rank/sort functionnalities}

% \HeaderNNIILevel
% \HeaderIILevel
\HeaderIIILevel

\label{statistical}


\index{Sample Statistics!Min - Max}
\index{Sample Statistics!Moments evaluation}
\index{Sample Statistics!Covariance}
\index{Sample Statistics!Skewness}
\index{Sample Statistics!Kurtosis}
\index{Sample Statistics!Empirical quantile}
\index{Sample Statistics!Empirical cumulative density function}
\index{Sample Statistics!Pearson correlation coefficient}
\index{Sample Statistics!Kendall's tau}
\index{Sample Statistics!Spearman correlation coefficient}
\index{Sample Statistics! Rank - Sort functionnalities}
\index{Sample Statistics!Cholesky factor}
\index{Quantile!Empirical estimation}



The objective of this Use Case is to describe the main statistical functionalities that Open TURNS enables to manipulate some data, represented by a NumericalSample.\\




Details on each object may be found in the User Manual  (\href{OpenTURNS_UserManual_TUI.pdf}{see User Manual - Statistics on sample / Numerical Sample}).\\




Open TURNS enables to calculate per components :
\begin{itemize}
\item min and max per component, with the methods {\itshape getMin, getMax}
\item range per component, with the method {\itshape computeRangePerComponent}
\item mean, variance, standard deviation , skewness and kurtosis  per component, with the methods {\itshape computeMean, computeVariancePerComponent, computeStandardDeviationPerComponent, computeSkewnessPerComponent, computeKurtosisPerComponent}
\item empirical median and other quantiles per component, with the methods {\itshape computeMedianPerComponent, computeQuantilePerComponent}
\end{itemize}

Open TURNS enables some global calculs :
\begin{itemize}
\item covariance of the sample, with the methods {\itshape computeCovariance},
\item standard deviation of the sample : the Cholesky factor of the covariance matrix, with the methods {\itshape computeStandardDeviation },
\item Pearson, Kendall and Spearman correlation matrix, with the methods {\itshape computePearsonCorrelation, computeKendallTau, computeSpearmanCorrelation},
\item empirical CDF evaluated on a point $\vect{x}$ : $\mathcal{P}(X_1 \leq x_1, \hdots, X_n \leq x_n)$, with the methods {\itshape computeEmpiricalCDF},
\item empirical tail CDF evaluated on a point $\vect{x}$ : $\mathcal{P}(X_1 > x_1, \hdots, X_n > x_n)$, with the methods {\itshape computeEmpiricalCDF},
\item empirical quantiles, with the method {\itshape computeQuantile}. For the quantile $x_q$ of order $q$, Open TURNS proceeds as follows : 
\begin{itemize}
   \item $\forall q \in [\frac{1}{2n}, 1-\frac{1}{2n}]$, then Open TURNS approximates the empirical  cumulative density function by interpolating all the middles of the steps and then evaluates  $x_q$ from this continuous approximation. 
   \item $\forall q \leq \frac{1}{2n}$, then Open TURNS returns $min(X_i)$.
   \item $\forall q > \frac{1}{2n}$, then Open TURNS returns $max(X_i)$.
\end{itemize}
\end{itemize}

At last, it is possible :
\begin{itemize}
\item to copy into a NumericalSample whose components are the respective ranks of the components, with the method {\itshape rank},
\item to copy into a NumericalSample whose components are all sorted in ascending order, with the method {\itshape sort },
\item to extract the $(i+1)$ component whose components are all sorted in ascending order, with the method {\itshape sort(i) },
\item to copy into a NumericalSample whose NumericalPoints are reordered such that the $(i+1)$ component is sorted in ascending order, with the method  {\itshape sortAccordingAComponent(i)},
\item to keep from the Numericalsample only the $i$ first points, with the method  {\itshape split(i)},
\item to translate the points of the NumericalSample, with the method  {\itshape translate },
\item to multiply all the components of the points by a factor, with the method  {\itshape  scale},
\item to remove a particular point from the NumericalSample, with the method  {\itshape  erase}.
\end{itemize}



\requirements{
  \begin{description}
  \item[$\bullet$] a numerical sample : {\itshape sample}
  \item[type:]  NumericalSample
  \end{description}
}
{
  \begin{description}
  \item[$\bullet$] statistical elements listed previously
  \item[type:]  NumericalPoint, SquareMatrix or CorrelationMatrix
  \end{description}
}

\textspace\\
Python script for this UseCase :

\begin{lstlisting}
  # Get min and max per component
  print "Min per component =" , sample.getMin()
  print "max per component =" , sample.getMax()

  # Get the range per component
  print "Range per component =" , sample.computeRangePerComponent()

  # Get the mean per component
  print "Mean =" , sample.computeMean()

  # Get the standard deviation per component
  print "Standard deviation per component =" , sample.computeStandardDeviationPerComponent()

  # Get the Variance per component
  print "Variance =" , sample.computeVariancePerComponent()

  # Get the Skewness per component
  print "Skewness =" , sample.computeSkewnessPerComponent()

  # Get the Kurtosis per component
  print "Kurtosis =" , sample.computeKurtosisPerComponent()

  # Get the median per component
  print "Median per component =" , sample.computeMedianPerComponent()

  # Get the empirical 0.95 quantile per component
  print "0.95 quantile per component =" , sample.computeQuantilePerComponent(0.95)

  # Get the sample covariance
  print "Covariance =" , sample.computeCovariance()

  # Get the sample standard deviation
  print "Standard deviation =" , sample.computeStandardDeviation()

  # Get the sample  Pearson correlation matrix
  print "Pearson correlation =" , sample.computePearsonCorrelation()

  # Get  the sample Kendall  correlation matrix
  print "Kendall correlation =" , sample.computeKendallCorrelation()

  # Get  the sample Spearman  correlation matrix
  print "Spearman correlation =" , sample.computeSpearmanCorrelation()

  # Get the value of the empirical CDF at point POINT
  print "Empirical CDF at point POINT = " , sample.computeEmpiricalCDF(POINT)

  # Get the value of the empirical CDF at point POINT
  print "Empirical CDF at point POINT = " , sample.computeEmpiricalCDF(POINT, True)

  # Get the empirical 0.95 quantile
  print "0.95 quantile  =" , sample.computeQuantile(0.95)
\end{lstlisting}
\textspace\\



To illustrate each method, we give here an example in dimension 2 : consider the following NumericalSample $numSample = [(1.3, 1.2); (4.1, 1.0); (2.3, 2.7)]$. Then,

At last, it is possible :
\begin{itemize}
\item $new = numSample.rank()$ : $new = [(0,1); (2,0); (1,2)]$,
\item $new = numSample.sort()$ : $new = [(1.3,1.0); (2.3,1.2); (4.1,2.7)]$,
\item $new = numSample.sort(0))$ : $new = [(1.3); (2.3); (4.1)]$,
\item $new = numSample.sortAccordingAComponent(1)$ : $new  = [(4.1, 1.0);(1.3, 1.2);(2.3, 2.7)]$,
\item $new = numSample.split(2)$ : $new  = [(2.3, 2.7)]$ and $numSample = [(1.3, 1.2); (4.1, 1.0)]$,
\item $new = numSample.translate(NumericalPoint(2,1.0) $ : $new  = [(2.3, 2.2); (4.1, 2.0); (3.3, 3.7)]$,
\item $new = numSample.scale(NumericalPoint(2,2.0) $ : $new  = [(2.6, 2.4); (8.2, 2.0); (4.6, 5.4)]$,
\item $new = numSample.erase(1) $ : $new  = [(1.3, 1.2); (2.3, 2.7)]$.
\end{itemize}
