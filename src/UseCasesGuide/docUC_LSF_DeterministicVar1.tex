% Copyright (c)  2005-2010 EDF-EADS-PHIMECA.
% Permission is granted to copy, distribute and/or modify this document
% under the terms of the GNU Free Documentation License, Version 1.2
% or any later version published by the Free Software Foundation;
% with no Invariant Sections, no Front-Cover Texts, and no Back-Cover
% Texts.  A copy of the license is included in the section entitled "GNU
% Free Documentation License".
\renewcommand{\filename}{docUC_LSF_DeterministicVar1.tex}
\renewcommand{\filetitle}{UC : Introducing some deterministic variables, using a LinearNumericalMathFunction}

% \HeaderNNIILevel
% \HeaderIILevel
\HeaderIIILevel


\label{linearNumericalMathFunction}

\index{Limit State Function!Reducing the initial limit state function}
\index{Limit State Function!LinearNumericalMathFunction}


The objective of this Use Case is to restrict a model which has been previously declared, to some of its variables.\\

Details on each object may be found in the User Manual  (\href{OpenTURNS_UserManual_TUI.pdf}{see User Manual - Base Objects / NumericalMathFunction}).\\



We suppose that the following limit state function {\itshape limitStateFunc} has been created in Open TURNS :
$$
\begin{array}{l|lcl}
  limitStateFunc : & \mathbb{R}^n  & \rightarrow & \mathbb{R}^p \\
  & \vect{X}      & \mapsto     & limitStateFunc(\vect{X})
\end{array}
$$
Suppose now that some of the input variables are deterministic : the random input vector is reduced to a subvector of $\vect{X}$ : $\vect{X}_{prob} \in \mathbb{R}^{n_{prob}}$, with $n_{prob} \leq n$.\\
Let's note $\vect{X} = (\vect{X}_{prob}, \vect{X}_{det})$.\\

In order to create the new limit state function associated to the random input vector $\vect{X}_{prob}$, it is necessary to compose the initial limit state function $limitStateFunc$ with the linear function $increase$ defined by :
$$
\begin{array}{l|lcl}
  increase : & \mathbb{R}^{n_{prob}}  & \rightarrow & \mathbb{R}^n \\
  & \vect{X}_{prob}               & \mapsto     & increase(\vect{X}_{prob}) = \mat{A} \vect{X}_{prob} + \vect{B}
\end{array}
$$
where $\mat{A}$ is the matrix in $\mathbb{M}_{n, n_{prob}}(\mathbb{R})$ defined by :
$$
\mat{A}= \left(
  \begin{array}{c}
    1_{n_{prob}}\\
    0
  \end{array}
\right)
$$
and $\vect{B}$ the vector in $\mathbb{R}^{n}$ defined by :
$$
\vect{B} = \left(
  \begin{array}{c}
    \vect{0}  \\
    \vect{X}_{det}
  \end{array}
\right)
$$

Then, the new limit state function associated to the random input vector $\vect{X}_{prob}$ is $$
newLimitStateFunc = limitStateFunc \circ increase
$$
defined by :
$$
\begin{array}{l|lcl}
  newLimitStateFunc : & \mathbb{R}^{n_{prob}}  & \rightarrow & \mathbb{R}^p \\
  & \vect{X}_{prob}                & \mapsto     & newLimitStateFunc(\vect{X}_{prob})
\end{array}
$$

The example here is the limit state function {\itshape poutre} defined in Eq.(\ref{equatPoutre}) and the random input vector $(E,F,L,I)$ that is reduced to the subvector $(E,F)$. The other variables $(L,I)$ are fixed to $(10.0, 5.0)$.\\

\requirements{
  \begin{description}
  \item[$\bullet$] the initial limit state function : {\itshape poutre}
  \item[type:] LinearNumericalMathFunction $(\mathbb{R}^4 \rightarrow \mathbb{R})$
  \end{description}
}
{
  \begin{description}
  \item[$\bullet$] the {\itshape increase} function
  \item[type:] NumericalMathFunction $(\mathbb{R}^2 \rightarrow \mathbb{R}^4)$
  \item[$\bullet$]  the new limit state function : {\itshape poutreReduced = poutre $\circ$ increase}
  \item[type:] NumericalMathFunction $(\mathbb{R}^2 \rightarrow \mathbb{R})$
  \end{description}
}

\textspace\\
Python script for this UseCase :

\begin{lstlisting}
  # Dimension of the random input vector
  stochasticDimension = 2

  # Dimension of the deterministic input vector
  deterministicDimension = 2

  # Dimension of the input vector of the limit state function 'poutre'
  inputDim = poutre.getInputNumericalPointDimension()

  # Fixe deterministic values for the two last variables
  # of the input vecteor (E,F,L,I)
  # L
  X2 = 10.0
  # I
  X3 = 5.0

  # Create the 'increase' linear function
  # a LinearNumericalMathFunction expression is :
  # linear * (X- center) + constant
  # center = null
  center = NumericalPoint(stochasticDimension)

  # constant term = (0.0, 0.0, X2, X3)^t
  constant = NumericalPoint(inputDim)
  constant[0] =  0.0
  constant[1] =  0.0
  constant[2] =  X2
  constant[3] =  X3

  # Linear term (lines number, columns number)
  linear = Matrix(inputDim,  stochasticDimension)
  linear[0,0] = 1.0
  linear[0,1] = 0.0
  linear[1,0] = 0.0
  linear[1,1] = 1.0
  linear[2,0] = 0.0
  linear[2,1] = 0.0
  linear[3,0] = 0.0
  linear[3,1] = 0.0

  # 'increase' = linear * (X- center) + constant
  increase = LinearNumericalMathFunction(center, constant, linear, "increase")

  # Create the new limit state function :
  # 'poutreReduced = poutre o increase'
  poutreReduced = NumericalMathFunction(poutre, increase)

  # Check if it worked
  x = NumericalPoint(increase.getInputNumericalPointDimension()
  x[0] = 50.0
  x[1] = 1.0
  print "poutreReduced(x)=", poutreReduced(x)
  xRef = NumericalPoint(inputDim)
  xRef[0] = x[0]
  xRef[1] = x[1]
  xRef[2] = X2
  xRef[3] = X3
  print "ref=", externalCode(xRef)
\end{lstlisting}

