% Copyright (c)  2005-2010 EDF-EADS-PHIMECA.
% Permission is granted to copy, distribute and/or modify this document
% under the terms of the GNU Free Documentation License, Version 1.2
% or any later version published by the Free Software Foundation;
% with no Invariant Sections, no Front-Cover Texts, and no Back-Cover
% Texts.  A copy of the license is included in the section entitled "GNU
% Free Documentation License".
\renewcommand{\filename}{docUC_MinMax_MixedDetRandExperimentPlane.tex}
\renewcommand{\filetitle}{UC : Creation of a mixed deterministic / random experiment plane}

% \HeaderNNIILevel
% \HeaderIILevel
\HeaderIIILevel

\label{determRandomExpPlane}

\index{Experiment Plane!Axial scheme}
\index{Experiment Plane!Mixed deterministic / random experiment plane}
\index{Experiment Plane!Scaling scheme}
\index{Experiment Plane!Translation scheme}

The objective of this Use Case is to  create a deterministic experiment plane which levels are defined from the probabilistic distribution of the input random vector.\\


Details on experiment planes  may be found in the Reference Guide (\href{OpenTURNS_ReferenceGuide.pdf}{see files Reference Guide - Step C -- Min-Max approach using Experiment Planes}).\\

Details on each object may be found in the User Manual  (\href{OpenTURNS_UserManual_TUI.pdf}{see User Manual - Experiment Planes}).\\


The example here is an axial experiment plane where levels are proportionnal to the standard deviation of each component of the random input vector, and centered on the mean vector of the random input vector. Be carefull to remain within the range of the distribution if the latter is bounded!\\
There are three levels  : +/-1, +/-2, +/-3 around a center fixed equal to the center point ($\vect{0}$).\\
The dilatation vector is composed of the standard deviation of each component of the random input vector.\\

\requirements{
  \begin{description}
  \item[$\bullet$] the input vector : {\itshape input}
  \item[type:] RandomVector
  \end{description}
}
{
  \begin{description}
  \item[$\bullet$] an experiment plane : {\itshape myPlane}
  \item[type:] Axial
  \item[$\bullet$] a sample of {\itshape input} according to {\itshape myPlane} : {\itshape inputSample}
  \item[type:] NumericalSample
  \end{description}
}

\textspace\\
Python script for this UseCase :

\begin{lstlisting}
  # In order to use the 'sqrt' function
  from math import *

  # Dimension of the use case : 4
  dim = 4

  # Give the levels of the experiment plane
  # here,  3 levels : +/-1, +/-2, +/-3
  levelsNumber = 3
  levels = NumericalPoint( (1., 2. 3.) )
  levels.setName( "Levels" )

  # Create the axial plane centered on the vector (0)
  # and with the levels 'levels'
  myPlane = Axial(dim, levels)

  # Generate the points according to the structure
  # of the experiment plane (in a reduced centered space)
  inputSample = myPlane.generate()

  # Scale the structure of the experiment plane
  # proportionnally to the standard deviation of each component
  # of 'input' in case of a RandomVector
  # Scaling vector for each dimension of the levels of the structure
  # to take into account the dimension of each component
  scaling = NumericalPoint(dim, 0)
  scaling[0] = sqrt(input.getCovariance()[0,0])
  scaling[1] = sqrt(input.getCovariance()[1,1])
  scaling[2] = sqrt(input.getCovariance()[2,2])
  scaling[3] = sqrt(input.getCovariance()[3,3])
  inputSample.scale(scaling)

  # Translate the nonReducedSample onto the center of the experiment plane
  # Translation vector for each dimension
  center = input.getMean()
  inputSample.translate(center)
\end{lstlisting}



