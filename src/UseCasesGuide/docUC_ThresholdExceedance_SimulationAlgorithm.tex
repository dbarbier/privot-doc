% Copyright (c)  2005-2010 EDF-EADS-PHIMECA.
% Permission is granted to copy, distribute and/or modify this document
% under the terms of the GNU Free Documentation License, Version 1.2
% or any later version published by the Free Software Foundation;
% with no Invariant Sections, no Front-Cover Texts, and no Back-Cover
% Texts.  A copy of the license is included in the section entitled "GNU
% Free Documentation License".
\renewcommand{\filename}{docUC_ThresholdExceedance_SimulationAlgorithm.tex}
\renewcommand{\filetitle}{UC : Creation of a Monte Carlo / LHS / Quasi Monte Carlo / Importance Sampling simulation algorithm}

% \HeaderNNIILevel
% \HeaderIILevel
\HeaderIIILevel


\label{simuAlgo}




\index{Threshold Probability!Monte Carlo}
\index{Threshold Probability!Quasi Monte Carlo}
\index{Threshold Probability!LHS}
\index{Threshold Probability!Importance sampling}


The objective of this Use Case is to create a simulation algorithm in order to evaluate in fine the probability of the specified event according to the specified method : Monte Carlo sampling, LHS sampling, Importance sampling or the Quasi Monte Carlo method.\\

The Importance sampling method requires the specification of the importance distribution according to which the numerical sample will be generated.\\

For the LHS method, the copula of the multi-variate distribution must be independent.\\

As the quasi-Monte Carlo method is based on deterministic sequences, we do not need to initialize the state of the random generator.\\



Details on simulation algorithms may be found in the Reference Guide (\href{OpenTURNS_ReferenceGuide.pdf}{see files Reference Guide - Step C -- Estimating the probability of an event using Sampling} and files around).\\

Details on each object may be found in the User Manual  (\href{OpenTURNS_UserManual_TUI.pdf}{see User Manual - Threshold exceedance probability evaluation with simulation}).\\


\requirements{
  \begin{description}
  \item[$\bullet$] the event we want to evaluate the pobability : {\itshape myEvent}
  \item[type:] Event
  \end{description}
}
{
  \begin{description}
  \item[$\bullet$] the Monte Carlo simulation algorithm : {\itshape myMCAlgo}
  \item[type:] Simulation
  \item[$\bullet$] the LHS simulation algorithm : {\itshape myLHSAlgo}
  \item[type:] Simulation
  \item[$\bullet$] the Importance Sampling simulation algorithm : {\itshape myISAlgo}
  \item[type:] Simulation
  \end{description}
}

\textspace\\
Python  script for this UseCase :

\begin{lstlisting}
  # Create a Monte Carlo algorithm
  myMCAlgo = MonteCarlo(myEvent)

  # Create a LHS algorithm
  # Care : the copula of the multi-variate distribution must be independent
  myLHSAlgo = LHS(myEvent)

  # Create a Randomized LHS algorithm
  # Care : the copula of the multi-variate distribution must be independent
  myRandomizedLHSAlgo = RandomizedLHS(myEvent)

  # Create a Importance Sampling  algorithm
  # from the distribution myImportanceDistribution
  myISAlgo = ImportanceSampling(myEvent, Distribution(myImportanceDistribution))

  # Create a Quasi Monte Carlo algorithm
  # Care : the copula of the multi-variate distribution must be independent
  myQMCAlgo = QuasiMonteCarlo(myEvent)

  # Create a Randomized Quasi Monte Carlo algorithm
  # Care : the copula of the multi-variate distribution must be independent
  myRandomizedQMCAlgo = RandomizedQuasiMonteCarlo(myEvent)

\end{lstlisting}
