% Copyright (c)  2005-2010 EDF-EADS-PHIMECA.
% Permission is granted to copy, distribute and/or modify this document
% under the terms of the GNU Free Documentation License, Version 1.2
% or any later version published by the Free Software Foundation;
% with no Invariant Sections, no Front-Cover Texts, and no Back-Cover
% Texts.  A copy of the license is included in the section entitled "GNU
% Free Documentation License".
\renewcommand{\filename}{docUC_Intro_PythonStandards}
\renewcommand{\filetitle}{Link with other python standards}

\HeaderNNIILevel
% \HeaderIILevel
% \HeaderIIILevel

Many Open TURNS object act like python ones, so they can be safely included in algorithms or passed as function parameters like any other python object.\\
The table Tab.(\ref{conv}) gives all the conversion possibilities between Open TURNS objects and python or scipy ones.

\begin{table}
  \begin{center}
    \begin{tabular}[Hhtbp]{|l|c|c|c|}
      \hline
      \backslashbox{~~~~~~~~from~~~}{~~~to~~~~~~~~~}   & list     & tuple    & array \\
      \hline
      NumericalPoint            & $\times$ & $\times$ & $\times$ \\
      \hline
      NumericalSample           & $\times$ & $\times$ & $\times$ \\
      \hline
      Matrix                    &   --     &   --     & $\times$ \\
      \hline
      SquareMatrix              &   --     &   --     & $\times$ \\
      \hline
      SymmetricMatrix           &   --     &   --     & $\times$ \\
      \hline
      CovarianceMatrix          &   --     &   --     & $\times$ \\
      \hline
      CorrelationMatrix         &   --     &   --     & $\times$ \\
      \hline
      Tensor                    &   --     &   --     & $\times$ \\
      \hline
      SymmetricTensor           &   --     &   --     & $\times$ \\
      \hline
      ComplexMatrix             &   --     &   --     & $\times$ \\
      \hline
      SquareComplexMatrix       &   --     &   --     & $\times$ \\
      \hline
      HermitianMatrix           &   --     &   --     & $\times$ \\
      \hline
      TriangularComplexMatrix   &   --     &   --     & $\times$ \\
      \hline
      NumericalScalarCollection & $\times$ & $\times$ & $\times$ \\
      \hline
      Indices                   & $\times$ & $\times$ & $\times$\footnote{the type is not preserved} \\
      \hline
      BoolCollection            & $\times$ & $\times$ & $\times$\footnote{the type is not preserved} \\
      \hline
      UnsignedLongCollection    & $\times$ & $\times$ & $\times$\footnote{the type is not preserved} \\
      \hline
      NumericalPointCollection  & $\times$ & $\times$ & $\times$\footnote{if all the NumericalPoint instances share the same dimension} \\
      \hline
    \end{tabular}
    \begin{tabular}[Hhtbp]{|l|c|c|c|}
      \hline
      \backslashbox{~~~~~~~~to~~~}{~~~from~~~~~~~~~}   & list     & tuple    & array \\
      \hline
      NumericalPoint            & $\times$ & $\times$ & $\times$ \\
      \hline
      NumericalSample           & $\times$ & $\times$ & $\times$ \\
      \hline
      Matrix                    &   --     &    --    & $\times$ \\
      \hline
      SquareMatrix              &   --     &    --    & $\times$ \\
      \hline
      SymmetricMatrix           &   --     &    --    & $\times$ \\
      \hline
      CovarianceMatrix          &   --     &    --    & $\times$ \\
      \hline
      CorrelationMatrix         &   --     &    --    & $\times$ \\
      \hline
      Tensor                    &   --     &    --    & $\times$ \\
      \hline
      SymmetricTensor           &   --     &    --    & $\times$ \\
      \hline
      ComplexMatrix             &   --     &    --    & $\times$ \\
      \hline
      SquareComplexMatrix       &   --     &    --    & $\times$ \\
      \hline
      HermitianMatrix           &   --     &    --    & $\times$ \\
      \hline
      TriangularComplexMatrix   &   --     &    --    & $\times$ \\
      \hline
      NumericalScalarCollection & $\times$ & $\times$ & $\times$ \\
      \hline
      Indices                   & $\times$ & $\times$ &   --     \\
      \hline
      BoolCollection            & $\times$ & $\times$ &   --     \\
      \hline
      UnsignedLongCollection    & $\times$ & $\times$ &   --     \\
      \hline
      NumericalPointCollection  & $\times$ & $\times$ & $\times$ \\
      \hline
    \end{tabular}
    \caption{\label{conv} Conversions of Open TURNS objects into python/scipy ones. A $\times$ means that the conversion works, a -- that it does not work and an empty cell means that no conversion is expected.}
  \end{center}
\end{table}

For example, the NumericalPoint acts as a sequence object (aka, a list or a tuple).

It can be defined a NumericalPoint from a python list, as follows:
\begin{center}
  \begin{lstlisting}
    point = NumericalPoint( [1.1, 2.2, 3.3, 4.4] )
  \end{lstlisting}
\end{center}

or from a python tuple, as follows:
\begin{center}
  \begin{lstlisting}
    point2 = NumericalPoint( (1.1, 2.2, 3.3, 4.4) )
  \end{lstlisting}
\end{center}

It can be printed:
\begin{center}
  \begin{lstlisting}
    print point
    print repr( point )
    print str( point )
  \end{lstlisting}
\end{center}

or iterated:
\begin{center}
  \begin{lstlisting}
    l = len( point )
    for x in point:
        print x
  \end{lstlisting}
\end{center}

The NumericalSample acts the same way. It can be define from a python list, as follows:
\begin{center}
  \begin{lstlisting}
    point = NumericalSample( [[1., 2.], [3., 4.], [5., 6.]])
  \end{lstlisting}
\end{center}
which represents the sample $(1,2), (3,4), 5,6)$.


In fact any Collection object is python-aware.

Moreover, as listed in the previous table, some Open TURNS types can be converted from/to the numpy array type.

Here is how to build a NumericalSample from a numpy array:
\begin{center}
  \begin{lstlisting}
    import numpy as np
    a0 = np.array( ((1.,2.,3.),(4.,5.,6)) )
    n0 = NumericalSample( a0 )
  \end{lstlisting}
\end{center}

and how to convert it back to array:
\begin{center}
  \begin{lstlisting}
    a1 = np.array( n0 )
  \end{lstlisting}
\end{center}

As we can see the Open TURNS type to/from numpy array type is made easily by simpy calling the constructor of the desired type.\\
Another convenient feature is that the tuple/list/array types are allowed when calling every method that requires a NumericalPoint, NumericalSample, Indices or Description object
instead of explicitely creating an object of the required type and setting its content.\\

For example when creating a NumericalMathFunction from variables and formulas using sequences of strings instead of Description objects:
\begin{center}
  \begin{lstlisting}
    myFunction = NumericalMathFunction( np.array(('x0','x1')), ['y0','y1'], ('x0+x1','x0-x1') )
  \end{lstlisting}
\end{center}

Another quick example for extracting the marginals of a Distribution object using a tuple of integers instead of a Indices object:
\begin{center}
  \begin{lstlisting}
    distribution = ComposedDistribution(...)
    myMarginal = distribution.getMarginal( (1,3) )
  \end{lstlisting}
\end{center}

