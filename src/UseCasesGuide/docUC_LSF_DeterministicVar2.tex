% Copyright (c)  2005-2010 EDF-EADS-PHIMECA.
% Permission is granted to copy, distribute and/or modify this document
% under the terms of the GNU Free Documentation License, Version 1.2
% or any later version published by the Free Software Foundation;
% with no Invariant Sections, no Front-Cover Texts, and no Back-Cover
% Texts.  A copy of the license is included in the section entitled "GNU
% Free Documentation License".
\renewcommand{\filename}{docUC_LSF_DeterministicVar2.tex}
\renewcommand{\filetitle}{UC : Introducing some deterministic variables, optimizing memory and CPU time}

% \HeaderNNIILevel
% \HeaderIILevel
\HeaderIIILevel



\index{Limit State Function!Reducing the initial limit state function}
\index{Random Vector!Output random vector}

The objective of this Use Case is to restrict a model which has been previously declared, to some of its variables, through an optimized way.\\

Details on each object may be found in the User Manual  (\href{OpenTURNS_UserManual_TUI.pdf}{see User Manual - Base Objects / NumericalMathFunction}).\\

Let's have the same context than in the UC\ref{linearNumericalMathFunction}. The idea here is to avoid the introduction of the potentially huge matrix $\mat{A}$ and the gradient matrix and hessian tensor  of the functions $increase$ and $poutre$. For that last problem, it is sufficient to define the gradient matrix and hessian tensor to the final function $poutreReduced$ from a finite difference technique.\\

The function {\itshape increase} is defined as follows :
$$
\begin{array}{l|lcl}
  increase : & \mathbb{R}^{n_{prob}}  & \rightarrow & \mathbb{R}^n \\
  &  \vect{X}_{prob} =
  \begin{array}{|l}
    "inputProb1" \\
    \cdots       \\
    "inputProbNprob"
  \end{array}
  & \mapsto     & increase(\vect{X}_{prob}) =
  \begin{array}{|l}
    "inputProb1" \\
    \cdots       \\
    "inputProbNprob" \\
    valDet1 \\
    \cdots       \\
    valDetNdet
  \end{array}
\end{array}
$$
where all the $(valDet1, ..., valDetNdet)$ are the $n_{det}$ values of the determinist components of $\vect{X}$.\\

The same example is re-written in the folloing Use Case.

\requirements{
  \begin{description}
  \item[$\bullet$] the initial limit state function : {\itshape poutre}
  \item[type:] LinearNumericalMathFunction $(\mathbb{R}^4 \rightarrow \mathbb{R})$
  \end{description}
}
{
  \begin{description}
  \item[$\bullet$] the {\itshape increase} function
  \item[type:] NumericalMathFunction $(\mathbb{R}^2 \rightarrow \mathbb{R}^4)$
  \item[$\bullet$]  the new limit state function : {\itshape poutreReduced = poutre $\circ$ increase}
  \item[type:] NumericalMathFunction $(\mathbb{R}^2 \rightarrow \mathbb{R})$
  \end{description}
}

\textspace\\
Python script for this UseCase :

\begin{lstlisting}
  # Dimension of the random input vector
  stochasticDimension = 2

  # Dimension of the deterministic input vector
  deterministicDimension = 2

  # Dimension of the input vector of the limit state function 'poutre'
  inputDim = poutre.getInputNumericalPointDimension()

  # Fixe deterministic values for the two last variables
  # of the input vecteor (E,F,L,I)
  # L
  X2 = 10.0
  # I
  X3 = 5.0

  # Create the analyticalfunction 'increase'
  increase = NumericalMathFunction( ("E", "F"), ("E", "F", "L", "I"), ("E", "F", str(X2), str(X3)))

  # Create the new limit state function :
  # 'poutreReduced = poutre o increase'
  poutreReduced = NumericalMathFunction(poutre, increase)

  # Give directly to the 'poutreReduced' function a gradient evaluation method
  # thanks to the finite difference technique
  # For example, gradient technique : non centered finite difference method
  myGradient = NonCenteredFiniteDifferenceGradient(NumericalPoint(2, 1.0e-7), poutreReduced.getEvaluationImplementation())
  print "myGradient = ", myGradient

  # Substitute the gradient
  poutreReduced.setGradientImplementation(myGradient)

  # Give directly to the 'poutreReduced' function a hessian evaluation method
  # thanks to the finite difference technique
  # type : non centered finite difference method
  myHessian = CenteredFiniteDifferenceHessian(NumericalPoint(2, 1.0e-7), poutreReduced.getEvaluationImplementation())
  print "myHessian = ", myHessian

  # Substitute the hessian
  poutreReduced.setHessianImplementation(myHessian)
\end{lstlisting}



