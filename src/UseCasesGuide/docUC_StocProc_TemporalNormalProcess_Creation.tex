% Copyright (c)  2005-2010 EDF-EADS-PHIMECA.
% Permission is granted to copy, distribute and/or modify this document
% under the terms of the GNU Free Documentation License, Version 1.2
% or any later version published by the Free Software Foundation;
% with no Invariant Sections, no Front-Cover Texts, and no Back-Cover
% Texts.  A copy of the license is included in the section entitled "GNU
% Free Documentation License".
\renewcommand{\filename}{docUC_StocProc_TemporalNormalProcess_Creation.tex}
\renewcommand{\filetitle}{UC : Creation of a  stationary  normal process from its covariance function}

% \HeaderNNIILevel
% \HeaderIILevel
\HeaderIIILevel

\label{TemporalNormalProcessCreation}

\index{Stochastic Process!Normal Process}

Let $\vect{X}(\omega,t)$ be a stationary normal process with zero mean defined  by  its covariance function  $\tau \mapsto \mat{C}^{stat}(\tau)$.\\

The covariance function $\mat{C}^{stat}$ may be defined from a particular second order model. Open TURNS only proposes the \emph{multivariate exponential model}. Other models will be implemented later. \\

{\bf The multivariate exponential model} : We derive a multivariate extension of the exponential temporal covariance function initially defined for scalar stochastic processes by:
\begin{equation}
\label{scalarExponential}
\forall s\in\mathbb{R},\quad C^{stat}(\tau)=\mathbb{E}[X(s)X(s+\tau)]=a^2e^{-\lambda|\tau|}
\end{equation}
where $\lambda>0$ and $a>0$.\\

A first straightforward extension is to consider the multivariate process $\vect{X}$ with independent components $X_1$,\dots,$X_d$, each of which being a scalar process with its own exponential covariance function :
\begin{equation}
\label{simpleMultivariateExponential1}
\forall s\in\mathbb{R},\quad \mat{C}^{stat}(\tau)=\mathbb{E}[\vect{X}(s)\vect{X}^t(s+\tau)]=\mat{\Delta}(\tau)
\end{equation}
where
\begin{equation}
\label{simpleMultivariateExponential2}
\mat{\Delta}(\tau)=
\begin{pmatrix}
  a_1^2e^{-\lambda_1|\tau|} & & & \\
                       &\ddots & & \text{\Huge{0}}\\
  \text{\Huge{0}}      & & & a_d^2e^{-\lambda_d|\tau|}
\end{pmatrix}
\end{equation}\\

A full multivariate extension should allow the User to also define a \emph{spatial} correlation $\mat{R}$, i.e. to build a covariance function $\mat{C}^{stat}(\tau)$ such that :
\begin{equation}
\label{fullMultivariateExponential1}
\forall t\in\mathbb{R},\quad\mat{\text{Cor}}(\vect{X}(t))=\mat{R}
\end{equation}
and such that the components $X_1$,\dots,$X_d$ of $\vect{X}$ are still scalar processes with covariance function (\ref{scalarExponential}). One way to do that is to define the full multivariate exponential model by :
\begin{equation}
\label{fullMultivariateExponential2}
\forall \tau\in\mathbb{R},\quad \mat{C}^{stat}(\tau)=\sqrt{\mat{\Delta}(\tau)}\,\mat{R}\sqrt{\mat{\Delta}(\tau)}
\end{equation}
where
\begin{equation}
\label{fullMultivariateExponential3}
\sqrt{\mat{\Delta}(\tau)}=
\begin{pmatrix}
  a_1e^{-\lambda_1|\tau|/2} & & & \\
                       &\ddots & & \text{\Huge{0}}\\
  \text{\Huge{0}}      & & & a_de^{-\lambda_d|\tau|/2}
\end{pmatrix}
\end{equation}
 This model is such that:
\begin{align}
\left(\mat{C}^{stat}(\tau)\right)_{i,j}&=a_ie^{-\lambda_i|\tau|/2}R_{i,j}a_je^{-\lambda_j|\tau|/2}& i\neq j\\
\left(\mat{C}^{stat}(\tau)\right)_{i,i}&=a_ie^{-\lambda_i|\tau|/2}R_{i,i}a_ie^{-\lambda_j|\tau|/2}=a_i^2e^{-\lambda_i|\tau|}& i=j
\end{align}

It is possible to define the exponential model from the spatial covariance matrix $\mat{C}^s$ rather than the correlation matrix $\mat{R}$ : 
\begin{equation}
  \forall t\in\mathbb{R},\quad\mat{\mathbb{E}}[\vect{X}(t)\vect{X}^t(t)]=\mat{C}^s
\end{equation}
since we have the relation :
\begin{equation}\label{RClink}
\mat{C}^s=\mat{A}\,\mat{R}\mat{A}
\end{equation}
where
\begin{equation}
\mat{A}=
\begin{pmatrix}
  a_1 & & & \\
                       &\ddots & & \text{\Huge{0}}\\
  \text{\Huge{0}}      & & & a_d
\end{pmatrix}
\end{equation}
 
Open TURNS enables to implement the multivariate exponential model thanks to the object {\itshape ExponentialModel} that can be built from different input data : 
\begin{itemize}
  \item the model (\ref{simpleMultivariateExponential1})-(\ref{simpleMultivariateExponential2}) is built from the data : 
    \begin{itemize}
       \item scale vector $\vect{\lambda}=(\lambda_1,\dots,\lambda_d)$,
       \item and the amplitude vector $\vect{a}=(a_1,\dots,a_d)$.
    \end{itemize}
  \item the model  (\ref{fullMultivariateExponential1})-(\ref{fullMultivariateExponential3}) is  built from the data : 
   \begin{itemize}
     \item the scale vector $\vect{\lambda}=(\lambda_1,\dots,\lambda_d)$, 
     \item the amplitude vector $\vect{a}=(a_1,\dots,a_d)$,
     \item and the spatial correlation matrix $\mat{R}$. 
   \end{itemize}
  \item the model  (\ref{fullMultivariateExponential1})-(\ref{fullMultivariateExponential3}) is  built from the data : 
    \begin{itemize}
      \item the scale vector $\vect{\lambda}=(\lambda_1,\dots,\lambda_d)$,
      \item the covariance matrix $\mat{C}^s$. Then  $\mat{C}^s$ is mapped into the associated correlation matrix $\mat{R}$ through relation (\ref{RClink}) and  the previous constructor is used.\\
    \end{itemize}
\end{itemize}



Details on each object may be found in the User Manual  (\href{OpenTURNS_UserManual_TUI.pdf}{see User Manual - StochasticProcess}).\\



\requirements{
  \begin{description}
  \item[$\bullet$]  $\vect{a}$, $\vect{\lambda}$   : {\itshape amplitude, scale}
  \item[type:]  NumericalPoint
  \end{description}

  \begin{description}
  \item[$\bullet$]  $\mat{R}$,  : {\itshape spatialCorrelation}
  \item[type:]  CorrelationMatrix
  \end{description}

  \begin{description}
  \item[$\bullet$]  $\mat{C}^s$,  : {\itshape spatialCovariance}
  \item[type:]  CovarianceMatrix
  \end{description}


  \begin{description}
  \item[$\bullet$] a time grid : {\itshape myTimeGrid}
  \item[type:]  RegularGrid
  \end{description}

}
{
  \begin{description}
  \item[$\bullet$] a covariance model : {\itshape myCovarianceModel }
  \item[type:] StationaryCovarianceModel
  \end{description}

  \begin{description}
  \item[$\bullet$] a second order model : {\itshape mySecondOrderModel }
  \item[type:] SecondOrderModel
  \end{description}

  \begin{description}
  \item[$\bullet$] the normal process : {\itshape myNormalProcess}
  \item[type:]  TemporalNormalProcess 
  \end{description}


}

\textspace\\
Python script for this UseCase :

\begin{lstlisting}
  # Create the covariance model
  # for example : the Exponential model

  # from the amplitude, scale and spatialCovariance
  myCovarianceModel = ExponentialModel(amplitude, scale, spatialCorrelation)
  # or from the scale and spatialCovariance
  myCovarianceModel = ExponentialModel(scale, spatialCovariance)

  # Create the normal process witht the temporal view
  myNormalProcess = TemporalNormalProcess(myCovarianceModel, myTimeGrid)
\end{lstlisting}



