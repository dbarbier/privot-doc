% Copyright (c)  2005-2010 EDF-EADS-PHIMECA.
% Permission is granted to copy, distribute and/or modify this document
% under the terms of the GNU Free Documentation License, Version 1.2
% or any later version published by the Free Software Foundation;
% with no Invariant Sections, no Front-Cover Texts, and no Back-Cover
% Texts.  A copy of the license is included in the section entitled "GNU
% Free Documentation License".
\renewcommand{\filename}{docUC_RespSurface_LeastSquaresApprox.tex}
\renewcommand{\filetitle}{UC : Linear Least Squares approximation from a sample of the input vector and the real function}

% \HeaderNNIILevel
% \HeaderIILevel
\HeaderIIILevel



\label{leastSquareApprox}


\index{Response Surface!Linear least squares approximation}
\index{Random Generator}


This Use Case  details the second method to construct a response surface : by least squares method from a sample of the input vector and the real function. The output sample is obtained  by evaluating the function on the input sample.\\



Details on response surface approximations may be found in the Reference Guide (\href{OpenTURNS_ReferenceGuide.pdf}{see files Reference Guide - Step Res. Surf. -- Polynomial Response Surfaces : principles and -- Least Square Method}).\\

Details on each object may be found in the User Manual  (\href{OpenTURNS_UserManual_TUI.pdf}{see User Manual - Parametric response surface / least squares approximation}).\\


\requirements{
  \begin{description}
  \item[$\bullet$] the limit state function : {\itshape myFunc}
  \item[type:]  NumericalMathFunction
  \item[$\bullet$] a sample of the input vector : {\itshape inputSample}
  \item[type:]  NumericalSample
  \end{description}
}
{
  \begin{description}
  \item[$\bullet$] linear approximation by least squares method {\itshape responseSurface}
  \item[type:] NumericalMathFunction
  \item[$\bullet$]  the coefficients of the linear approximation $myFunc(\vect{X}) = \vect{\vect{A}}\vect{X} + \vect{B}$
  \item[type:] Matrix ($\vect{\vect{A}}$) , NumericalPoint ($\vect{B}$)
  \end{description}
}

\textspace\\
Python  script for this Use Case :

\begin{lstlisting}
  # Create the LinearLeastSquares algorithm
  myLeastSquares = LinearLeastSquares(inputSample, myFunc)

  # Perform the algorithm
  myLeastSquares.run()
  print "myLeastSquares=", myLeastSquares

  # Stream out the results :
  # get the matrix A :
  linear = myLeastSquares.getLinear()
  print "A = ", linear

  # Get the constant term B :
  constant = myLeastSquares.getConstant()
  print "B = ", constant

  # Get the linear response surface
  responseSurface = myLeastSquares.getResponseSurface()
  print "responseSurface=", responseSurface

  # Compare the two models at a particular point 'inPoint'
  print "(myFunc", inPoint, ")=", myFunc(inPoint)
  print "responseSurface(", inPoint, ")=", responseSurface(inPoint)
\end{lstlisting}
