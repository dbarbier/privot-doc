% Copyright (c)  2005-2010 EDF-EADS-PHIMECA.
% Permission is granted to copy, distribute and/or modify this document
% under the terms of the GNU Free Documentation License, Version 1.2
% or any later version published by the Free Software Foundation;
% with no Invariant Sections, no Front-Cover Texts, and no Back-Cover
% Texts.  A copy of the license is included in the section entitled "GNU
% Free Documentation License".
\renewcommand{\filename}{docUC_StocProc_Process.tex}
\renewcommand{\filetitle}{UC : Manipulation of a process}

% \HeaderNNIILevel
 \HeaderIILevel
%\HeaderIIILevel

\label{UCprocess}


\index{Stochastic Process!Process Manipulation}


We recall that  $\underline{X}(\omega,t)$ is a stochastic process, where $\omega \in \Omega$ is an event, $t \in \mathbb{R}$ is the time and $\underline{X}(\omega,t) \in \mathbb{R}^d$ are the observed values of the process at each time $t$.\\

The objective here is to manipulate a stochastic process, once created according to a particular model : 
\begin {itemize}
  \item to extract its marginal process $j$ for $ \in [0,d-1]$ : $\underline{X}_j(\omega,t)$ thanks to the method \emph{getMarginalProcess};
  \item to get its dimension $d$  thanks to the method \emph{getDimension};
  \item to get one or several realization(s) of the process, thanks to the methods \emph{getRealization}, \emph{getSample};
  \item to get its time grid, thanks to the method \emph{getTimeGrid};
  \item to check wether the process is normal, thanks to the method \emph{isNormal};
  \item to check wether the process is stationnary, thanks to the method \emph{isStationnary};
\end{itemize}

Details on each object may be found in the User Manual  (\href{OpenTURNS_UserManual_TUI.pdf}{see User Manual - Stochastic Process}).\\

\requirements{
  \begin{description}
  \item[$\bullet$] a stochastic process {\itshape myProcess}
  \item[type:]  Process
  \end{description}
}
{
  \begin{description}
  \item[$\bullet$] a time grid : {\itshape myTimeGrid}
  \item[type:]  RegularGrid
  \end{description}

  \begin{description}
  \item[$\bullet$] a time series : {\itshape myRealization}
  \item[type:]  TimeSeries
  \end{description}

  \begin{description}
  \item[$\bullet$] a sample of time series : {\itshape myRealisationSample}
  \item[type:] ProcessSample
  \end{description}

  \begin{description}
  \item[$\bullet$] a stochastic process {\itshape myMarginalProcess}
  \item[type:]  Process
  \end{description}
}

\textspace\\
Python script for this UseCase :

\begin{lstlisting}
  # Get the dimension d of the process  
  dimension = myProcess.getDimension()

  # Get the time grid of the process
  myTimeGrid = myProcess.getTimeGrid()

  # Get a realisation of the process
  myRealisation = myProcess.getRealization()

  # Get several realisations of the process
  number = 10
  myRealizationSample = myProcess.getSample(number)

  # Get the marginal of the process at index i
  # Care! Numerotation begins at i=0
  myMarginalProcess = myProcess.getMarginalProcess(i)

  # Get the marginal of the process at index in indices
  myMarginalProcess = myProcess.getMarginalProcess(indices)

  # Check  wether the process is normal
  print myProcess.isNormal()

  # Check  wether the process is stationnary
  print myProcess.isStationnary()
\end{lstlisting}
