% Copyright (c)  2005-2010 EDF-EADS-PHIMECA.
% Permission is granted to copy, distribute and/or modify this document
% under the terms of the GNU Free Documentation License, Version 1.2
% or any later version published by the Free Software Foundation;
% with no Invariant Sections, no Front-Cover Texts, and no Back-Cover
% Texts.  A copy of the license is included in the section entitled "GNU
% Free Documentation License".
\renewcommand{\filename}{docUC_ThresholdExceedance_StandardEventManipulation.tex}
\renewcommand{\filetitle}{UC : Manipulation of a StandardEvent}

% \HeaderNNIILevel
% \HeaderIILevel
\HeaderIIILevel

\index{Event!Standard space}

This section gives elements to manipulate an {\itshape StandardEvent} in Open TURNS .\\



Details on isoproabilitic transformations  may be found in the Reference Guide (\href{OpenTURNS_ReferenceGuide.pdf}{see files Reference Guide - Step C -- Isoprobabilistic transformation preliminary to FORM-SORM methods}).\\


Details on each object may be found in the User Manual  (\href{OpenTURNS_UserManual_TUI.pdf}{see User Manual - Threshold exceedance probability evaluation with reliability algorithm / Reliability Algorithms}).\\

The example here is an output variable {\itshape output} defined from the limit state function {\itshape f} and the input random  vector {\itshape input}. The event considered is :
$$
myEvent = \{ output=f(input) \leq -1.5 \}.
$$

\requirements{
  \begin{description}
  \item[$\bullet$] an event expressed in the physical space : {\itshape myEvent}
  \item[type:] Event
  \item[$\bullet$] the associated event in the standard space : {\itshape myStandardEvent}
  \item[type:] StandardEvent
  \end{description}
}
{
  \begin{description}
  \item[$\bullet$] none
  \end{description}
}

\textspace\\
Python  script for this UseCase :

\begin{lstlisting}
  # myEvent : E = (output=f(input), operator : <, threshold : -1,5)

  # Realization of 'input' as antecedent of 'output'
  print "myStandardEvent (as a RandomVector) antecedent realization =" , RandomVector(myStandardEvent).getImplementation().getAntecedent().getRealization()

  # Realization of 'myEvent' as a Bernoulli
  print "myStandardEvent realization=" , myStandardEvent.getRealization()

  # Sample of 10 realizations of 'myEvent  as a Bernoulli
  print "myStandardEvent sample=" , myStandardEvent.getNumericalSample(10)

  # Realization of 'input' as antecedent of 'myEvent'
  print "myStandardEvent antecedent realization=" , myStandardEvent.getImplementation().getAntecedent().getRealization()
\end{lstlisting}


