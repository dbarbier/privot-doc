% Copyright (c)  2005-2010 EDF-EADS-PHIMECA.
% Permission is granted to copy, distribute and/or modify this document
% under the terms of the GNU Free Documentation License, Version 1.2
% or any later version published by the Free Software Foundation;
% with no Invariant Sections, no Front-Cover Texts, and no Back-Cover
% Texts.  A copy of the license is included in the section entitled "GNU
% Free Documentation License".
\renewcommand{\filename}{docUC_StocProc_SpectralNormalProcess_Creation.tex}
\renewcommand{\filetitle}{UC : Creation of a  stationary  normal process from its spectral density function}

% \HeaderNNIILevel
% \HeaderIILevel
\HeaderIIILevel

\label{SpectralNormalProcessCreation}
\index{Stochastic Process!Normal Process}

Let $\vect{X}(\omega,t)$ be a stationary normal process with zero mean defined  by  its unilateral spectral density function $f \in \mathbb{R}^+ \mapsto \mat{G}(f)$ defined by in (\ref{univG}). \\


In that version, Open TURNS only implements the Cauchy spectral model, thanks to the object {\itshape CauchyModel}, which is the model associated to the exponential covariance model, defined in (\ref{TemporalNormalProcessCreation}). The Cauchy spectral model has the following spectral denity function $f \in \mathbb{R} \mapsto \mat{S}(f)$ : 
\begin{equation}\label{cauchyModel}
  S_{ij}(f) = \displaystyle \frac{4R_{ij}a_ia_j(\lambda_i+ \lambda_j)}{(\lambda_i+ \lambda_j)^2 + (4\pi f)^2}
\end{equation}
where $\mat{R}$, $\vect{a}$ and $\vect{\lambda}$ are the parameters of the Exponential covariance model. The relation (\ref{cauchyModel}) can be explicited with the spatuial covariance function  $\mat{C}^{s}(\tau)$ defined in (\ref{RClink}).\\
 

Furthermore, Open TURNS allows the User to define its own spectral model, with the object {\itshape UserDefinedSpectralModel} as illustrated in \ref{SpectralModelCreation}\\


The example is illustrated after the requirements and the python scipting. \\ \\
Details on each object may be found in the User Manual  (\href{OpenTURNS_UserManual_TUI.pdf}{see User Manual - StochasticProcess}).\\


\requirements{
  \begin{description}
  \item[$\bullet$]  $\vect{a}$, $\vect{\lambda}$   : {\itshape amplitude, scale}
  \item[type:]  NumericalPoint
  \end{description}

  \begin{description}
  \item[$\bullet$]  $\mat{R}$,  : {\itshape spatialCorrelation}
  \item[type:]  CorrelationMatrix
  \end{description}

  \begin{description}
  \item[$\bullet$]  $\mat{C}^s$,  : {\itshape spatialCovariance}
  \item[type:]  CovarianceMatrix
  \end{description}


  \begin{description}
  \item[$\bullet$] a time grid : {\itshape myTimeGrid}
  \item[type:]  RegularGrid
  \end{description}

}
{
  \begin{description}
  \item[$\bullet$] a spectral model : {\itshape mySpectralModel }
  \item[type:] SpectralModel
  \end{description}

  \begin{description}
  \item[$\bullet$] the normal process : {\itshape myNormalProcess}
  \item[type:]  SpectralNormalProcess 
  \end{description}

}

\textspace\\
Python script for this UseCase :

\begin{lstlisting}
  # Create the spectral model
  # for example : the Cauchy model

  # from the amplitude, scale and spatialCovariance
  mySpectralModel = CauchyModel(amplitude, scale, spatialCorrelation)
  # or from the scale and spatialCovariance
  mySpectralModel = CauchyModel(scale, spatialCovariance)

  # Create the normal process witht the spectral view
  myNormalProcess = SpectralNormalProcess(mySpectralModel, myTimeGrid)
\end{lstlisting}



