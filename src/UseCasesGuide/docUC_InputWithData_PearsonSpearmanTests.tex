% Copyright (c)  2005-2010 EDF-EADS-PHIMECA.
% Permission is granted to copy, distribute and/or modify this document
% under the terms of the GNU Free Documentation License, Version 1.2
% or any later version published by the Free Software Foundation;
% with no Invariant Sections, no Front-Cover Texts, and no Back-Cover
% Texts.  A copy of the license is included in the section entitled "GNU
% Free Documentation License".
\renewcommand{\filename}{docUC_InputWithData_PearsonSpearmanTests.tex}
\renewcommand{\filetitle}{UC : Particular manipulations of the Pearson and Spearman tests, when the first sample is of dimension superior to 1.}

% \HeaderNNIILevel
% \HeaderIILevel
\HeaderIIILevel


\index{Independence Test!ChiSquared test}
\index{Independence Test!Pearson test}
\index{Independence Test!Spearman test}

The objective of this Use Case is to decide whether two samples follow a monotonous or linear relation in the case where the first sample is of dimension $>1$.\\
The Pearson and Spearman tests are evaluated successively between some (or all) coordinates of the first sample and the second one, which must be of dimension 1.\\




Details on the Pearson and Spearman tests  may be found in the Reference Guide (\href{OpenTURNS_ReferenceGuide.pdf}{see files Reference Guide - Step B -- Pearson correlation test, Step B -- Spearman correlation test}).\\

Details on each object may be found in the User Manual  (\href{OpenTURNS_UserManual_TUI.pdf}{see User Manual - Statistics on sample /  Hypothesis Test}).\\




\requirements{
  \begin{description}
  \item[$\bullet$] one continuous scalar numerical sample of dimension n : {\itshape continuousSample1}
  \item[type:]  NumericalSample
  \item[$\bullet$] one continuous scalar numerical sample of dimension 1 : {\itshape continuousSample2}
  \item[type:]  NumericalSample
  \end{description}
}
{
  \begin{description}
  \item[$\bullet$] tests results : {\itshape resultPartialPearson, resultFullPearson, resultPartialSpearman, resultFullSpearman}
  \item[type:] TestResultCollection
  \end{description}
}

\textspace\\
Python script for this UseCase :

\begin{lstlisting}

  # Partial Pearson Test : test if two scalar samples which form a gaussian vector are independent (based on the evaluation of the linear correlation coefficient)
  # H0 : independent samples (linear correlation coefficient = 0)
  # Test = True <=> independent samples (linear correlation coefficient = 0)
  # p-value threshold : probability of the H0 reject zone : 1-0.90
  # p-value : probability (test variable decision > test variable decision evaluated on the samples)
  # Test = True <=> p-value > p-value threshold

  # selection of coordinates of continuousSample1 to be tested to continuousSample2
  # for example, coordinates 1, 2, 3, 4, 5, (suppose n>5)
  selection = range(5)

  # Perform the Partial Pearson Test
  resultPartialPearson = HypothesisTest.PartialPearson(continuousSample1, continuousSample2, selection, 0.90)

  # Print the global result of the Pearson Test
  print "Test global result : ", resultPartialPearson

  # Print result of the Pearson Test for each coordinate tested
  for i in range(5) :
  print "Test Succes for Coordinate =  ", selection[i], "? ", (resultPartialPearson[i].getBinaryQualityMeasure()==1)

  # Get the p-value of the Pearson Test
  print "p-value of the Pearson Test = ", resultPartialPearson[i].getPValue()

  # Get the p-value threshold of the  Test
  print "p-value threshold for Coordinate =  ", selection[i], " = ", resultPartialPearson[i].getThreshold()

  # Full Pearson Test : it performs the partial Pearson test on the whole coordinates of the first sample

  # Perform the Full Pearson Test
  resultFullPearson = HypothesisTest.FullPearson(continuousSample1, continuousSample2, 0.90)

  # Same manipulations than those effected on resultPartialPearson to get the results

  # Partial Spearman Test : test if two scalar samples have a monotonous relation
  # H0 : no monotonous relation between both samples
  # Test = True <=> no monotonous relation
  # p-value threshold : probability of the H0 reject zone : 1-0.90
  # p-value : probability (test variable decision > test variable decision evaluated on the samples)
  # Test = True <=> p-value > p-value threshold

  # selection of coordinates of continuousSample1 to be tested to continuousSample2
  # for example, coordinates 1, 2, 3, 4, 5, (suppose n>5)
  selection = range(5)

  # Perform the Partial Spearman Test
  resultPartialSpearman = HypothesisTest.PartialSpearman(continuousSample1, continuousSample2, selection, 0.90)

  # Print the global result of the Spearman Test
  print "Test global result : ", resultPartialSpearman

  # Print result of the Spearman Test for each coordinate tested
  for i in range(5) :
  print "Test Succes for Coordinate =  ", selection[i], "? ", (resultPartialSpearman[i].getBinaryQualityMeasure()==1)

  # Get the p-value of the Spearman Test
  print "p-value of the Spearman Test = ", resultPartialSpearman[i].getPValue()

  # Get the p-value threshold of the  Test
  print "p-value threshold for Coordinate =  ", selection[i], " = ", resultPartialSpearman[i].getThreshold()

  # Full Spearman Test : it performs the partial Pearson test on the whole coordinates of the first sample

  # Perform the Full Spearman Test
  resultFullSpearman = HypothesisTest.FullSpearman(continuousSample1, continuousSample2, 0.90)

  # Same manipulations than those effected on resultPartialSpearman to get the results
\end{lstlisting}


