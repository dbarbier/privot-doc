% Copyright (c)  2005-2010 EDF-EADS-PHIMECA.
% Permission is granted to copy, distribute and/or modify this document
% under the terms of the GNU Free Documentation License, Version 1.2
% or any later version published by the Free Software Foundation;
% with no Invariant Sections, no Front-Cover Texts, and no Back-Cover
% Texts.  A copy of the license is included in the section entitled "GNU
% Free Documentation License".
\renewcommand{\filename}{docUC_StocProc_NormalIntro.tex}
\renewcommand{\filetitle}{Intro}

% \HeaderNNIILevel
% \HeaderIILevel
%\HeaderIIILevel


\index{Stochastic Process!Normal Process}




Let $\vect{X}(\omega,t)$ be a multivariate random process  of dimension $d$.  We note, when the quantities exist :
\begin{itemize}
\item  $\vect{m}$ its \emph{mean function},  defined for all $t>0$, by $\vect{m}(t)=\mathbb{E}[\vect{X}_t] \in \mathbb{R}^d$ , 
\item $\mat{C}$ its \emph{temporal covariance function},  defined for all $(s,t)>0$, by $\mat{C}(s, t)=\mathbb{E}[(\vect{X}_s-\vect{m}(s))(\vect{X}_t-\vect{m}(t))^t] \in  \mathcal{M}_{d \times d}(\mathbb{R})$,
\item  $\mat{R}$ its \emph{temporal correlation function}, defined for all $(s,t)>0$, by $\mat{R}(s, t)  \in  \mathcal{M}(\mathbb{R}^d,\mathbb{R}^d)$ where for all $(i,j)$, $R_{ij}(s, t)=C_{ij}(s, t)/\sqrt{C_{ii}(s, t)C_{jj}(s,t)}$.
\end{itemize}

We recall here some useful defintiions.\\

{\bf Normal process} : A stochastic process is {\itshape normal}  if all its finite dimensional joint distributions are normal, which means that for all $k  \in  \mathbb{N}$ and $I_k \in \mathbb{N}^*$, with $card I_k = k$, there exist $\vect{m}_1,\dots,\vect{m}_k\in\mathbb{R}^d$ and $\mat{C}_{1,\dots,k}\in\mathcal{M}_{kd\times kd}(\mathbb{R})$ such that :
\begin{eqnarray}
   \mathbb{E}[\exp\left\{i\vect{X}_{I_k}^t \vect{U}_{k}  \right\}] = 
\exp\left\{i\vect{U}_{k}^t\vect{M}_{k}-\frac{1}{2}\vect{U}_{k}^t\mat{C}_{1,\dots,k}\vect{U}_{k}\right\}
\end{eqnarray}
where $\vect{X}_{I_k}^t = (\vect{X}_{t_1}^t, \hdots, \vect{X}_{t_k}^t)$ the concatenation in one vector of the $k$ vectors $\vect{X}_{t_i}$, $\vect{U}_{k}^t = (\vect{u}_{1}^t, \hdots, \vect{u}_{k}^t)$ and $\vect{M}_{k}^t = (\vect{m}_{1}^t, \hdots, \vect{m}_{k}^t)$  the concatenation in one vector of the $k$ vectors $\vect{u}_{i}$ or $\vect{m}_{i}$, and $\mat{C}_{1,\dots,k}$ is the symmetric matrix : 
\begin{equation}\label{covMatrix}
\mat{C}_{1,\dots,k} = \left(
\begin{array}{cccc}
  \mat{C}(t_1, t_1) & \mat{C}(t_1, t_2) & \hdots & \mat{C}(t_1, t_{k}) \\
   \hdots &\mat{C}(t_2, t_2)  & \hdots & \mat{C}(t_2, t_{k}) \\
  \hdots & \hdots & \hdots & \hdots \\
   \hdots & \hdots & \hdots & \mat{C}(t_{k}, t_{k}) 
\end{array}
\right)
\end{equation}

A normal process is entirely defined by its mean function $t \mapsto \vect{m}(t)$ and its temporal covariance function  $(s,t) \mapsto \mat{C}(s,t)$.\\


{\bf Weak stationarity (second order stationarity) } : A process $\vect{X}(\omega,t)$ is \emph{ weakly stationary} or \emph{stationary of second order} if its mean function is constant and its covariance function is invariant by time translation : 
\begin{eqnarray}\label{stat2order}
 \forall t>0,&   \vect{m}(t)   = \vect{m} \\
 \forall (s,t,h) >0,  &  \mat{C}(s,s+h)  = \mat{C}(t,t+h)
\end{eqnarray}
Abusing the notation, we use $\mat{C}^{stat}(\tau)$ for $\mat{C}(s, s+\tau)$ as this quantitiy does not depend on $s$.\\

 A process $\vect{X}(\omega,t)$ is \emph{ \bf stationary} if all its finite dimensional joint distributions  are the same. A normal process with second order stationarity is stationary.\\
 

{\bf Spectral density function} :  If $\vect{X}(\omega,t)$  is a stationary process, we  define the \emph{ bilateral spectral density function} $\mat{S}(f) \in \mathcal{M}(\mathbb{R}^d \times \mathbb{R}^n)$ as the Fourier transform of the covariance function $\mat{C}^{stat}$ :
\begin{equation} \label{specdensFunc}
  \forall f \in \mathbb{R}, \, \mat{S}(f) = \int_{\mathbb{R}}\exp\left\{  -2i\pi f \tau \right\} \mat{C}^{stat}(\tau)\, d\tau
\end{equation}
and the  \emph{ unilateral spectral density function} $\mat{G}(f)$ defined by : 
\begin{equation}\label{univG}
  \forall f \geq 0, \mat{G}(f) = 2\mat{S}(f)
\end{equation}

The  covariance function $\mat{C}^{stat}$ may be evaluated from the spectral density function $\mat{S}(f)$  if $\mat{S}$ is $L^1(\mathcal{M}_{dd}(\mathbb{C}))$ as follows :
\begin{equation} \label{cspectransform}
  \mat{C}^{stat}(\tau)  = \int_{\mathbb{R}}\exp\left\{  2i\pi f \tau \right\} \mat{S}(f)\, df
\end{equation}




 In Open TURNS, all the stationary normal processes are centered : 
 \begin{equation}
 \vect{m}(t)   = \vect{m} = \vect{0}
 \end{equation}
and  defined either by : 
\begin{itemize} 
  \item their covariance function  $\tau \mapsto \mat{C}^{stat}(\tau)$. Then the normal process is used trhough its temporal view, thanks to the object {\itshape TemporalNormalProcess} (see UC \ref{TemporalNormalProcessCreation}),
  \item or their unilateral spectral density function $f \in \mathbb{R}^+ \mapsto \mat{G}(f)$. Then the normal process is used trhough its spectral view, thanks to the object {\itshape SpectralNormalProcess} (see UC \ref{SpectralNormalProcessCreation}).
\end{itemize}

In order to manipulate the same normal process through both the temporal and spectral views, it is necessary to create a second order model that insures the coherence between the convariance function $\tau \mapsto \mat{C}^{stat}(\tau)$ and the specatrl density fucntion  $f \in \mathbb{R}^+ \mapsto \mat{G}(f)$ through the relation (\ref{specdensFunc}). In that purpose, the object {\itshape SecondorderModel} is built then used to create a {\itshape TemporalNormalProcess} and the associated {\itshape SpectralNormalProcess} (see UC (\ref{SecondOrderNormalProcessCreation})).
