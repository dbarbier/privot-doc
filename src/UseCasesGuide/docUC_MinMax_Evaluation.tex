% Copyright (c)  2005-2010 EDF-EADS-PHIMECA.
% Permission is granted to copy, distribute and/or modify this document
% under the terms of the GNU Free Documentation License, Version 1.2
% or any later version published by the Free Software Foundation;
% with no Invariant Sections, no Front-Cover Texts, and no Back-Cover
% Texts.  A copy of the license is included in the section entitled "GNU
% Free Documentation License".
\renewcommand{\filename}{docUC_MinMax_Evaluation.tex}
\renewcommand{\filetitle}{UC : Min/Max research  from an experiment plane and sensitivity analysis}

% \HeaderNNIILevel
% \HeaderIILevel
\HeaderIIILevel

\index{Min-Max!}
\index{Sensitivity Analysis!Min-Max}


The objective of this Use Case is to evaluate the min and max values of the output variable of interest from a numerical sample and to evaluate the gradient of the limit state function defining the output variable of interest at a particular point. \\

Details on experiment planes  may be found in the Reference Guide (\href{OpenTURNS_ReferenceGuide.pdf}{see files Reference Guide - Step C -- Min-Max approach using Experiment Planes}).\\

Details on each object may be found in the User Manual  (\href{OpenTURNS_UserManual_TUI.pdf}{see User Manual - Statistics on Sample / Numerical Sample}).\\

The numerical sample of the output variable of interest may be obtained as follows :
\begin{itemize}
\item create an experiment plane of the input random vector : deterministic scheme (see Use Case \ref{determExpPlane}) or random scheme (see Use Case \ref{randomExpPlane}), mixed deterministic/random scheme (see Use Case \ref{determRandomExpPlane}), or by re-using a previously elaborated experiment plane (see Use Case \ref{fixedExpPlane}),
\item evaluate the output variable of interest on each point of the experiment plane.
\end{itemize}

The example here is the limit state function {\itshape poutre} defined in Eq.(\ref{equatPoutre}) with the random input vector $(E,F,L,I)$.\\

\requirements{
  \begin{description}
  \item[$\bullet$] the numerical sample of the input random vector : {\itshape inputSample}
  \item[type:] NumericalSample
  \item[$\bullet$] the limit state function : {\itshape poutre}
  \item[type:] NumericalMathFunction
  \item[$\bullet$] the input vector where gradient is evaluated : {\itshape input0}
  \item[type:] NumericalPoint
  \end{description}
}
{
  \begin{description}
  \item[$\bullet$] the min and max of the output variable of interest
  \item[type:] NumericalPoint
  \item[$\bullet$] the deterministic gradient of the output variable of interest at $input_0$
  \item[type:] Matrix
  \end{description}
}

\textspace\\
Python script for this UseCase :

\begin{lstlisting}
  # Generate the  values of the output variable of interest
  # 'output = poutre(input)' corresponding to 'inputSample'
  outputSample = poutre(inputSample)
  print "outputSample = ", outputSample

  # Get the min and the max of the output variable, component by component
  min = outputSample.getMin()
  max = outputSample.getMax()
  print  "max =  ", max
  print  "min =  ", min

  # Get the gradient of 'poutre' with respect to 'input'
  # at a particular point 'input0'
  sensitivity = poutre.gradient(input0)
  print "sensitivity at point input0 = ", sensitivity
\end{lstlisting}


