% Copyright (c)  2005-2010 EDF-EADS-PHIMECA.
% Permission is granted to copy, distribute and/or modify this document
% under the terms of the GNU Free Documentation License, Version 1.2
% or any later version published by the Free Software Foundation;
% with no Invariant Sections, no Front-Cover Texts, and no Back-Cover
% Texts.  A copy of the license is included in the section entitled "GNU
% Free Documentation License".
\renewcommand{\filename}{docUC_StocProc_ARMA_Estimation_Whittle.tex}
\renewcommand{\filetitle}{UC : Estimation of an ARMA model using the Whittle likelihood function}

% \HeaderNNIILevel
% \HeaderIILevel
\HeaderIIILevel

\label{ARMAEstimationWhittle}


\index{Stochastic Process!ARMA}


The objective of the UseCase is to estimate an ARMA model from a time series using the Whittle estimator. \\
We make assumption that the white noise is gaussian and that the order of the model is known (AR and MA sizes).
The coefficients of the model and the variance of the a priori gaussian white noise have to be estimated. 
The Whittle method bases on the maximization of the likelihood function in frequency domain. 
Let us present quickly the estimator (which is implemented currently in OpenTURNS only for dimension $1$). \\ 

We consider a time series of size $n$ for which an ARMA representation is available and orders were identified.
Thus by using the notation of an ARMA process given in equation \ref{dim1}, the spectrum writes :

\begin{equation}{\label{arma_spectrum}}
f(\lambda, \theta, \sigma^2) = \frac{\sigma^2}{2 \pi} \frac{|1 + b_1 \exp(-\imath \lambda) + \ldots + b_q \exp(-\imath q \lambda)|^2}{|1 + a_1 \exp(-\imath \lambda) + \ldots + a_p \exp(-\imath p \lambda)|^2}
\end{equation}

with :
\begin{itemize}
\item $\theta$=$(a_1, a_2,...,a_p,b_1,...,b_q)$ are the $p+q$ parameters of the model;
\item $\sigma^2$ is the variance of the gaussian white noise;
\item $\lambda$ is the frequency value;
\end{itemize}

We notice $g$ the ratio function in the spectrum expression as :
\begin{equation*}
g(\lambda) = \frac{|1 + b_1 \exp(-\imath \lambda) + \ldots + b_q \exp(-\imath q \lambda)|^2}{|1 + a_1 \exp(-\imath \lambda) + \ldots + a_p \exp(-\imath p \lambda)|^2}
\end{equation*}

The Whittle likelihood associated to such a process is defined as following :

\begin{equation}{\label{arma_likelihood}}
\log Lw(\theta, \sigma^2) = - \sum_{j=1}^{m} log f(\lambda_j,  \theta, \sigma^2) - \frac{1}{2 \pi} \sum_{j=1}^{m} \frac{I(\lambda_j)}{f(\lambda_j,  \theta, \sigma^2)}
\end{equation}

with :
\begin{itemize}
\item $I$ is the periodogram. A tapered periodogram is prefered to this last one using for example the Welch method because of better statistical properties;
\item $\lambda_j$ is the $j-th$ Fourier frequency, $\lambda_j = \frac{2 \pi j}{n}$, $j=1 \ldots m$ with $m$ the largest integer contained in $\frac{n-1}{2}$
\end{itemize}

$p+q+1$ scalar coefficients have to be estimated by maximizing the likelihood function. We notice that in the \ref{arma_likelihood}, we could isolate the variance.
Indeed by derivating the likelihood with respect to $\sigma^2$, we get the optimal variance which maximizes the likelihood :
\begin{equation*}
\sigma^{2*} = \frac{1}{m} \sum_{j=1}^{m} \frac{I(\lambda_j)}{g(\lambda_j)}
\end{equation*}

We introduce the optimal variance expression in equation \ref{arma_likelihood} and we get a new and reduced expression of the likelihood function :
\begin{equation*}
\log Lw(\theta) = m (\log(2 \pi) - 1) - m \log\left( \frac{1}{m} \sum_{j=1}^{m} \frac{I(\lambda_j)}{g(\lambda_j)}\right) - \sum_{j=1}^{m} g(\lambda_j)
\end{equation*}
The $\theta$ maximizing this expression have to be estimated. 

Notice that for a sample of realizations of an ARMA process, the estimate is similar except that the periodogram is computed differently. \\ \\

In OpenTURNS, the $WhittleFactory$ enables to estimate the coefficients of an ARMA model using a time series (or a collection).
OpenTURNS proceeds as follows :

\begin{itemize}
\item The value $p$ and $q$ are given to the \emph{WhittleFactory}. Also, by default, the Welch method is used to compute the periodogram (\emph{WelchFactory} with default parameters);
\item The computation of the parameters of the model is done by maximizing the reduced equation of the likelihood function.
This is done thanks to the method \emph{build} of the object \emph{WhittleFactory}
\item The produced object is of type \emph{ARMA}. The white noise is gaussian with $\sigma^{2*}$.
\end{itemize}

Details on each object may be found in the User Manual (\href{OpenTURNS_UserManual_TUI.pdf}{see User Manual - Stochastic Process}).\\


\requirements{
  \begin{description}
  \item[$\bullet$] order of the model : {\itshape p, q}
  \item[type:]  integer
  \end{description}


  \begin{description}
  \item[$\bullet$] a time series or collection of time series : {\itshape myTimeSeries}, {\itshape mySample}
  \item[type:]  TimeSeries or ProcessSample
  \end{description}

  \begin{description}
  \item[$\bullet$] the spectral factory : {\itshape mySpectralFactory}
  \item[type:]  SpectralFactory
  \end{description}

}
{
  \begin{description}
  \item[$\bullet$] the Whittle factory : {\itshape myFactory}
  \item[type:]  WhittleFactory
  \end{description}

  \begin{description}
  \item[$\bullet$] the white noise : {\itshape myWhiteNoise}
  \item[type:]  WhiteNoise
  \end{description}

  \begin{description}
  \item[$\bullet$] an ARMA process {\itshape myARMA}
  \item[type:]  ARMA
  \end{description}

}

\textspace\\
Python script for this UseCase :

\begin{lstlisting}

  # Build a Whittle factory with default SpectralModelFactory (WelchFactory)
  myFactory = WhittleFactory(p, q)

  # For example, set WelchFactory as SpectralModelFactory
  # the parameters are such as we split the time series in four blocs without overlap
  myFilteringWindow = Hanning()
  mySpectralFactory = Welch(myFilteringWindow, 4, 0)
  
  # Before setting the spectral model factory, get the default one
  myDefaultSpectralModelFactory = myFactory.getSpectralModelFactory()

  # now fix the new one
  myFactory.setSpectralModelFactory(mySpectralFactory)

  # Estimate the arma model from a time series
  myARMA = factory.build(myTimeSeries)

  # Estimate the arma model from a process sample
  #myARMA = factory.build(mySample)

  # Get the white noise of the arma given
  myWhiteNoise = myARMA.getWhiteNoise()

\end{lstlisting}

