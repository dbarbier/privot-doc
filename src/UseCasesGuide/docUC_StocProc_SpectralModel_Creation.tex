% Copyright (c)  2005-2010 EDF-EADS-PHIMECA.
% Permission is granted to copy, distribute and/or modify this document
% under the terms of the GNU Free Documentation License, Version 1.2
% or any later version published by the Free Software Foundation;
% with no Invariant Sections, no Front-Cover Texts, and no Back-Cover
% Texts.  A copy of the license is included in the section entitled "GNU
% Free Documentation License".
\renewcommand{\filename}{docUC_StocProc_SpectralModel_Creation.tex}
\renewcommand{\filetitle}{UC : Creation of a spectral model}

% \HeaderNNIILevel
% \HeaderIILevel
\HeaderIIILevel

\label{SpectralModelCreation}
\index{Stochastic Process!Normal Process}

A Normal process is parametrized by its second order model which might be a covariance model or a spectral model. 
In this part, we illustrate how user might define a spectral model and use it later to create a Normal process. \\

A spectral model is seen as a mathematical function $f : \mathbb{R} \rightarrow \mathbb{M}_d(\mathbb{C})$: for each frequency $\omega$ is associated an hermitian complex matrix.
OpenTURNS allows the User to define its own spectral model thanks to the object {\itshape UserDefinedSpectralModel}. 
Recalling the definition of such a model, inputs needed are :
\begin{itemize}
\item A frequency grid given with the object \emph{RegularGrid}. The grid should be of from $\left\{-\omega,-\omega + \delta \omega,\ldots, \omega \right\}$ with $\delta \omega$ the frequency step;
\item A collection of hermitian matrices corresponding to the spectral function thanks to the object \emph{HermitianMatrixCollection}. 
All elements of the collection should be of same dimension, which is also the dimension of the model.
\end{itemize}

As the object \emph{UserDefinedSpectralModel} inherits from \emph{SpectralModel}, it is possible to get the spectral density function for a given frequency thanks to the method {\itshape computeSpectralDensity}.
OpenTurns proceeds as following to return the result :
\begin{itemize}
\item It computes an approximation of the index in the frequency grid using the minimal frequency and frequency step values;
\item If index is higher than the size of the grid, than it returns an exception. Otherwise, it returns the hermitian matrix stored at the computed index;
\end{itemize}
Notice that this corresponds to a piecwise function.\\ \\

Details on each object may be found in the User Manual  (\href{OpenTURNS_UserManual_TUI.pdf}{see User Manual - StochasticProcess}).\\


\requirements{

  \begin{description}
  \item[$\bullet$] a frequency grid : {\itshape myFrequencyGrid}
  \item[type:]  RegularGrid
  \end{description}

  \begin{description}
  \item[$\bullet$] a collection of hermitian matrices : {\itshape myCollection}
  \item[type:]  HermitianMatrixCollection
  \end{description}

}
{
  \begin{description}
  \item[$\bullet$] a spectral model : {\itshape mySpectralModel }
  \item[type:] SpectralModel
  \end{description}

}

\textspace\\
Python script for this UseCase :

\begin{lstlisting}
  # Create the spectral model
  mySpectralModel = UserDefinedSpectralModel(myFrequencyGrid, myCollection)

  # Get the spectral density function computed at first frequency values
  frequency = myFrequencyGrid.getStart()
  frequencyStep = myFrequencyGrid.getStep()
  myHermitian = mySpectralModel.computeSpectralDensity(frequency)

  # Get the spectral function at frequency + 0.5 * frequencyStep
  mySpectralModel.computeSpectralDensity(frequency + 0.5 * frequencyStep)
  # Result is similar as myHermitian
 
\end{lstlisting}



As an example, we illustrate how to create a modified low pass model with exponential decrease as following :
\begin{itemize}
\item If $|f| < 3 Hz$, the spectral function is $1.0$
\item If $|f| > 3 Hz$, the spectral function is done by $\exp \left(- 2.0 (|f| - 3.0)^2 \right)$
\item In our example, the frequency grid is choosen as [-5,5] with $\delta f = 0.5$ Hz
\end{itemize}


\begin{lstlisting}

  # Create the frequency grid
  fmin = -5.0
  df = 0.5
  N = int(-2 * fmin / df) + 1
  myFrequencyGrid =  RegularGrid(fmin, df, N)

  # compute the index such as frequency is -3.0 Hz and 3.0 Hz
  index1 = int((-3.0 - fmin)/df)
  index2 = int((3.0 - fmin)/df)

  myCollection = HermitianMatrixCollection()
  for k in range(index1):
    frequency = myFrequencyGrid.getStart() + k * myFrequencyGrid.getStep()
    matrix = HermitianMatrix(1)
    matrix[0, 0] = exp(-2 * (abs(frequency) - 3.0)**2)
    myCollection.add(matrix)

  for k in range(index1, index2):
    matrix = HermitianMatrix(1)
    matrix[0, 0] = 1.0
    myCollection.add(matrix)

  for k in range(index2, N):
    frequency = myFrequencyGrid.getStart() + k * myFrequencyGrid.getStep()
    matrix = HermitianMatrix(1)
    matrix[0, 0] = exp(-2 * (abs(frequency) - 3.0)**2)
    myCollection.add(matrix)
      
  # Create the spectral model
  mySpectralModel = UserDefinedSpectralModel(myFrequencyGrid, myCollection)

  # Get the spectral density function computed at first frequency values
  frequency = myFrequencyGrid.getStart()
  frequencyStep = myFrequencyGrid.getStep()
  myHermitian = mySpectralModel.computeSpectralDensity(frequency)

  # Get the spectral function at frequency + 0.5 * frequencyStep
  mySpectralModel.computeSpectralDensity(frequency + 0.5 * frequencyStep)
  # Result is similar as myHermitian
 
\end{lstlisting}
