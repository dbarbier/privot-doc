% Copyright (c)  2005-2010 EDF-EADS-PHIMECA.
% Permission is granted to copy, distribute and/or modify this document
% under the terms of the GNU Free Documentation License, Version 1.2
% or any later version published by the Free Software Foundation;
% with no Invariant Sections, no Front-Cover Texts, and no Back-Cover
% Texts.  A copy of the license is included in the section entitled "GNU
% Free Documentation License".
\renewcommand{\filename}{docUC_MinMax_SpecifiedExperimentPlane.tex}
\renewcommand{\filetitle}{UC : Re-use of a specified numerical sample as experiment plane}

% \HeaderNNIILevel
% \HeaderIILevel
\HeaderIIILevel

\label{fixedExpPlane}


\index{Experiment Plane!Fixed experiment plane}

The objective of this Use Case is to enable to re-use a experiment plane, previously elaborated.\\

Details on experiment planes  may be found in the Reference Guide (\href{OpenTURNS_ReferenceGuide.pdf}{see files Reference Guide - Step C -- Min-Max approach using Experiment Planes}).\\

Details on each object may be found in the User Manual  (\href{OpenTURNS_UserManual_TUI.pdf}{see User Manual - Experiment Planes / Stratified Experiment Planes}).\\


This functionality is particularly interesting in the polynomial chaos technique, where the evaluation of the coefficients by regression requires the discretisation of a particular integral : to do that, the User may want to re-use a pre-existing numerical sample (for example, the sparse Smolyak grid) to parameterize algorithms which work with experiment plane structure (and not numerical sample one).\\

When recalled, the {\itshape generate} method regives the specified numerical sample.\\


\requirements{
  \begin{description}
  \item[$\bullet$] the previously elaborated  numerical sample : {\itshape mySample}
  \item[type:] NumericalSample
  \end{description}
}
{
  \begin{description}
  \item[$\bullet$] the experiment based on the numerical sample : {\itshape myFixedExperiment}
  \item[type:] FixedExperiment
  \end{description}
}

\textspace\\
Python script for this UseCase :

\begin{lstlisting}
  # Create a fixed experiment plane
  myFixedExperiment = FixedExperiment(mySample)
\end{lstlisting}







