% Copyright (c)  2005-2010 EDF-EADS-PHIMECA.
% Permission is granted to copy, distribute and/or modify this document
% under the terms of the GNU Free Documentation License, Version 1.2
% or any later version published by the Free Software Foundation;
% with no Invariant Sections, no Front-Cover Texts, and no Back-Cover
% Texts.  A copy of the license is included in the section entitled "GNU
% Free Documentation License".
\renewcommand{\filename}{docUC_CentralUncertainty_SobolIndices.tex}
\renewcommand{\filetitle}{UC : Sensitivity analysis : Sobol indices}

% \HeaderNNIILevel
% \HeaderIILevel
\HeaderIIILevel


\index{Sensitivity!Sobol indices}

The objective of the Use Case is to measure the correlation between input variables and the output result of a model described by a numerical function, it is called sensitivity analysis. Sobol indices allow to evaluate the importance of a single variable or a specific set of variables. Here Sobol indices are estimated by sampling, from two input samples and a numerical function.\\
In theory, Sobol indices range is $\left[0; 1\right]$ ; the more the indice value is close to 1 the more the variable is important toward the output of the function. The Sobol indices can be computed at different orders.\\ First order indices evaluate the importance of one variable at a time ($d$ indices stored in a NumericalPoint, with $d$ the input dimension of the model).\\
Second order indices evaluate the importance of every pair of variables ($\binom{d}{2} = \frac{d \times \left( d-1\right) }{2}$ indices stored via a SymmetricMatrix).\\
The $d$ total indices give the relative importance of every variables except the variable $X_i$, for every variable.\\
Computation of first and total order indices requires $N \times (d+2)$ calls to the function, and $N \times (2 \times d + 2)$ for first, second order and total indices.


Details on the  Taylor variance decomposition method may be found in the Reference Guide (\href{OpenTURNS_ReferenceGuide.pdf}{see files Reference Guide - Step C' -- Sensitivity analysis using Sobol indices}).\\

Details on each object may be found in the User Manual  (\href{OpenTURNS_UserManual_TUI.pdf}{see User Manual - Statistics on Sample / Sensitivity Analysis}).\\


\requirements{
  \begin{description}

  \item[$\bullet$] an input sample : {\itshape inputSample1}, which marginals are independently distributed
  \item[type:] NumericalSample
  \item[$\bullet$] an input sample : {\itshape inputSample2}, must be of the same dimension, size and independent from $inputSample1$
  \item[type:] NumericalSample
  \item[$\bullet$] a function : {\itshape function}, which input dimension must fit the dimension of the two samples
  \item[type:] NumericalMathFunction
  \end{description}
}
{
  \begin{description}
  \item[$\bullet$] the different Sobol indices
  \item[type:] NumericalPoint, for first and total indices
  \item[type:] SymmetricMatrix, for second order indices
  \end{description}
}
\textspace\\
Python script for this UseCase :

\begin{lstlisting}
  # Initialize computation
  sensitivityAnalysis = SensitivityAnalysis( firstInputSample, secondInputSample, function )

  # Allow multithreading if available
  sensitivityAnalysis.setBlockSize( ResourceMap.Get( "parallel-threads" ) )

  # Compute second order indices (first, second and total order indices are computed together)
  secondOrderIndice = sensitivityAnalysis.getSecondOrderIndices()

  # Retrieve first order indices
  firstOrderIndice = sensitivityAnalysis.getFirstOrderIndices()

  # Retrieve total order indices
  totalOrderIndice = sensitivityAnalysis.getTotalOrderIndices()

  # Print some indices
  print "First order Sobol indice of Y|X1 = %.6f" % firstOrderIndices[0]
  print "Total order Sobol indice of Y|X3 = %.6f" % totalOrderIndices[2]
  print "Second order Sobol indice of Y|X1,X3 = %.6f" % secondOrderIndices[0,2]

\end{lstlisting}

