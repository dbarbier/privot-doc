% Copyright (c)  2005-2010 EDF-EADS-PHIMECA.
% Permission is granted to copy, distribute and/or modify this document
% under the terms of the GNU Free Documentation License, Version 1.2
% or any later version published by the Free Software Foundation;
% with no Invariant Sections, no Front-Cover Texts, and no Back-Cover
% Texts.  A copy of the license is included in the section entitled "GNU
% Free Documentation License".
\renewcommand{\filename}{docUC_MinMax_RandomExperimentPlane.tex}
\renewcommand{\filetitle}{UC : Creation of a random experiment plane : Monte Carlo, LHS patterns}

% \HeaderNNIILevel
% \HeaderIILevel
\HeaderIIILevel

\label{randomExpPlane}


\index{Experiment Plane!Monte Carlo experiment plane}
\index{Experiment Plane!LHS experiment plane}
\index{Random Generator}

The objective of this Use Case is to define a random experiment plane  : the experiment plane does not follow a specified scheme any more. The experiment grid is generated according to a specified distribution and a specified number of points.\\


Details on experiment planes  may be found in the Reference Guide (\href{OpenTURNS_ReferenceGuide.pdf}{see files Reference Guide - Step C -- Min-Max approach using Experiment Planes}).\\

Details on each object may be found in the User Manual  (\href{OpenTURNS_UserManual_TUI.pdf}{see User Manual - Experiment Planes / Random experiment Planes}).\\


Open TURNS proposes two sampling methods to generate the experiment grid :
\begin{itemize}
\item the Monte Carlo method : the numerical sample is generated by sampling the specified distribution. When recalled, the {\itshape generate} method regenerates a new numerical sample.
\item the LHS method : the numerical sample is generated by sampling the specified distribution with the LHS technique :  some cellulars are determined, with the same probabilistic content according to the specified distribution, then cellulars are selected verifying the constraint <<one by line and column>>, then points are selected among these selected cellulars. When recalled, the {\itshape generate} method regenerates a new numerical sample : the point selection within the cellulars changes but not the cellulars selection. To change the cellular selection, it is necessary to create a new LHS Experiment.
\end{itemize}

Before any simulation, it is usefull to initialize the state of the random generator, as defined in the Use Case \ref{randomGenerator}.\\

\requirements{
  \begin{description}
  \item[$\bullet$] the specified distribution : {\itshape distribution}
  \item[type:] Distribution
  \item[$\bullet$] the number of points of the experiment plane : {\itshape number}
  \item[type:] UnsignedLong
  \end{description}
}
{
  \begin{description}
  \item[$\bullet$] the sample generated : {\itshape experimentSample}
  \item[type:] NumericalSample
  \end{description}
}

\textspace\\
Python script for this UseCase :

\begin{lstlisting}
  # Create a Monte Carlo experiment plane
  myRandomExp = MonteCarloExperiment(distribution, number)

  # Create a LHS experiment plane
  myRandomExp = LHSExperiment(distribution, number)

  # Generate the experiment plane numerical sample
  experimentSample = myRandomExp.generate()
\end{lstlisting}

