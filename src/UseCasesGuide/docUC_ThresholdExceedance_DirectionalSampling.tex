% Copyright (c)  2005-2010 EDF-EADS-PHIMECA.
% Permission is granted to copy, distribute and/or modify this document
% under the terms of the GNU Free Documentation License, Version 1.2
% or any later version published by the Free Software Foundation;
% with no Invariant Sections, no Front-Cover Texts, and no Back-Cover
% Texts.  A copy of the license is included in the section entitled "GNU
% Free Documentation License".
\renewcommand{\filename}{docUC_ThresholdExceedance_DirectionalSampling.tex}
\renewcommand{\filetitle}{UC : Creation of a Directional Sampling  simulation algorithm}

% \HeaderNNIILevel
% \HeaderIILevel
\HeaderIIILevel



\index{Threshold Probability!Directional sampling}


The Directional Sampling simulation operates in the standard space. The maximum distant point of the standard space equal to 8 by default (this value may be changed with the {\itshape setMaximumDistance()} method).\\


Details on the directional sampling method may be found in the Reference Guide (\href{OpenTURNS_ReferenceGuide.pdf}{see files Reference Guide - Step C -- Directional Sampling}).\\

Details on each object may be found in the User Manual  (\href{OpenTURNS_UserManual_TUI.pdf}{see User Manual - Threshold exceedance probability evaluation with simulation / Directional Sampling}).\\


The Directional Sampling simulation method is defined from :
\begin{itemize}
\item an event,
\item a Root Strategy :
  \begin{itemize}
  \item  RiskyAndFast : for each direction, we check whether there is a sign change of the standard limit state function between the maximum distant point (at distance {\itshape MaximumDistance} from the center of the standard space) and the center of the standard space. \\
    In case of sign change, we research one root in the segment [origin, maximum distant point] with the selected non linear solver.\\
    As soon as founded, the segment [root, infinity point] is considered within the failure space.

  \item MediumSafe : for each direction, we go along the direction by step of length {\itshape stepSize} from the origin to the maximum distant point (at distance {\itshape MaximumDistance} from the center of the standard space) and we check whether there is a sign change on each segment so formed.\\
    At the first sign change, we research one root in the concerned segment with the selected non linear solver. Then, the segment [root, maximum distant point] is considered within the failure space. \\
    If {\itshape stepSize} is small enough, this strategy garantees us to find the root which is the nearest from the origin.

  \item SafeAndSlow : for each direction, we go along the direction by step of length {\itshape stepSize} from the origin to the maximum distant point(at distance {\itshape MaximumDistance} from the center of the standard space) and we check whether there is a sign change on each segment so formed.\\
    We go until the maximum distant point.  Then, for all the segments where we detected the presence of a root, we research the root with the selected non linear solver. We evaluate the contribution to the failure probability of each segment. \\
    If {\itshape stepSize} is small enough, this strategy garantees us to find all the roots in the direction and the contribution of this direction to the failure probability is precisely evaluated.
  \end{itemize}

\item a Non Linear Solver :
  \begin{itemize}
  \item Bisection : bisection algorithm,
  \item Secant : based on the evaluation of a segment between the two last iterated points,
  \item Brent : mix of Bisection, Secant and inverse quadratic interpolation.
  \end{itemize}

\item and a Sampling Strategy :
  \begin{itemize}
  \item RandomDirection : we generate one point on the sphere unity according to the uniform distribution and we consider both opposite directions so formed. So one set of direction is composed of 2 directions.
  \item OrthogonalDirection : this strategy is parameterized by $k\in \mathbb{N}$. We generate one direct orthonormalized base $(e_1, \dots, e_n)$ within the set of orthonormalized bases. We consider all the renormalized linear combinations of $k$ vectors within the $n$ vectors of the base, where the coefficients of the linear combinations are equal to ${+1, -1}$. There are $C_n^k 2^k$ new vectors $v_i$. We consider each direction defined by each vector $v_i$. So one set of direction is composed of $C_n^k 2^k$ directions.\\
    If $k=1$, we consider all the axes of the standard space.
  \end{itemize}
\end{itemize}


The example here is a Directional Sampling simulation method defined by :
\begin{itemize}
\item its parameters by default (CASE 1) :  Root Strategy by default : Slow and Safe, Non Linear Solver : Brent algorithm, Sampling Strategy : Random Direction,
\item some other parameters (CASE 2) :  Root Strategy by default : Risky And Fast, Non Linear Solver : Brent algorithm, Sampling Strategy : Orthogonal Direction.
\end{itemize}
\vspace*{0.2cm}

\requirements{
  \begin{description}
  \item[$\bullet$] the event in physical space :  {\itshape myEvent}
  \item[type:] Event
  \item[$\bullet$] the dimension of the output variable of interest :  {\itshape dimOutput}
  \item[type:] UnsignedLong
  \end{description}
}
{
  \begin{description}
  \item[$\bullet$] Directional Sampling algorithm
  \item[type:] Simulation
  \end{description}
}

\textspace\\
Python  script for this UseCase :


\begin{lstlisting}
  # CASE 1 : Directional Sampling from an event
  # Root Strategy by default : Safe And Slow
  # Non Linear Solver : Brent algorithm
  # Sampling Strategy : Random Direction

  # Create a Directional Sampling algorithm
  myAlgo = DirectionalSampling(myEvent)


  # CASE 2 : Directional Sampling from an event, a root strategy
  # and a directional sampling strategy
  # Root Strategy by default : MediumSafe
  # Non Linear Solver : Brent algorithm
  # Sampling Strategy : Orthogonal Direction

  # Create a Directional Sampling algorithm
  myAlgo2 = DirectionalSampling(myEvent, RootStrategy(MediumSafe()), SamplingStrategy(OrthogonalDirection(dimOutput,2)))
\end{lstlisting}

