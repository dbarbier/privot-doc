% Copyright (c)  2005-2010 EDF-EADS-PHIMECA.
% Permission is granted to copy, distribute and/or modify this document
% under the terms of the GNU Free Documentation License, Version 1.2
% or any later version published by the Free Software Foundation;
% with no Invariant Sections, no Front-Cover Texts, and no Back-Cover
% Texts.  A copy of the license is included in the section entitled "GNU
% Free Documentation License".
\renewcommand{\filename}{docUC_OVI_PolyChaosExp.tex}
\renewcommand{\filetitle}{UC : Creation of the output variable of interest from the result of a polynomial chaos expansion}

% \HeaderNNIILevel
% \HeaderIILevel
\HeaderIIILevel


\label{RandomVectorPolynomialChaos}



\index{Random Vector!Output random vector}

The objective of this Use Case is to define the output variable of interest as the result of a polynomial chaos algorithm which defined a particular response surface (refer to \ref{polynomialchaosexpansion}).\\

Details on the polynomial chaos expansion may be found in the Reference Guide (\href{OpenTURNS_ReferenceGuide.pdf}{see files Reference Guide - Step Resp. Surf. -- Polynomial Chaos Expansion}).\\

Details on each object may be found in the User Manual  (\href{OpenTURNS_UserManual_TUI.pdf}{see User Manual - Non Parametric Response Surface by Functional Chaos Expansion}).\\


\requirements{
  \begin{description}
  \item[$\bullet$] the result structure of a polynomial chaos algorithm : {\itshape polynomialChaosResult}
  \item[type:] a FunctionalChaosResult
  \end{description}
}
{
  \begin{description}
  \item[$\bullet$] the new output variable of interest : {\itshape newOuputVariableOfInterest}
  \item[type:] a RandomVector
  \end{description}
}

\textspace\\
Python script for this UseCase :

\begin{lstlisting}
  # Create the new ouput variable of interest
  # based on the meta model
  # evaluated from the polynomial chaos algorithm
  newOuputVariableOfInterest = RandomVector(polynomialChaosResult)
\end{lstlisting}
