% Copyright (c)  2005-2010 EDF-EADS-PHIMECA.
% Permission is granted to copy, distribute and/or modify this document
% under the terms of the GNU Free Documentation License, Version 1.2
% or any later version published by the Free Software Foundation;
% with no Invariant Sections, no Front-Cover Texts, and no Back-Cover
% Texts.  A copy of the license is included in the section entitled "GNU
% Free Documentation License".
\renewcommand{\filename}{docUC_CentralUncertainty_CorrelationAnalysis.tex}
\renewcommand{\filetitle}{UC : Correlation analysis on samples : Pearson and Spearman coefficients, PCC, PRCC, SRC, SRRC coefficients}

% \HeaderNNIILevel
% \HeaderIILevel
\HeaderIIILevel

\label{correlationAnalysis}


\index{Correlation!Pearson correlation coefficient}
\index{Correlation!Partial Pearson correlation coefficient (PCC)}
\index{Correlation!Spearman correlation coefficient}
\index{Correlation!Partial rank correlation coefficient (PRCC)}
\index{Correlation!Standard regression coefficient (SRC)}
\index{Correlation!Standard rank regression coefficient (SRRC)}


This Use Case  describes the correlation analysis we can perform between the input random  vector, described by a numerical sample, and the output variable of interest described by a numerical sample too.\\

Details on experiment planescorrelation coefficients may be found in the Reference Guide (\href{OpenTURNS_ReferenceGuide.pdf}{see files Reference Guide - Step C' -- Uncertainty Ranking using Pearson's correlation} and files around it).\\

Details on each object may be found in the User Manual  (\href{OpenTURNS_UserManual_TUI.pdf}{see User Manual - Statistics on Sample / Correlation analysis}).\\


\requirements{
  \begin{description}
  \item[$\bullet$] a first numerical sample : {\itshape inputSample}, may be of dimension $\geq 1$
  \item[type:] NumericalSample
  \item[$\bullet$] a second numerical sample : {\itshape outputSample}, must be of dimension =1
  \item[type:] NumericalSample
  \end{description}
}
{
  \begin{description}
  \item[$\bullet$] the different correlation coefficients : {\itshape PCCcoefficient, PRCCcoefficient, SRCcoefficient, SRRCcoefficient, pearsonCorrelation, spearmanCorrelation}
  \item[type:] NumericalPoint
  \end{description}
}

\textspace\\
Python script for this UseCase :

\begin{lstlisting}
  # PCC coefficients evaluated between the outputSample and each coordinate of inputSample
  PCCcoefficient = CorrelationAnalysis.PCC(inputSample, outputSample)

  # PRCC evaluated between the outputSample and each coordinate of inputSample (based on the rank values)
  PRCCcoefficient = CorrelationAnalysis.PRCC(inputSample, outputSample)

  # SRC evaluated between the outputSample and each coordinate of inputSample
  SRCcoefficient = CorrelationAnalysis.SRC(inputSample, outputSample)

  # SRRC evaluated between the outputSample and each coordinate of inputSample (based on the rank values)
  SRRCcoefficient = CorrelationAnalysis.SRRC(inputSample, outputSample)

  # Pearson Correlation Coefficient
  # CARE :  inputSample must be of dimension 1
  pearsonCorrelation = CorrelationAnalysis.PearsonCorrelation(inputSample, outputSample)

  # Spearman Correlation Coefficient
  # CARE :  inputSample must be of dimension 1
  spearmanCorrelation = CorrelationAnalysis.SpearmanCorrelation(inputSample, outputSample)
\end{lstlisting}

