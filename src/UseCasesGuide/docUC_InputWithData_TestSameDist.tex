% Copyright (c)  2005-2010 EDF-EADS-PHIMECA.
% Permission is granted to copy, distribute and/or modify this document
% under the terms of the GNU Free Documentation License, Version 1.2
% or any later version published by the Free Software Foundation;
% with no Invariant Sections, no Front-Cover Texts, and no Back-Cover
% Texts.  A copy of the license is included in the section entitled "GNU
% Free Documentation License".
\renewcommand{\filename}{docUC_InputWithData_TestSameDist.tex}
\renewcommand{\filetitle}{UC : Do two samples have the same distribution : QQ-plot visual test, Smirnov numerical test}

% \HeaderNNIILevel
% \HeaderIILevel
\HeaderIIILevel



\index{Graph!QQ-plot}
\index{Fitting Test!QQ-plot}
\index{Comparison of distribution test!Smirnov}

The objective of this Use Case is to decide whether both samples follow the same distribution or not. \\

To help the decision, Open TURNS  proposes one visual test and one numerical test :
\begin{itemize}
\item the QQ-plot visual test : Open Turns associates the empirical quantiles of each data from the both numerical samples,

\item the Smirnov test : it tests if both samples (continuous ones only) follow the same distribution. If $F_{n_1}^{*}$ and $F_{n_2}^{*}$ are the empirical cumulative density functions of both samples of size $n_1$ and $n_2$, the Smirnov test evaluates the decision variable :
  $$
  D^2 = \displaystyle \sqrt{\frac{n_1n_2}{n_1+n_2}} \sup_{x}|F_{n_1}^{*}(x) - F_{n_2}^{*}(x)|
  $$
  which tends towards the Kolmogorov distribution. The hypothesis of same distribution is rejected if $D^2$ is too high (depending on the p-value threshold).
\end{itemize}



Details on the QQ-polt and Kolmogorov-Smirnov  tests may be found in the Reference Guide (\href{OpenTURNS_ReferenceGuide.pdf}{see files Reference Guide - Step B -- Using QQ-plot to compare two samples and Step B -- Comparison of two samples using Kolmogorov-Smirnov test}).\\

Details on each object may be found in the User Manual  (\href{OpenTURNS_UserManual_TUI.pdf}{see User Manual - Statistics on sample /  Visual Test and Hypothesis Test}).\\


\requirements{
  \begin{description}
  \item[$\bullet$] two numerical continuous samples of dimension 1  : {\itshape continuousSample1, continuousSample2}
  \item[type:]  NumericalSample
  \end{description}
}
{
  \begin{description}
  \item[$\bullet$] the files containing the QQ-plot graph : {\itshape twoSamplesQQPlot.png, twoSamplesQQPlot.eps}
  \item[type:]  files at format PNG or EPS or FIG
  \item[$\bullet$] test result : {\itshape resultSmirnov}
  \item[type:] TestResult
  \end{description}
}

\textspace\\
Python script for this UseCase  :

\begin{lstlisting}
  # GRAPH 1 : QQ-plot graph
  # Generate the Graph structure for the QQ-plot graph
  # number of points of the graph fixed to 100 (20 by default)
  twoSamplesQQPlot = VisualTest.DrawQQplot(continuousSample1, continuousSample2, 100)

  # Impose a bounding box : x-range and y-range
  # boundingBox = [xmin, xmax, ymin, ymax]
  myBoundingBox = NumericalPoint( (xmin, xmax, ymin, ymax) )
  twoSamplesQQPlot.setBoundingBox(myBoundingBox)

  # In order to see the graph without creating the associated files
  Show(twoSamplesQQPlot)

  # Draw the graph on the file twoSamplesQQPlot.png and twoSamplesQQPlot.eps
  # if the file adress is not fulfilled, the file is created in the current directory
  twoSamplesQQPlot.draw("twoSamplesQQPlot")

  # View the bitmap file
  ViewImage(twoSamplesQQPlot.getBitmap())

  # Check if it worked
  print "bitmap = " , twoSamplesQQPlot.getBitmap()
  print "postscript = " , twoSamplesQQPlot.getPostscript()

  # Smirnov Test : test if two samples have a monotonous relation
  # H0 : same continuous distribution
  # Test = True <=> same continuous distribution
  # p-value threshold : probability of the H0 reject zone : 1-0.90
  # p-value : probability (test variable decision > test variable decision evaluated on the samples)
  # Test = True <=> p-value > p-value threshold
  resultSmirnov = HypothesisTest.Smirnov(continuousSample1, continuousSample2, 0.90)

  # Print result of the Smirnov Test
  print "Test Succes ? ", (resultSmirnov.getBinaryQualityMeasure()==1)

  # Get the p-value of the Smirnov Test
  print "p-value of the Smirnov Test = ", resultSmirnov.getPValue()

  # Get the p-value threshold of the  Test
  print "p-value threshold = ", resultSmirnov.getThreshold()
\end{lstlisting}



