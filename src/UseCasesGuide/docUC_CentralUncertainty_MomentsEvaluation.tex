% Copyright (c)  2005-2010 EDF-EADS-PHIMECA.
% Permission is granted to copy, distribute and/or modify this document
% under the terms of the GNU Free Documentation License, Version 1.2
% or any later version published by the Free Software Foundation;
% with no Invariant Sections, no Front-Cover Texts, and no Back-Cover
% Texts.  A copy of the license is included in the section entitled "GNU
% Free Documentation License".
\renewcommand{\filename}{docUC_CentralUncertainty_MomentsEvaluation.tex}
\renewcommand{\filetitle}{UC : Moments evaluation of a random sample of the output variable of interest}

% \HeaderNNIILevel
% \HeaderIILevel
\HeaderIIILevel




\index{Sample Statistics!Moments evaluation}
\index{Graph!Taylor variance decomposition importance factors}
\index{Graph Manipulation!ViewImage}
\index{Graph Manipulation!Show}

The objective of this Use Case  is to evaluate the mean and standard deviation of the output variable of interest by generating a random sample of the output variable of interest and evaluate the empirical indicators from that sample.\\



Details on empirical moments evaluation  may be found in the Reference Guide (\href{OpenTURNS_ReferenceGuide.pdf}{see files Reference Guide - Step C -- Estimating the mean and variance using the Monte Carlo Method}).\\

Details on each object may be found in the User Manual  (\href{OpenTURNS_UserManual_TUI.pdf}{see User Manual - Statistics on Sample / Numerical sample}).\\

\requirements{
  \begin{description}
  \item[$\bullet$] the output variable of interest : {\itshape output}, which may be of dimension $\geq 1$
  \item[type:] RandomVector which implementation is a CompositeRandomVector
  \end{description}
}
{
  \begin{description}
  \item[$\bullet$] Mean and covariance of the variable of interest
  \item[type:] NumericalPoint, CovarianceMatrix
  \end{description}
}

\textspace\\
Python script for this UseCase :

\begin{lstlisting}
  # Create a random sample of the output variabe of interest of size 1000
  size = 1000
  outputSample = output.getNumericalSample(size)

  # Get the empirical mean
  empiricalMean = outputSample.computeMean()
  print "Empirical Mean = ", empiricalMean

  # Get the empirical covariance matrix
  empiricalCovarianceMatrix = outputSample.computeCovariance()
  print "Empirical Covariance Matrix = ", empiricalCovarianceMatrix

  # Get the standard deviation of the i-th component of the output variabe of interest
  # Import the sqrt functionality from the math python library
  from math import sqrt
  for i in range(output.getDimension()) :
  print "Standard deviation of component", i+1,  " = ", sqrt(empiricalCovarianceMatrix[i,i])
\end{lstlisting}

