% Copyright (c)  2005-2010 EDF-EADS-PHIMECA.
% Permission is granted to copy, distribute and/or modify this document
% under the terms of the GNU Free Documentation License, Version 1.2
% or any later version published by the Free Software Foundation;
% with no Invariant Sections, no Front-Cover Texts, and no Back-Cover
% Texts.  A copy of the license is included in the section entitled "GNU
% Free Documentation License".
\renewcommand{\filename}{docUC_InputNoData_InputRandomVector.tex}
\renewcommand{\filetitle}{UC : Creation  of the random input vector from a distribution}

% \HeaderNNIILevel
% \HeaderIILevel
\HeaderIIILevel


\label{UsualRandomVector}

\index{Random Vector!Input random vector}


The objective of this Use Case is to model a random  vector described by its joint probability density function. This random vector is called a {\itshape UsualRandomvector}. This UC is particularly adapted to the input random vector.\\

Details on each object may be found in the User Manual  (\href{OpenTURNS_UserManual_TUI.pdf}{see User Manual - Probabilistic modeling / Random Vector}).\\

\noindent%
\requirements{
  \begin{description}
  \item[$\bullet$] the input distribution : {\itshape inputDistribution}
  \item[type:] Distribution
  \end{description}
}
{
  \begin{description}
  \item[$\bullet$] the random input vector : {\itshape inputRandomVector}
  \item[type:] RandomVector which implementation is a UsualRandomVector
  \end{description}
}

\textspace\\
Python script for this UseCase :


\begin{lstlisting}

  # Create the UsualRandomVector 'inputRandomVector' from
  # the Distribution 'inputDistribution'
  inputRandomVector = RandomVector(inputDistribution)
\end{lstlisting}



