% Copyright (c)  2005-2010 EDF-EADS-PHIMECA.
% Permission is granted to copy, distribute and/or modify this document
% under the terms of the GNU Free Documentation License, Version 1.2
% or any later version published by the Free Software Foundation;
% with no Invariant Sections, no Front-Cover Texts, and no Back-Cover
% Texts.  A copy of the license is included in the section entitled "GNU
% Free Documentation License".
\renewcommand{\filename}{docUC_ProcessSample.tex}
\renewcommand{\filetitle}{UC : Manipulation of a process sample}

% \HeaderNNIILevel
\HeaderIILevel
%\HeaderIIILevel

\label{UCprocessSample}


\index{Stochastic Process!Process Sample}


We recall that  $\underline{X}(\omega,t)$ is a stochastic process, where $\omega \in \Omega$ is an event, $t \in \mathbb{R}$ is the time and $\underline{X}(\omega,t) \in \mathbb{R}^d$ is the observed values of the process at each time $t$.\\

The objective here is to create and manipulate a process sample which is a sample of realizations of the same stochastic process. In particular, all the process included share the same time grid.\\
We note a process sample : $\left\{(t_i, \underline{X}_i^k)_{0 \leq i \leq n-1} \right\}_{k=1 \hdots N}$ where :
\begin{itemize}
\item $N$ is the number of time series of the process sample;
\item $n$ is the number of points of the time grid, which is common to all the time series ;
\item $d$ is the dimension of the stochastic process : for all $i \in \{0,\dots,n\}$ and $k\in\{0,\dots,N-1\}$, $\underline{X}_i^k \in \mathbb{R}^d$.
\end{itemize}

Given a process sample, it is possible to evaluate :
\begin{itemize}
\item its stochastic mean  $\vect{m}(t) = E[\underline{X}(\omega,t)] \in \mathbb{R}^d$ depending on $t$.  $\vect{m}(t)$ is evaluated on the time grid $(t_0, \hdots, t_{n-1})$ as follows :
  \begin{equation}\label{stocMean}
    \forall  i\in[0,n], \vect{m}(t_i) = E[\underline{X}(\omega,t_i)]  \simeq \displaystyle  \frac{1}{N} \sum_{k=0}^{N-1} \underline{X}_i^k
  \end{equation}
  The stochastic mean is computed with the method \emph{computeMean()} which returns the time series  $(t_i,\vect{m}(t_i))_{i=0 \hdots n}$ of dimension $d$ and size $n$.

\item its temporal mean $(\vect{g}(\omega_k))_{0 \leq k \leq N-1}$, which is the temporal mean of each time series of the process sample, depending on the alea $\omega_k$. It is evaluated as follows, where the time is discretized on the time grid :
  \begin{equation}\label{tempMean}
    \displaystyle \forall  k\in[0,N-1], \vect{g}(\omega_k) =  \frac{1}{t_{n-1} - t_0} \int_{t_0}^{t_{n-1}} \underline{X}(\omega_k,t) \, dt \simeq  \frac{1}{n} \sum_{i=0}^{n-1} \underline{X}_i^k
  \end{equation}
  The temporal mean is computed with the method \emph{computeTemporalMean()} which returns a numerical sample of dimension $d$ and size $N$.

\item its quantiles per component $\vect{Q}(t, q)$ of level $q$ such that $\forall i\in\{1,\dots,d\},\:Q_i(t, q) = \inf\{x\in\Rset\,|\,\mathbb{P}(X_i(\omega,t)\leq x)>q\}$ depending on $t$. $\vect{Q}(t, q)$ is evaluated on the time grid $(t_0, \hdots, t_{n-1})$ using the empirical quantile of level $q$ of the sample associated to each time stamp.

  The quantiles per component are computed with the method \emph{computeQuantilePerComponent(q)} which returns the time series  $(t_i,\vect{Q}(t_i, q))_{i=0 \hdots n}$ of dimension $d$ and size $n$.

\end{itemize}

In case of ergodic process, we have the equality :
\begin{equation}\label{ergodic}
  \forall t >0, \forall \omega_k\in \Omega, \ \vect{m}(t) =  \vect{g}(\omega_k)
\end{equation}
which implies that the stochastic mean is invariant with time : $\forall t >0, \vect{m}(t)=\vect{m}$ and the temporal mean is independent of the alea $\omega_k$ : $ \forall \omega_k\in \Omega,  \vect{g}(\omega_k)= \vect{g}$.\\


Details on each object may be found in the User Manual  (\href{OpenTURNS_UserManual_TUI.pdf}{see User Manual - Stochastic Process}).\\

\requirements{
  \begin{description}
  \item[$\bullet$] a collection of time series : {\itshape collection}
  \item[type:]  TimeSeriesCollection
  \end{description}

  \begin{description}
  \item[$\bullet$] the number of process in the process sample : {\itshape N}
  \item[type:]  UnsignedLong
  \end{description}

  \begin{description}
  \item[$\bullet$] the dimension of the processes  : {\itshape d}
  \item[type:]  UnsignedLong
  \end{description}

  \begin{description}
  \item[$\bullet$] a time grid : {\itshape myTimeGrid}
  \item[type:]  RegularGrid
  \end{description}

  \begin{description}
  \item[$\bullet$] a time serie : {\itshape myTimeSeries}
  \item[type:]  TimeSeries
  \end{description}

  \begin{description}
  \item[$\bullet$] a quantile level : {\itshape q}
  \item[type:]  NumericalScalar
  \end{description}
}
{
  \begin{description}
  \item[$\bullet$] a sample of processes : {\itshape myProcessSample}
  \item[type:]  ProcessSample
  \end{description}

  \begin{description}
  \item[$\bullet$] the stochastic mean process : {\itshape myStochasticMeanProcess}
  \item[type:] TimeSeries
  \end{description}

  \begin{description}
  \item[$\bullet$] the temporal mean process : {\itshape myTemporalMean}
  \item[type:] NumericalSample
  \end{description}

  \begin{description}
  \item[$\bullet$] the quantiles per component : {\itshape myQuantiles}
  \item[type:] TimeSeries
  \end{description}
}

\textspace\\
Python script for this UseCase :

\begin{lstlisting}
  # Creation using a TimeSeries and size arguments
  # The return object is a ProcessSample
  # where the time serie is duplicated size-times
  myProcessSample = ProcessSample(size, myTimeSeries)

  # Creation using a  collection of TimeSeries
  # This collection must have the same RegularGrid
  myProcessSample = ProcessSample(collection)

  # It is possible to create a process sample with default values
  # Here, the process sample contains N-times the same time serie
  # each time serie being of dimension d
  # fulfilled with 0 values
  # and associated to the time grid myTimeGrid
  myProcessSample = ProcessSample(myTimeGrid, d, N)

  # Get the common time grid of all the processes
  # included in the  process sample
  timeGrid = myProcessSample.getTimeGrid()

  # Get the number N of processes of the process sample
  myProcessSampleSize = myProcessSample.getSize()

  # Get the dimension d of the processes of the process sample
  myProcessSampleDimension = myProcessSample.getDimension()

  # Add a TimeSeries to the process sample
  myProcessSample.add(timeSerie)

  # Get the time serie of index i
  # Care! Numerotation begins at i=0
  myTimeSerieIndexI = myProcessSample[i]

  # Set a particular time serie at index j
  # Care! Numerotation begins at j=0
  myProcessSample[j] = myTimeSeries

  # Compute the stochastic mean process of several time series
  # The result is a time serie
  myStochasticMeanProcess = myProcessSample.computeMean()

  # Compute the temporal mean of several time series
  # The result is a numerical sample
  myTemporalMean = myProcessSample.computeTemporalMean()

  # Compute the quantiles per component associated to the level q
  # The result is a time serie
  myQuantiles = myProcessSample.computeQuantilePerComponent(q)

\end{lstlisting}
