% Copyright (c)  2005-2010 EDF-EADS-PHIMECA.
% Permission is granted to copy, distribute and/or modify this document
% under the terms of the GNU Free Documentation License, Version 1.2
% or any later version published by the Free Software Foundation;
% with no Invariant Sections, no Front-Cover Texts, and no Back-Cover
% Texts.  A copy of the license is included in the section entitled "GNU
% Free Documentation License".
\renewcommand{\filename}{docUC_ThresholdExceedance_Event.tex}
\renewcommand{\filetitle}{UC : Creation of an event in the physical and the standard spaces}

% \HeaderNNIILevel
% \HeaderIILevel
\HeaderIIILevel

\label{StandardPhysicalEvent}

\index{Event!Physical space}
\index{Event!Standard space}


This section gives elements to create events in the physical space {\itshape Event} and in the standard space {\itshape StandardEvent}.\\



Details on isoproabilitic transformations  may be found in the Reference Guide (\href{OpenTURNS_ReferenceGuide.pdf}{see files Reference Guide - Step C -- Isoprobabilistic transformation preliminary to FORM-SORM methods}).\\


Details on each object may be found in the User Manual  (\href{OpenTURNS_UserManual_TUI.pdf}{see User Manual - Threshold exceedance probability evaluation with reliability algorithm / Reliability Algorithms}).\\



The example here is the output variable {\itshape output} defined from the limit state function {\itshape poutre} defined in  Eq.(\ref{equatPoutre}) and the input random  vector $(E,F,L,I)$. The event considered is :
$$
myEvent = \{ (E,F,L,I) \in \mathbb{R}^4 / poutre(E,F,L,I) \leq -1.5\}.
$$

\requirements{
  \begin{description}
  \item[$\bullet$] the random input vector : {\itshape input}
  \item[type:] RandomVector which implementation is a UsualRandomVector
  \item[$\bullet$] the output variable of interest : {\itshape output} of dimension 1
  \item[type:] RandomVector which implementation is a CompositeRandomVector
  \end{description}
}
{
  \begin{description}
  \item[$\bullet$] the events in the physical and standard spaces : {\itshape myEvent}, {\itshape myStandardEvent}
  \item[type:] Event and StandardEvent
  \end{description}
}

\textspace\\
Python  script for this UseCase :

\begin{lstlisting}
  # To have a beautifull graph of importance factors,
  # give the output random variable a name
  output.setName("Output_1")

  # Create an event in the physical space
  # from the variable of interest 'output'
  # Give it a name ofr graphs
  myEvent = Event(output, ComparisonOperator(Less()), -1.5, "Event 1")

  # Create an standard event in the standard space
  # 1 : from the variable of interest 'output'
  myStandardEvent = StandardEvent(output,ComparisonOperator(Less()), 1.0)

  # 2 : Build a standard event based on an event
  myStandardEvent2 = StandardEvent(myEvent)
\end{lstlisting}


