% Copyright (c)  2005-2010 EDF-EADS-PHIMECA.
% Permission is granted to copy, distribute and/or modify this document
% under the terms of the GNU Free Documentation License, Version 1.2
% or any later version published by the Free Software Foundation;
% with no Invariant Sections, no Front-Cover Texts, and no Back-Cover
% Texts.  A copy of the license is included in the section entitled "GNU
% Free Documentation License".
\renewcommand{\filename}{docUC_RegularGrid.tex}
\renewcommand{\filetitle}{UC : Creation of a time grid}

% \HeaderNNIILevel
 \HeaderIILevel
%\HeaderIIILevel

\label{UCtimeGrig}


\index{Stochastic Process!Time Grid}


This section details first how to create a time grid.\\

All the stochastic process are associated to a discrete time grid that writes $(t_0, t_1, \hdots, t_{n})$.\\ 
The time grid can be defined using $(t_{Min}, n, \Delta t)$ where $n$ is the number of steps. $\Delta t$ the time step between two consecutive time stamps and $t_0 = t_{Min}$ is the initial time of the grid. Then,  $t_k = t_{Min} + k \Delta t$ and $t_{Max} = t_{Min} +  n \Delta t$.

Details on each object may be found in the User Manual  (\href{OpenTURNS_UserManual_TUI.pdf}{see User Manual -  Stochastic Process}).\\

\requirements{
  \begin{description}
  \item[$\bullet$] initial time {\itshape $tMin$}
  \item[type:]  NumericalScalar
  \end{description}

  \begin{description}
  \item[$\bullet$] time step : {\itshape $timeStep$}
  \item[type:]  NumericalScalar
  \end{description}

  \begin{description}
  \item[$\bullet$] number of timesteps   : {\itshape $n$}
  \item[type:]  integer
  \end{description}

}
{
  \begin{description}
  \item[$\bullet$] a time grid : {\itshape myRegularGrid}
  \item[type:]  RegularGrid
  \end{description}
}

\textspace\\
Python script for this UseCase :

\begin{lstlisting}
  # Create a time grid

  # Define the min bound of the time grid, the number of steps points
  # and the time step between two time stamps
  # Care ! this point number does not include the bounds (total number of points is n+1)
  tMin = 0.
  timeStep = 0.1
  n = 10

  # Create the RegularGrid
  myRegularGrid = RegularGrid(tMin, timeStep, n)

  # Get the min, max, step and number of points of a RegularGrid
  myMin = myRegularGrid.getStart()
  myMax = myRegularGrid.getEnd()
  myStep = myRegularGrid.getStep()
  myRegularGridSize = myRegularGrid.getN()


\end{lstlisting}



