% Copyright (c)  2005-2010 EDF-EADS-PHIMECA.
% Permission is granted to copy, distribute and/or modify this document
% under the terms of the GNU Free Documentation License, Version 1.2
% or any later version published by the Free Software Foundation;
% with no Invariant Sections, no Front-Cover Texts, and no Back-Cover
% Texts.  A copy of the license is included in the section entitled "GNU
% Free Documentation License".
\renewcommand{\filename}{docUC_Intro_GeneralCommands}
\renewcommand{\filetitle}{Some useful general commands}

\HeaderNNIILevel

$\boldsymbol{\Longrightarrow}$ {\bf help command}\\

The command {\itshape help} gives detailed information on each  object of the {\itshape openturns} python library. For example, to get information on the object {\itshape NumericalPoint}, the command is :
\begin{center}
  \begin{lstlisting}
    help(NumericalPoint)
  \end{lstlisting}
\end{center}
or
\begin{center}
  \begin{lstlisting}
    help(openturns.NumericalPoint)
  \end{lstlisting}
\end{center}
according to the way the  {\itshape openturns} python module has been loaded.\\
In order to quit the \emph{help} document, tape the key {\itshape q}.\\

$\boldsymbol{\Longrightarrow}$ {\bf Styles in LOG messages}\\

It is possible to set the style of the LOG messages by using the \emph{TTY} class. Type the command :
\begin{center}
  \begin{lstlisting}
    help(TTY)
  \end{lstlisting}
\end{center}
in order to get the codes of all the possible styles. For example, the bold red color $REDBG$ is coded by $13$.\\
Then to print a message in a specific style, associated to the code $N$, type the command :
\begin{center}
  \begin{lstlisting}
    print TTY.GetColor(N), 'my message in style code'
  \end{lstlisting}
\end{center}

In order to inhibit any future change of style, type the command : 
\begin{center}
  \begin{lstlisting}
    TTY.ShowColors('false')
  \end{lstlisting}
\end{center}
and to make it possible again, type the command :
\begin{center}
  \begin{lstlisting}
    TTY.ShowColors('true')
  \end{lstlisting}
\end{center}
By default, it is possible to change the messages style.\\

$\boldsymbol{\Longrightarrow}$ {\bf List of methods}\\

If {\itshape myObject} is one instance of an {\itshape openturns} object, then the command :
\begin{center}
  \begin{lstlisting}
    myObject.[TAB]
  \end{lstlisting}
\end{center}
lists all the methods proposed by the object {\itshape myObject}.\\

More generally, to list all the methods proposed by the {\itshape NumericalMathFunction} object, type the command :
\begin{center}
  \begin{lstlisting}
    NumericalMathFunction.[TAB]
  \end{lstlisting}
\end{center}


{$\boldsymbol{\Longrightarrow}$ \bf Automatic completion}\\

In order to have some automatic completion of the {\itshape openturns} objects and their methods, it is necessary to type the following command in the current python session :
\begin{center}
  \begin{lstlisting}
    import readline
    import rlcompleter
    readline.parse_and_bind("tab: complete")
  \end{lstlisting}
\end{center}

These commands may be written in the file {\itshape .pythonrc.py} put in the root repertory {\itshape \$HOME} : it will be automatically taken into account  for current python sessions.\\
Then, in order to complete and list all the {\itshape openturns} objects which begin by {\itshape Num}, the command is :
\begin{center}
  \begin{lstlisting}
    Num[TAB]
  \end{lstlisting}
\end{center}

where $[TAB]$ is the Tabulation key.\\


$\boldsymbol{\Longrightarrow}$ {\bf Manipulation of a $\boldsymbol{NumericalSample}$}\\

To get the value $value$ of the  $(j+1)-th$ component of $(i+1)-th$ point of the NumericalSample $myNumSample$, type the following command : 

\begin{center}
  \begin{lstlisting}
    value = myNumSample[i,j]
  \end{lstlisting}
\end{center}

To set the value $myValue$ of the  $(j+1)-th$ component of $(i+1)-th$ point of the NumericalSample $myNumSample$, type the following command : 

\begin{center}
  \begin{lstlisting}
    myNumSample[i,j] = myValue
  \end{lstlisting}
\end{center}