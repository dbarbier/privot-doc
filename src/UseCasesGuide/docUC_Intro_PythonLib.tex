% Copyright (c)  2005-2010 EDF-EADS-PHIMECA.
% Permission is granted to copy, distribute and/or modify this document
% under the terms of the GNU Free Documentation License, Version 1.2
% or any later version published by the Free Software Foundation;
% with no Invariant Sections, no Front-Cover Texts, and no Back-Cover
% Texts.  A copy of the license is included in the section entitled "GNU
% Free Documentation License".
\renewcommand{\filename}{docUC_Intro_PythonLib}
\renewcommand{\filetitle}{Loading the openturns python library}

\HeaderNNIILevel
% \HeaderIILevel
% \HeaderIIILevel



In order to write a python file using fonctionalities proposed by the \emph{openturns} python module, it is necessary to load the module in the python shell. If there is no danger to overload functionalities coming from other python modules, the loading command is :

\lstset{language=python, keywordstyle=\color{black}\bfseries,tabsize=2,framexleftmargin=8mm,frame=shadowbox,rulesepcolor=\color{black},numbers=left,breaklines=true}

\begin{center}
  \begin{lstlisting}
    from openturns import *
  \end{lstlisting}
\end{center}
Otherwise, if some functionalities of the {\itshape openturns} python module might overload some functionalities coming from other python modules, it is preferable to launch the command :
\begin{center}
  \begin{lstlisting}
    import openturns
  \end{lstlisting}
\end{center}
In that second case, each call to an {\itshape openturns} type must be accompagnied by the prefix {\itshape openturns}. For example, to create a {\itshape NumericalPoint} of dimension 2, the command is {\itshape myNumericalPoint = openturns.NumericalPoint(2)}.\\

In order to visualize graphics through the TUI, it is necessary to import the functionality {\itshape ViewImage} from the  {\itshape openturns.viewer} module, thanks to the command :
\begin{center}
  \begin{lstlisting}
    from openturns.viewer import ViewImage
  \end{lstlisting}
\end{center}

The command :
\begin{center}
  \begin{lstlisting}
    dir()
  \end{lstlisting}
\end{center}
gives a general overview of the whole objects proposed by the \emph{openturns} python library.\\

If you want to remove the welcome message that shows the current OpenTURNS version, you can either add the {\itshape --silent} command line option:
\begin{verbatim}
  python myScript.py --silent
\end{verbatim}

or set the {\itshape OPENTURNS\_PYTHON\_SILENT} environment variable to some non empty value, e.g.

\begin{verbatim}
  export OPENTURNS_PYTHON_SILENT=Y
  python myScript.py
\end{verbatim}


