% Copyright (c)  2005-2010 EDF-EADS-PHIMECA.
% Permission is granted to copy, distribute and/or modify this document
% under the terms of the GNU Free Documentation License, Version 1.2
% or any later version published by the Free Software Foundation;
% with no Invariant Sections, no Front-Cover Texts, and no Back-Cover
% Texts.  A copy of the license is included in the section entitled "GNU
% Free Documentation License".
\renewcommand{\filename}{docUC_StocProc_ARMA_Creation.tex}
\renewcommand{\filetitle}{UC : Creation of an ARMA model}

% \HeaderNNIILevel
% \HeaderIILevel
\HeaderIIILevel

\label{ARMACreation}


\index{Stochastic Process!ARMA}




The creation of an ARMA model requires the data of the AR and MA coefficients which are : 
\begin{itemize}
\item a list of scalars in the unidmensional case : $(a_1, \hdots, a_p)$ for the AR-coefficients and  $(b_1, \hdots, b_q)$ for the MA-coefficients, defined thanks to a \emph{NumericalPoint}
\item a list of square matrix $( \mat{A}_{\, 1}, \hdots, \mat{A}{\, _p})$ for the AR-coefficients and  $(\mat{B}_{\, 1}\, \hdots, \mat{B}_{\, q} )$ for the MA-coefficients, defined thanks to a \emph{SquareMatrixCollection}
\end{itemize}
Il also requires the definition of a white noise $ \vect{\varepsilon}$ that contains the  time grid of the process.\\

The current state of an ARMA model is characterized by its last $p$ values and the last $q$ values of its white noise. It is possible to get that state thanks to the methods \emph{getState}. \\



It is possible to create an ARMA with a specific current state. That specific current state is  taken into account to  generate possible futurs but not to generate realizations (in order to respect the stationarity property of the model).\\

At the creation step, Open TURNS checks whether the process $ARMA(p,q)$ is stationnary, by evaluating the roots of the polynomials (\ref{PolPhi})  associated to the homogeneous system (\ref{dimnHom}). When the process is not stationnary, Open TURNS sends a message to prevent the User.\\


Details on each object may be found in the User Manual  (\href{OpenTURNS_UserManual_TUI.pdf}{see User Manual - Stochastic Process}).\\



\requirements{
  \begin{description}
  \item[$\bullet$] coefficients : {\itshape myARList}, {\itshape myMAList}
  \item[type:]  NumericalPoint or SquareMatrixCollection
  \end{description}

  \begin{description}
  \item[$\bullet$] a white noise : {\itshape myWhiteNoise}
  \item[type:]  WhiteNoise
  \end{description}

  \begin{description}
  \item[$\bullet$] the last realizations : {\itshape $myLastValues, myLastNoiseValues$}
  \item[type:]  NumericalSample
  \end{description}

}
{
  \begin{description}
  \item[$\bullet$] the AR and MA coefficients : {\itshape myARCoef, myMACoef}
  \item[type:]  ARMACoefficients
  \end{description}

  \begin{description}
  \item[$\bullet$] an ARMA process {\itshape myARMA}
  \item[type:]  ARMA
  \end{description}

  \begin{description}
  \item[$\bullet$] an ARMAState {\itshape myARMAState}
  \item[type:]  ARMAState
  \end{description}
}

\textspace\\
Python script for this UseCase :

\begin{lstlisting}
  # CASE 1 : Whithout specifying the current state

  # Create the AR and MA coefficients

  # From the lists of the coefficeints
  # which are vectors in dimension 1 and 
  # square matrix in diemnsion d
  myARCoef = ARMACoefficients(myARList)
  myMACoef = ARMACoefficients(myMAList)
  
  
  # Create the ARMA model

  # From the AR-MA coefficients, the white noise 
  # whithout specifying the current state
  myARMA = ARMA(myARCoef, myMACoef, myWhiteNoise)

   
  # CASE 2 : Specifying the current state
  # Usefull to get possible futurs from the current state 

  # Define the current state of the ARMA
  
  # Set the last p-values of the process 
  # and the last q-values of the noise
  myARMAState = ARMAState(myLastValues, myLastNoiseValues)
  
  
  # Create the ARMA model
 
  # From the AR-MA coefficients, the white noise and a specific state
  myARMA = ARMA(myARCoef, myMACoef, myWhiteNoise, myARMAState)
\end{lstlisting}

