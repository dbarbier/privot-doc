% Copyright (c)  2005-2010 EDF-EADS-PHIMECA.
% Permission is granted to copy, distribute and/or modify this document
% under the terms of the GNU Free Documentation License, Version 1.2
% or any later version published by the Free Software Foundation;
% with no Invariant Sections, no Front-Cover Texts, and no Back-Cover
% Texts.  A copy of the license is included in the section entitled "GNU
% Free Documentation License".
\renewcommand{\etapemethodo}{Resp. Surf.}
\renewcommand{\nomfichier}{docref_SurfRep_OrthoPoly}
\renewcommand{\titrefiche}{Orthogonal polynomials}

%abbreviations
\newcommand{\bs}[1]{\underline #1}
\newcommand{\E}[1]{\mathbb{E}\left[ #1 \right]}
\newcommand{\Esp}[2]{{\rm E}_{#1} \left[ #2 \right]}
\newcommand{\var}[1]{\mathbb{V} \left[ #1 \right]}
\newcommand{\idx}{\bs{\alpha}}
\newcommand{\Set}{\mathcal{A}}
\newcommand{\ED}{\mathcal{X}}
\newcommand{\Rn}{\mathbb{R}^{n}}
\newcommand{\RP}{\mathbb{R}^{P}}
\newcommand{\RQ}{\mathbb{R}^{Q}}
\newcommand{\ie}{i.e.}
\newcommand{\R}{\mathbb{R}}

\Header

\MathematicalDescription{

\underline{\textbf{Goal}} \vspace{2mm}

This section provides some mathematical details on sequences of orthogonal polynomials. Some of these sequences will be used to construct the basis of the so-called \emph{polynomial chaos expansion}. \\



\textbf{Mathematical framework} \vspace{2mm}

 The orthogonal polynomials are associated to an inner product, defined as follows:\\
  Given an \emph{interval of orthogonality} $[\alpha,\beta]$ ($\alpha \in \mathbb{R} \cup \{-\infty\}$, $\beta \in \mathbb{R} \cup \{\infty\}$, $\alpha < \beta$) and a weight function $w(x)> 0 $, every pair of polynomials $P$ and $Q$ are orthogonal iff  : $$ \langle P,Q \rangle = \int_{\alpha}^{\beta}P(x)Q(x)~w(x) dx = 0.$$

  Therefore, a sequence of orthogonal polynomials $(P_n)_{n\geq 0}$ ($P_n$: polynomial of degree $n$) verifies: $$\langle P_m,P_n\rangle = 0 \text{~~for every~~} m \neq n.$$
  The chosen inner product induces a norm on polynomials in the usual way: $$\parallel P_n\parallel=\langle P_n,P_n \rangle^{1/2}.$$
 % From orthogonal polynomials, one can easily get orthonormal polynomials.\\

In the following, we consider weight functions $w(x)$ corresponding to \emph{probability density functions}, which satisfy:
$$ \int_{\alpha}^{\beta} \; w(x) \;  dx \, \, = \,\, 1 $$
Moreover, we consider families of \emph{orthonormal} polynomials $(P_n)_{n\geq 0}$, that is polynomials with a unit norm:
$$ \|P_n\| \, \, = \, \, 1.$$

  \underline{Recurrence:} Any sequence of orthogonal polynomials has a recurrence formula relating any three consecutive polynomials as follows: $$P_{n+1}\ =\ (a_nx+b_n)\ P_n\ +\ c_n\ P_{n-1}.$$ \\

\textbf{Orthogonormal polynomials with respect to usual probability distributions} \vspace{2mm}

  Below, a table showing an example of particular (normalized) orthogonal polynomials associated with \emph{continuous} weight functions.\\


  \begin{center}

    \begin{tabular}{|c|c|c|c|}
      \hline
      Ortho. poly. &  $P_n(x)$ &  Weight $w(x)^{\strut}$                                   &  Recurrence coefficients $(a_n,b_n,c_n)$   \\
      \hline\hline
      Hermite  &  ${He}_n(x)^{\strut}$            &  $\displaystyle \frac{1}{\sqrt{2 \pi}} e^{-\frac{x^2}{2}}$  &   $\begin{array}{ccc} a_n & = & \frac{1}{\sqrt{n+1}} \\     b_n & = & 0 \\ c_n & = &  - \sqrt{\frac{n}{n+1}} \end{array} $ \\
      \hline
      Legendre  &  $\begin{array}{c} {Le}_n(x) \\ \\ \alpha>-1 \\ \end{array}$                  &  $\displaystyle \frac{1}{2}^{\strut} \times \mathbb{I}_{[-1,1]}(x)$    &  $\begin{array}{ccc} a_n & = & \frac{\sqrt{(2n+1)(2n+3)}}{n+1} \\     b_n & = & 0 \\ c_n & = &  -\frac{ n \sqrt{2n+3} }{ (n+1)\sqrt{2n-1} } \end{array} $ \\
      \hline
      \begin{tabular}{c} Generalized \\ Laguerre \\ \end{tabular}  &  $L_n^{(\alpha)}(x)$        &  $\displaystyle \frac{x^{k-1}}{\Gamma(k)}~e^{-x} \mathbb{I}_{[0,+\infty[}(x)  $  &  $\begin{array}{ccc}  \omega_{n} & = & \left((n+1)(n+k+1) \right)^{-1/2} \\ a_n & = & \omega_{n} \\     b_n & = & -(2n+k+1)~\omega_{n} \\ c_n & = &  -\sqrt{(n+k)n}~\omega_{n} \end{array} $ \\

     % &  $\alpha>-1$                  &                   &      \\
      \hline
      Jacobi  &  $\begin{array}{c} J^{(\alpha,\beta)}_n(x) \\ \\ \\ \alpha,\beta>-1 \\ \end{array}$  &  $ \frac{(1-x)^{\alpha}(1+x)^{\beta}}{2^{\alpha + \beta + 1} B(\beta + 1, \alpha + 1)} \mathbb{I}_{[-1,1]}(x)$  &  $\begin{array}{ccc}  K_{1,n} & = & \frac{2n+\alpha + \beta + 3}{(n+1)(n+\alpha+1)(n+\beta+1)(n+\alpha+\beta+1)} \\ \\ K_{2,n} & = & \frac{1}{2} \sqrt{(2n + \alpha + \beta + 1) K_{1,n}} \\ \\a_n & = & K_{2,n}(2n+\alpha + \beta + 2)  \\   \\  b_n & = & K_{2,n}\frac{(\alpha - \beta)(\alpha + \beta)}{2n+\alpha+\beta} \\ \\ c_n & = & - \frac{2n+\alpha+\beta + 2}{2n+\alpha+\beta} \Big[(n+\alpha)(n+\beta) \\ & & \times (n+\alpha+\beta)n\frac{K_{1,n}}{2n+\alpha+\beta-1}\Big]^{1/2}  \end{array} $ \\
%$ \frac{2^{\alpha+\beta+1}}{2n+\alpha+\beta+1}  \frac{\Gamma(n+\alpha+1)^{\strut}\Gamma(n+\beta+1)}{n!~\Gamma(n+\alpha+\beta+1)}$     \\
     % &  $\alpha,\beta>-1$                  &                   &      \\
      \hline

    \end{tabular}

  \end{center}

% 
%   These particular polynomials are choosen because of the similarity between their weights and PDFs of some usual laws (Normal, Uniform, Gamma, etc.). \\

Furthermore, two families of orthonormal polynomials with respect to \emph{discrete} probability distributions are available in OpenTURNS, namely the so-called Charlier and Krawtchouk polynomials:

  \begin{center}

    \begin{tabular}{|c|c|c|c|}
      \hline
      Ortho. poly. &  $P_n(x)$ &  Probability mass function                                   &  Recurrence coefficients $(a_n,b_n,c_n)$   \\
      \hline\hline
      Charlier  &  $\begin{array}{c} Ch^{(\lambda)}_n(x) \\ \\ \lambda>0 \\ \end{array}$            &  $\begin{array}{c} \displaystyle{\frac{\lambda^k}{k!}~e^{-\lambda}} \\ \\ k=0,1,2,\dots \\ \end{array}$  &   $\begin{array}{ccc} a_n & = & - \frac{1}{\sqrt{\lambda (n+1)}} \\   \\  b_n & = & \frac{n+\lambda}{\sqrt{\lambda (n+1)}} \\ \\ c_n & = &  - \sqrt{1 - \frac{1}{n+1}} \end{array} $ \\
      \hline
      Krawtchouk$^{\dagger}$ & $\begin{array}{c} Kr^{(m,p)}_n(x) \\ \\ m \in \mathbb{N}~,~p \in [0,1] \\ \end{array}$ & $\begin{array}{c} \displaystyle{\binom{m}{k}p^k (1-p)^{m-k}} \\ \\ k=0,1,2,\dots \\ \end{array}$ & $\begin{array}{ccc} a_n & = & - \frac{1}{\sqrt{(n+1)(m-n)p(1-p)}} \\   \\  b_n & = & \frac{p(m-n)+n(1-p)}{\sqrt{(n+1)(m-n)p(1-p)}} \\ \\ c_n & = &  - \sqrt{(1 - \frac{1}{n+1})(1+\frac{1}{m-n})} \end{array} $ \\
      \hline

    \end{tabular}

  \end{center} 
{\footnotesize $^{\dagger}$ The Krawtchouk polynomials are only defined up to a degree $n$ equal to $m-1$. Indeed, for $n=m$, some factors of the denominators of the recurrence coefficients would be equal to zero.} \\

To sum up, the distribution types handled by Open TURNS are reported in the table below together with the associated families of orthonormal polynomials. \\

%\begin{table}[!ht] 
\begin{center}
%\caption{\small \em Correspondence between usual continuous distributions and families of orthogonal polynomials}
%\label{tab:3.1}
\begin{tabular}{ccc}
\hline
Distribution & Support & Polynomial \\
\hline
Normal $\mathcal{N}(0,1)$      &  $\mathbb{R}$ & Hermite \\
Uniform	$\mathcal{U}(-1,1)$	&   $[-1,1] $     & Legendre \\
Gamma $\mathcal{G}(k,1,0)$   & $(0,+\infty)$   & Laguerre \\
%Chebyshev &  $(-1,1)$       & Chebychev \\
Beta $\mathcal{B}(\alpha,\beta,-1,1)$     &  $(-1,1)$       & Jacobi \\
Poisson $\mbox{Pois}(\lambda)$ & $\mathbb{N}$ & Charlier \\
Binomial $\mbox{Binom}(m,p)$ & $\{0,\dots,m\}$ & Krawtchouk$^{\dagger}$ \\
\hline
\end{tabular}  
\end{center}
{\footnotesize $^{\dagger}$ It is recalled that the Krawtchouk polynomials are only defined up to a degree $n$ equal to $m-1$.} \\
\\

\textbf{Orthogonal polynomials with respect to arbitrary probability distributions} \vspace{2mm}

It is also possible to generate a family of orthonormal polynomials with respect to an arbitrary probability distribution $w(x)$. The well-known \emph{Gram-Schmidt} algorithm can be used to this end. Note that this algorithm gives a constructive proof of the existence of orthonormal bases.\\

However it is known to be numerically unstable, so alternative procedures are often used in practice. The two available algorithms in OpenTURNS so far are the so-called \emph{modified Gram-Schmidt algorithm} and the \emph{modified Chebyshev algorithm}. Another procedure based on Cholesky decomposition should be also available soon.
}
{-
}

\Methodology{
Within the global methodology, the orthogonal polynomial families are generated to construct the basis of the polynomial chaos expansion. The latter corresponds to a specific approach in order to tackle step C: ``Uncertainty propagation''. Then the various uncertainty propagation techniques may be applied at a negligible computational cost. The method requires to have fulfilled the two following steps:
\begin{itemize}
  \item step A1: identify of an input vector $\vect{X}$ of sources of uncertainties and an output variable of interest $\underline{Y} = h(\underline{X})$;
  \item step B: identify one of the proposed techniques to estimate a probabilistic model of the input vector $\vect{X}$.
  \end{itemize}
}
{-
}



