%Copyright (c)  2005-2010 EDF-EADS-PHIMECA.
%  Permission is granted to copy, distribute and/or modify this document
%  under the terms of the GNU Free Documentation License, Version 1.2
%  or any later version published by the Free Software Foundation;
%  with no Invariant Sections, no Front-Cover Texts, and no Back-Cover
%  Texts.  A copy of the license is included in the section entitled "GNU
%  Free Documentation License".
\renewcommand{\nomfichier}{docref_Cprime212_SobolIndices}
\renewcommand{\titrefiche}{Sensivity analysis using Sobol indices}

\Header

\MathematicalDescription{
\underline{\textbf{Goal}} \vspace{2mm}

This method is concerned with analysing the influence the random vector $\vect{X} = \left( X^1,\ldots,X^{n_X} \right)$ has on a random variable $Y^k$ which is being studied for uncertainty. Here we attempt to evaluate the part of variance of $Y^k$ due to the different components $X^i$. \vspace{2mm}

\underline{\textbf{Principle}} \vspace{2mm}

The estimators for the mean of  $ m_{Y^j} $ and the standard deviation $\sigma$ of $Y^k$ can be obtained from a first sample, as Sobol indices estimation requires two samples of the input variables : $(X^1,\ldots,X^{n_X})$, that is two sets of $N$ vectors of dimension $n_X$ $(x_{11}^{(1)},\ldots,x_{1n_X})^{(1)}$,\ldots,$(x_{N^1}^{(1)},\ldots,x_{Nn_X}^{(1)})$ and $(x_{11}^{(2)},\ldots,x_{1n_X})^{(2)}$,\ldots,$(x_{N^1}^{(2)},\ldots,x_{Nn_X}^{(2)})$

The estimation of sensivity indices for first order consists in estimating the quantity
\begin{displaymath}
V_i = Var \left[ \mathbb{E} \left[ Y^k \vert X_i \right] \right] = \mathbb{E} \left[ \mathbb{E} \left[ Y^k \vert X_i \right] ^2 \right]  - \mathbb{E} \left[ \mathbb{E} \left[ Y^k \vert X_i \right] \right] ^2 = U_i - \mathbb{E} \left[ Y^k \right] ^2
\end{displaymath}

Sobol proposes to estimate the quantity $U_i = \mathbb{E} \left[ \mathbb{E} \left[ Y^k \vert X_i \right] ^2 \right]$ by swaping every variables in the two samples except the variable $X_i$ between the two calls of the function :
\begin{displaymath}
\hat U_i = \frac{1}{N}\sum_{k=1}^N{ Y^k\left( x_{k1}^{(1)},\dots, x_{k(i-1)}^{(1)},x_{ki}^{(1)},x_{k(i+1)}^{(1)},\dots,x_{kn_X}^{(1)}\right) \times Y^k\left( x_{k1}^{(2)},\dots,x_{k(i-1)}^{(2)},x_{ki}^{(1)},x_{k(i+1)}^{(2)},\dots,x_{kn_X}^{(2)}\right)}
\end{displaymath}

Then the $n_X$ first order indices are estimated by
\begin{displaymath}
\hat S_i = \frac{\hat V_i}{\hat \sigma^2} = \frac{\hat U_i - m_{Y^k}^2}{\hat \sigma^2}
\end{displaymath}

For the second order, the two variables $X_i$ and $X_j$ are not swapped to estimate $U_{ij}$, and so on for higher orders, assuming that order $< n_X$. 
Then the $\binom {n_X}{2}$ second order indices are estimated by
\begin{displaymath}
\hat S_{ij} = \frac{\hat V_{ij}}{\hat \sigma^2} = \frac{\hat U_{ij} - m_{Y^k}^2 - \hat V_i - \hat V_j}{\hat \sigma^2}
\end{displaymath}

For the $n_X$ total order indices $T_i$, we only swap the variable $X_i$ between the two samples.
}
{
  --}

\Methodology{
The results produced as output of this method are the estimated relative (indices values belong to $\left[0; 1\right]$ ) variance contributions of subsets of variables  $\hat S_i, \hat S_{ij}, \hat T_i$ that the user may use to order the variables $X^i$ hierarchically. \\
This method of hierarchical ordering is particularly useful :
\begin{itemize}
\item	when the study of uncertainty is concerned with the central dispersion of the variable of interest $Y^j$ and not with its extreme values.
\item when we have no particular hypothesis on the model other than the independance of the input variables $X_i$.
\item when the size $N$ of both samples is high enough to provide a 'reasonable' picture of reality (the law of large numbers assures this method will show a $N^{-\frac{1}{2}}$ convergence order).
\end{itemize}
}
{


The following references provide an interesting bibliographic starting point to further study of the method described here:
\begin{itemize}
  \item Saltelli, A. (2002). ``Making best use of model evaluations to compute sensitivity indices", Computer Physics Communication, 145, 580-297 
\end{itemize}
}


\Example{
  --
}

