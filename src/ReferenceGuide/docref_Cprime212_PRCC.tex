% Copyright (c)  2005-2010 EDF-EADS-PHIMECA.
% Permission is granted to copy, distribute and/or modify this document
% under the terms of the GNU Free Documentation License, Version 1.2
% or any later version published by the Free Software Foundation;
% with no Invariant Sections, no Front-Cover Texts, and no Back-Cover
% Texts.  A copy of the license is included in the section entitled "GNU
% Free Documentation License".
\renewcommand{\nomfichier}{docref_Cprime212_PRCC}
\renewcommand{\titrefiche}{Uncertainty Ranking using Partial Rank Correlation Coefficients}

\Header

\MathematicalDescription{
  \underline{\textbf{Goal}} \vspace{2mm}

  This method is concerned with analyzing the influence the random vector $\vect{X} = \left( X^1,\ldots,X^{n_X} \right)$ has on the random variable $Y^j$ which is being studied for uncertainty. Here we attempt to measure monotonic relationships that exist between $Y^j$ and the different components $X^i$. \vspace{2mm}

  \underline{\textbf{Principle}} \vspace{2mm}

  The basic method of hierarchical ordering using Spearman's coefficients (see \otref{docref_Cprime212_Spearman}{Uncertainty Ranking using Spearman}) is concerned with the case where the variable $Y^j$ monotonically depends on $n_X$ variables $\left\{ X^1,\ldots,X^{n_X} \right\}$ but this can be misleading when statistical dependencies between the variables $X^i$ exist. In such a situation, the partial rank correlation coefficients can be more useful in ordering the uncertainty hierarchically: the partial rank correlation coefficients $\textrm{PRCC}_{X^i,Y^j}$  between the variables $Y^j$ and $X^i$ attempts to measure the residual influence of $X^i$ on $Y^j$ once influences from all other variables $X^j$ have been eliminated.

  The estimation for each partial rank correlation coefficient $\textrm{PRCC}_{X^i,Y^j}$  uses a set made up of $N$ values $\left\{ (y^j1,x_1^1,\ldots,x_1^{n_X}),\ldots,(y^jN,x_N^1,\ldots,x_N^{n_X}) \right\}$   of the vector $(Y^j,X^1,\ldots,X^{n_X})$. This requires the following three steps to be carried out:

  \begin{enumerate}
  \item Determine the effect of other variables $\left\{ X^j,\ j\neq i \right\}$ on $Y^j$ by linear regression (see \otref{docref_B234_LinearRegression}{Linear Regression}); when the values of variable $\left\{ X^j,\ j\neq i \right\}$ are known, the average forecast for the value of $Y^j$ is then available in the form of the equation:
    $$\widehat{Y^j} = \sum_{k \neq i,\ 1 \leq k \leq n_X} \widehat{a}_k X^k $$

  \item Determine the effect of other variables $\left\{ X^j,\ j\neq i \right\}$ on $X^i$ by linear regression; when the values of variable $\left\{ X^j,\ j\neq i \right\}$ are known, the average forecast for the value of $Y^j$ is then available in the form of the equation:
    $$\widehat{X}^i = \sum_{k \neq i,\ 1 \leq k \leq n_X} \widehat{b}_k X^k $$

  \item $\textrm{PRCC}_{X^i,Y^j}$ is then equal to the Spearman's correlation coefficient $\widehat{\rho}^S_{Y^j-\widehat{Y^j},X^i-\widehat{X}^i}$  estimated for the variables  $Y^j-\widehat{Y^j}$ and $X^i-\widehat{X}^i$ on the $N$-sample of simulations (see \otref{docref_B232_Spearman}{Spearman's Coefficient}).
  \end{enumerate}

  One can then class the $n_X$ variables $X^1,\ldots, X^{n_X}$ according to the absolute value of the partial rank correlation coefficients: the higher the value of $\left| \textrm{PRCC}_{X^i,Y^j} \right|$, the greater the impact the variable $X^i$ has on $Y^j$.
}
{
  -
}

\Methodology{
  After a propagation of uncertainty (step C) using \otref{docref_C221_MonteCarloStd}{Standard Monte Carlo} simulation, a hierarchy of sources of uncertainty can be obtained Partial Rank Correlation Coefficients. In fact, the $N$ simulations enable the pairs  $(y^j1, x^i_1)$, $(y^j2, x^i_2)$,\ldots, $(y^jN, x^i_N)$ to be generated, where:
  \begin{itemize}
  \item $\underline{X} = \left\{ X^1,\ldots,X^n \right\}$ describes the input vector specified in  step A "Specifying Criteria and the Case Study",
  \item $Y^j$ describes the final variable of interest or output variable defined in the same step.
  \end{itemize}

  The results produced as output of this method are partial rank correlation coefficients $\textrm{PRCC}_{X^i,Y^j}$, that the user may use, taking absolu
}
{
  This method of hierarchical ordering is particularly useful:
  \begin{itemize}
  \item when the study of uncertainty is concerned with the central dispersion of the variable of interest $Y^j$ and not with its extreme values,
  \item when the relationships between $Y^j$ and each of the components of $\underline{X}$ are monotonic relationships (so that Spearman's correlation coefficient can be interpreted),
  \item when the number $N$ of Monte-Carlo simulations is significantly higher than the number $n_X$ of input random variables (it is preferable to have $N/n_X$ at least greater than a factor of 10 so that the estimation of the $n_X$ partial rank correlation coefficients provides a reasonable picture of reality).
  \end{itemize}

  Readers interested in the assumptions made for multiple linear regression models and in the tests needed to validate these assumptions are referred to \otref{docref_B234_LinearRegression}{Linear Regression}.

  Other methods of uncertainty ranking can be applied after Monte-Carlo simulation, requiring a lesser number $N$ of simulations, are described in \otref{docref_Cprime212_Pearson}{Uncertainty Ranking using Pearson}, \otref{docref_Cprime212_Spearman}{Uncertainty ranking using Spearman}.

  The following references provide an interesting bibliographic starting point to further study of the method described here:
  \begin{itemize}
  \item Saltelli, A., Chan, K., Scott, M. (2000). "Sensitivity Analysis", John Wiley \& Sons publishers, Probability and Statistics series
  \item J.C. Helton, F.J. Davis (2003). "Latin Hypercube sampling and the propagation of uncertainty analyses of complex systems". Reliability Engineering and System Safety 81, p.23-69
  \item J.P.C. Kleijnen, J.C. Helton (1999). "Statistical analyses of scatterplots to identify factors in large-scale simulations, part 1 : review and comparison of techniques". Reliability Engineering and System Safety 65, p.147-185
  \end{itemize}}
