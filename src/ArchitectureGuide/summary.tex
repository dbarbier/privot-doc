        %Copyright (c)  2005-2010 EDF-EADS-PHIMECA.
        %  Permission is granted to copy, distribute and/or modify this document
        %  under the terms of the GNU Free Documentation License, Version 1.2
        %  or any later version published by the Free Software Foundation;
        %  with no Invariant Sections, no Front-Cover Texts, and no Back-Cover
        %  Texts.  A copy of the license is included in the section entitled "GNU
        %  Free Documentation License".
        \vspace{0.5in}
        \begin{center}
        \fontshape{sc} \large \centering
        \OT\ project: General architecture specifications \\
        \vspace{0.3in}
        \emph{\fontshape{sc} Abstract}
        \vspace{0.5in}
        \end{center}

        The \OT\ project is an open source development project. It aims at developing a computational platform designed to carry out industrial studies on uncertainty processing and risk analysis.

        This platform is intended to be released as an Open Source contribution to a wide audience whose technical skills are very diverse. Another goal of the project is to make the community of users ultimately responsible for the platform and its evolution by contributing to its maintenance and developing new functions.

        This architecture specifications document therefore serves two purposes:
        \begin{itemize}
        \item to provide the design principles that govern the platform, in order to guide the development teams in their development process;
        \item to inform external users about the platform's architecture and its design, in order to facilitate their first steps with the platform.
        \end{itemize}

        In the first section of this document, we will introduce the concepts that governed the construction of the platform. These concepts resulted from the requirements analysis carried out with the users and the developers, following the UML approach. The general functions of the platform allowed us to categorize the concepts by giving us a more synthetic and global view of its components.


        The second section details the technical choices that were retained for the platform's development.

