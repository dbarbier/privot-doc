%Copyright (c)  2005-2010 EDF-EADS-PHIMECA.
%  Permission is granted to copy, distribute and/or modify this document
%  under the terms of the GNU Free Documentation License, Version 1.2
%  or any later version published by the Free Software Foundation;
%  with no Invariant Sections, no Front-Cover Texts, and no Back-Cover
%  Texts.  A copy of the license is included in the section entitled "GNU
%  Free Documentation License".\section{Modeling package}

\subsection{Importing a numerical sample from a file}
\emph{Input data} \newline
\begin{tabular*}{\linewidth}{p{1in}p{2.5in}p{2in}}
  \textbf{ID} & \textbf{Type} & \textbf{Value} \\
  nameOfFile & FileName &
\end{tabular*}
\newline \newline
\emph{Output data} \newline
\begin{tabular*}{\linewidth}{p{1in}p{2.5in}p{2in}}
  \textbf{ID} & \textbf{Type} & \textbf{Value} \\
  sample & NumericalSample &
\end{tabular*}
\newline
\newline
\emph{Sequence of operations in the use case}
\begin{verbatim}
  std::ifstream fileStream( pathToFile ) ;
  CSVNumericalSampleFactory csvFact( fileStream ) ;
  NumericalSample sample = csvFact.getNumericalSample() ;
\end{verbatim}

\subsection{Multidimensional normal distribution inference based on a sample, using a parametric method following the method of moments}
\emph{Input data} \newline
\begin{tabular*}{\linewidth}{p{1in}p{2.5in}p{2in}}
  \textbf{ID} & \textbf{Type} & \textbf{Value} \\
  sample & NumericalSample &
\end{tabular*}
\newline \newline
\emph{Output data} \newline
\begin{tabular*}{\linewidth}{p{1in}p{2.5in}p{2in}}
  \textbf{ID} & \textbf{Type} & \textbf{Value} \\
  distribution & Normal &
\end{tabular*}
\newline
\newline
\emph{Sequence of operations in the use case}
\begin{verbatim}
  NormalFactory< MomentMethod > myFact ;
  Normal distribution = myFact.getDistribution( sample ) ;
\end{verbatim}
\emph{Method NormalFactory\textless MomentMethod \textgreater::getDistribution()}
\begin{verbatim}
  NumericalPoint mean = Normal::getMeanValue< MomentMethod >( sample ) ;
  CovarianceMatrix cov = Normal::getCovariance< MomentMethod >( sample ) ;
  return Normal( mean, cov ) ;
\end{verbatim}

\subsection{Multidimensional normal distribution inference based on a sample, using a parametric method following the method of maximum likelihood}
\emph{Input data} \newline
\begin{tabular*}{\linewidth}{p{1in}p{2.5in}p{2in}}
  \textbf{ID} & \textbf{Type} & \textbf{Value} \\
  sample & NumericalSample &
\end{tabular*}
\newline \newline
\emph{Output data} \newline
\begin{tabular*}{\linewidth}{p{1in}p{2.5in}p{2in}}
  \textbf{ID} & \textbf{Type} & \textbf{Value} \\
  distribution & Normal &
\end{tabular*}
\newline
\newline
\emph{Sequence of operations in the use case}
\begin{verbatim}
  NormalFactory< MaximumLikelihood > myFact ;
  Normal distribution = myFact.getDistribution( sample ) ;
\end{verbatim}
\emph{Method NormalFactory\textless MaximumLikelihood \textgreater::getDistribution()}
\begin{verbatim}
  NumericalPoint mean = Normal::getMeanValue< MaximumLikelihood >( sample ) ;
  CovarianceMatrix cov = Normal::getCovariance< MaximumLikelihood >( sample ) ;
  return Normal( mean, cov ) ;
\end{verbatim}

\subsection{Multidimensional exponential distribution inference based on a sample, using a parametric method following the method of moments}
\emph{Input data} \newline
\begin{tabular*}{\linewidth}{p{1in}p{2.5in}p{2in}}
  \textbf{ID} & \textbf{Type} & \textbf{Value} \\
  sample & NumericalSample &
\end{tabular*}
\newline \newline
\emph{Output data} \newline
\begin{tabular*}{\linewidth}{p{1in}p{2.5in}p{2in}}
  \textbf{ID} & \textbf{Type} & \textbf{Value} \\
  distribution & Exponential &
\end{tabular*}
\newline
\newline
\emph{Sequence of operations in the use case}
\begin{verbatim}
  ExponentialFactory< MomentMethod > myFact ;
  Exponential distribution = myFact.getDistribution( sample ) ;
\end{verbatim}
\emph{Method ExponentialFactory\textless MomentMethod \textgreater::getDistribution()}
\begin{verbatim}
  NumericalScalar lambda = Exponential::getLambda< MomentMethod >( sample ) ;
  return Exponential( lambda ) ;
\end{verbatim}

\subsection{Multidimensional exponential distribution inference based on a sample, using a parametric method following the method of maximum likelihood}
\emph{Input data} \newline
\begin{tabular*}{\linewidth}{p{1in}p{2.5in}p{2in}}
  \textbf{ID} & \textbf{Type} & \textbf{Value} \\
  sample & NumericalSample &
\end{tabular*}
\newline \newline
\emph{Output data} \newline
\begin{tabular*}{\linewidth}{p{1in}p{2.5in}p{2in}}
  \textbf{ID} & \textbf{Type} & \textbf{Value} \\
  distribution & Exponential &
\end{tabular*}
\newline
\newline
\emph{Sequence of operations in the use case}
\begin{verbatim}
  ExponentialFactory< MaximumLikelihood > myFact ;
  Exponential distribution = myFact.getDistribution( sample ) ;
\end{verbatim}
\emph{Method ExponentialFactory\textless MaximumLikelihood \textgreater::getDistribution()}
\begin{verbatim}
  NumericalScalar lambda = Exponential::getLambda< MaximumLikelihood >( sample ) ;
  return Exponential( lambda ) ;
\end{verbatim}

\subsection{Definition of a mixture distribution with constant parameters by the user}
\emph{Input data} \newline
\begin{tabular*}{\linewidth}{p{1in}p{2.5in}p{2in}}
  \textbf{ID} & \textbf{Type} & \textbf{Value} \\
  distribution1 & Normal & \\
  distribution1 & Exponential & \\
  w1 & NumericalScalar & 0.4 \\
  w2 & NumericalScalar & 0.6
\end{tabular*}
\newline \newline
\emph{Output data} \newline
\begin{tabular*}{\linewidth}{p{1in}p{2.5in}p{2in}}
  \textbf{ID} & \textbf{Type} & \textbf{Value} \\
  mix & Mixture & w1*distribution1 + w2*distribution2
\end{tabular*}
\newline
\newline
\emph{Sequence of operations in the use case}
\begin{verbatim}
  distribution1.setWeight( w1 ) ;
  distribution2.setWeight( w2 ) ;
  DistributionCollection myColl ;
  myColl.add( distribution1 ) ;
  myColl.add( distribution2 ) ;
  Mixture mix( myColl ) ;
\end{verbatim}

\subsection{Definition of a non-parametric mixture distribution using the kernel method}
\emph{Input data} \newline
\begin{tabular*}{\linewidth}{p{1in}p{2.5in}p{2in}}
  \textbf{ID} & \textbf{Type} & \textbf{Value} \\
  sample & NumericalSample & \\
  fact & DistributionFactory & \\
  h & CovarianceMatrix &
\end{tabular*}
\newline \newline
\emph{Output data} \newline
\begin{tabular*}{\linewidth}{p{1in}p{2.5in}p{2in}}
  \textbf{ID} & \textbf{Type} & \textbf{Value} \\
  mix & Mixture &
\end{tabular*}
\newline
\newline
\emph{Sequence of operations in the use case}
\begin{verbatim}
  Mixture mix = Mixture::getKernelRebuildDistribution( sample, fact, h ) ;
\end{verbatim}
\emph{Method Mixture::getKernelRebuildDistribution()}
\begin{verbatim}
  if ( sample.size() == 0 ) throw Exception ;
  Kernel K = fact.getKernel() ;
  if ( h.rank()�!= sample.dim() ) throw Exception ;
  DistributionCollection myColl ;
  NumericalPoint Xi ;
  for Xi in sample :
  UsualDistribution distribution = fact.getDistribution( Xi, h ) ;
  distribution.setWeight( 1. / sample.size() ) ;
  myColl.add( distribution ) ;
  end for
  Mixture mix( myColl ) ;
  return mix
\end{verbatim}

\subsection{Definition of a non-parametric mixture distribution using the kernel method}
\emph{Input data} \newline
\begin{tabular*}{\linewidth}{p{1in}p{2.5in}p{2in}}
  \textbf{ID} & \textbf{Type} & \textbf{Value} \\
  sample & NumericalSample & \\
  fact & DistributionFactory & \\
  algo & KernelAlgorithm &
\end{tabular*}
\newline \newline
\emph{Output data} \newline
\begin{tabular*}{\linewidth}{p{1in}p{2.5in}p{2in}}
  \textbf{ID} & \textbf{Type} & \textbf{Value} \\
  mix & Mixture &
\end{tabular*}
\newline \newline
\emph{Sequence of operations in the use case}
\begin{verbatim}
  Mixture mix = Mixture::getKernelRebuildDistribution( sample, fact, algo ) ;
\end{verbatim}
\emph{Method Mixture::getKernelRebuildDistribution()}
\begin{verbatim}
  if ( sample.size() == 0 ) throw Exception ;
  Kernel K = fact.getKernel() ;
  CovarianceMatrix h = algo( sample, K ) ;
  if ( h.dim()�!= sample.dim() ) throw Exception ;
  DistributionCollection myColl ;
  NumericalPoint Xi ;
  for Xi in sample :
  UsualDistribution distribution = fact.getDistribution( Xi, h ) ;
  distribution.setWeight( 1. / sample.size() ) ;
  myColl.add( distribution ) ;
  end for
  Mixture mix( myColl ) ;
  return mix
\end{verbatim}
\emph{Method KernelAlgorithm::operator()}
\begin{verbatim}
  [...]
  NumericalScalar normL2K = K.getL2Norm() ;
  CovarianceMatrix covK = K.getCovariance() ;
  [...]
\end{verbatim}

\subsection{Definition of a random vector based on constant parameters}
\emph{Input data} \newline
\begin{tabular*}{\linewidth}{p{1in}p{2.5in}p{2in}}
  \textbf{ID} & \textbf{Type} & \textbf{Value} \\
  distribution1 & Normal & dim = 1, mean = 0.1, sigma = 0.3 \\
  distribution2 & Exponential & dim = 1, lambda = 0.8 \\
  mat & SpearmanCorrelationMatrix & Non diagonal coefficients = 0.2
\end{tabular*}
\newline \newline
\emph{Output data} \newline
\begin{tabular*}{\linewidth}{p{1in}p{2.5in}p{2in}}
  \textbf{ID} & \textbf{Type} & \textbf{Value} \\
  P & ConcreteRandomVector & (distribution1, distribution2) correlated by mat
\end{tabular*}
\newline
\newline
\emph{Sequence of operations in the use case}
\begin{verbatim}
  DistributionCollection myColl ;
  myColl.add( distribution1 ) ;
  myColl.add( distribution2 ) ;
  if ( myColl.size()�!= mat.rank() ) throw Exception ;
  FictiveCorrelationMatrix< NormalFamily > fictiveMat = mat ;
  NormalCopula copula( fictiveMat ) ;
  ConcreteRandomVector P( AssemblyDistribution( myColl, copula ) ) ;
\end{verbatim}

\subsection{Definition of a random vector as a function of other random vectors}
\emph{Input data} \newline \newline
\begin{tabular*}{\linewidth}{p{1in}p{2.5in}p{2in}}
  \textbf{ID} & \textbf{Type} & \textbf{Value} \\
  X & RandomVector & \\
  F & NumericalMathFunction & \\
\end{tabular*}
\newline \newline
\emph{Output data} \newline \newline
\begin{tabular*}{\linewidth}{p{1in}p{2.5in}p{2in}}
  \textbf{ID} & \textbf{Type} & \textbf{Value} \\
  Y & CompositeRandomVector & Y = F(X)
\end{tabular*}
\newline
\newline
\emph{Sequence of operations in the use case}
\begin{verbatim}
  CompositeRandomVector Y( F, X ) ;
\end{verbatim}

\subsection{Definition of a failure event}
\emph{Input data} \newline
\begin{tabular*}{\linewidth}{p{1in}p{2.5in}p{2in}}
  \textbf{ID} & \textbf{Type} & \textbf{Value} \\
  X & RandomVector & dim = n \\
  F & NumericalMathFunction & inNumericalPointDim = n, \newline outNumericalPointDim = 1 \\
  threshold & NumericalScalar &
\end{tabular*}
\newline \newline
\emph{Output data} \newline
\begin{tabular*}{\linewidth}{p{1in}p{2.5in}p{2in}}
  \textbf{ID} & \textbf{Type} & \textbf{Value} \\
  evt & FailureEvent & F(X) < threshold
\end{tabular*}
\newline
\newline
\emph{Sequence of operations in the use case}
\begin{verbatim}
  CompositeRandomVector Y( F, X ) ;
  FailureEvent evt( Y, Less, threshold ) ;
\end{verbatim}

\section{Propagation package}

\subsection{Generation of a numerical sample according to a distribution}
\emph{Input data} \newline
\begin{tabular*}{\linewidth}{p{1in}p{2.5in}p{2in}}
  \textbf{ID} & \textbf{Type} & \textbf{Value} \\
  distribution & Distribution &
\end{tabular*}
\newline \newline
\emph{Output data} \newline
\begin{tabular*}{\linewidth}{p{1in}p{2.5in}p{2in}}
  \textbf{ID} & \textbf{Type} & \textbf{Value} \\
  sample & NumericalSample &
\end{tabular*}
\newline
\newline
\emph{Sequence of operations in the use case}
\begin{verbatim}
  UnsignedLong nbRealization = 100 ;
  NumericalSample sample = Distribution.getNumericalSample( nbRealization ) ;

\end{verbatim}
\emph{Method Distribution::getNumericalSample()}
\begin{verbatim}
  NumericalSample sample ;
  UnsignedLong i ;
  for i in 1..nbRealization :
  NumericalPoint point = realize() ;
  sample.add( point ) ;
  end for
  return sample ;
\end{verbatim}

\subsection{Generation of a numerical sample according to the distribution of a vector function of other random vectors}
\emph{Input data} \newline
\begin{tabular*}{\linewidth}{p{1in}p{2.5in}p{2in}}
  \textbf{ID} & \textbf{Type} & \textbf{Value} \\
  V & CompositeRandomVector & V = F(X)
\end{tabular*}
\newline \newline
\emph{Output data} \newline
\begin{tabular*}{\linewidth}{p{1in}p{2.5in}p{2in}}
  \textbf{ID} & \textbf{Type} & \textbf{Value} \\
  sample & NumericalSample &
\end{tabular*}
\newline
\newline
\emph{Sequence of operations in the use case}
\begin{verbatim}
  UnsignedLong nbRealization = 100 ;
  NumericalSample sample = V.getNumericalSample( nbRealization ) ;
\end{verbatim}
\emph{Method CompositeRandomVector::getNumericalSample()}
\begin{verbatim}
  RandomVector X = getAntecedent() ;
  NumericalMathFunction F = getNumericalMathFunction() ;
  NumericalSample sample ;
  UnsignedLong i ;
  for i in 1..nbRealization :
  NumericalPoint inPoint = X.realize() ;
  NumericalPoint outPoint = F( inPoint ) ;
  sample.add( outPoint ) ;
  end for
  return sample ;
\end{verbatim}

\subsection{Computation of a mean value with a first order SRSS}
\emph{Input data} \newline
\begin{tabular*}{\linewidth}{p{1in}p{2.5in}p{2in}}
  \textbf{ID} & \textbf{Type} & \textbf{Value} \\
  Y & CompositeRandomVector & Y = F(X)
\end{tabular*}
\newline \newline
\emph{Output data} \newline
\begin{tabular*}{\linewidth}{p{1in}p{2.5in}p{2in}}
  \textbf{ID} & \textbf{Type} & \textbf{Value} \\
  meanY & NumericalPoint &
\end{tabular*}
\newline
\newline
\emph{Sequence of operations in the use case}
\begin{verbatim}
  RandomVector X = Y.getAntecedent() ;
  NumericalMathFunction F = Y.getNumericalMathFunction() ;
  NumericalPoint meanX = X.getMeanValue() ;
  NumericalPoint meanY = F( meanX ) ;
\end{verbatim}

\subsection{Computation of a covariance with a first order SRSS}
\emph{Input data} \newline
\begin{tabular*}{\linewidth}{p{1in}p{2.5in}p{2in}}
  \textbf{ID} & \textbf{Type} & \textbf{Value} \\
  Y & CompositeRandomVector & Y = F(X)
\end{tabular*}
\newline \newline
\emph{Output data} \newline
\begin{tabular*}{\linewidth}{p{1in}p{2.5in}p{2in}}
  \textbf{ID} & \textbf{Type} & \textbf{Value} \\
  covY & CovarianceMatrix &
\end{tabular*}
\newline
\newline
\emph{Sequence of operations in the use case}
\begin{verbatim}
  RandomVector X = Y.getAntecedent() ;
  NumericalMathFunction F = Y.getNumericalMathFunction() ;
  NumericalPoint meanX = X.getMeanValue() ;
  Matrix gradF = F.gradient( meanX ) ;
  CovarianceMatrix covX = X.getCovariance() ;
  CovarianceMatrix covY = gradF.transpose() * covX * gradF ;
\end{verbatim}

\subsection{Computation of a threshold crossing probability using Monte Carlo (counting method)}
\emph{Input data} \newline
\begin{tabular*}{\linewidth}{p{1in}p{2.5in}p{2in}}
  \textbf{ID} & \textbf{Type} & \textbf{Value} \\
  evt & FailureEvent & \\
  mat & CovarianceMatrix & \\
\end{tabular*}
\newline \newline
\emph{Output data} \newline
\begin{tabular*}{\linewidth}{p{1in}p{2.5in}p{2in}}
  \textbf{ID} & \textbf{Type} & \textbf{Value} \\
  P & NumericalPoint & \\
\end{tabular*}
\newline
\newline
\emph{Sequence of operations in the use case}
\begin{verbatim}
  UnsignedLong nbMaxRealization = 1000 ;
  UncertaintyAlgorithm mc = MonteCarlo( evt, nbMaxRealization, mat ) ;

  Thread th( mc ) ;
  th.run() ;
  ThreadStatus status = th.query() ;

  NumericalPoint P ;

  if ( status.isDone() ) P = mc.result() ;
\end{verbatim}
\emph{Method MonteCarlo::run()}
\begin{verbatim}
  [...]
  evt.realize() ;
  [...]
\end{verbatim}

\subsection{Computation of a threshold crossing probability using FORM}
\emph{Input data} \newline
\begin{tabular*}{\linewidth}{p{1in}p{2.5in}p{2in}}
  \textbf{ID} & \textbf{Type} & \textbf{Value} \\
  evt & FailureEvent & \\
  Pstart & NumericalPoint & \\
  Pref & NumericalPoint &
\end{tabular*}
\newline \newline
\emph{Output data} \newline
\begin{tabular*}{\linewidth}{p{1in}p{2.5in}p{2in}}
  \textbf{ID} & \textbf{Type} & \textbf{Value} \\
  P & NumericalPoint &
\end{tabular*}
\newline
\newline
\emph{Sequence of operations in the use case}
\begin{verbatim}
  UnsignedLong nbMaxRealization = 1000 ;
  OptimizationAlgorithm optim = SQP( nbMaxRealization ) ;
  UncertaintyAlgorithm solv = FORM( evt, optim, Pref, Pstart ) ;

  Thread th( solv ) ;
  th.run() ;
  ThreadStatus status = th.query() ;

  NumericalPoint P ;

  if ( status.isDone() ) P = solv.result() ;
\end{verbatim}
\emph{Method FORM::run()}
\begin{verbatim}
  // Prerequisite : we get the random vector X which should be the only terminal
  //                leaf of evt's dependency tree

  DistributionCollection myColl = X.getMarginalDistributions() ;

  FictiveCorrelationMatrix< NormalFamily > M =
  X.getFictiveCorrelationMatrix() ;

  NumericalMathFunction T    = Nataf( myColl, M ) ;
  NumericalMathFunction Tinv = InverseNataf( myColl, M ) ;

  NumericalPoint stdPref   = T( Pref ) ;
  NumericalPoint stdPstart = T( Pstart ) ;

  NumericalMathFunction objectif = L2_2Norm( stdPref.dim() ) ;
  NumericalMathFunction XtoY = ... ;
  // we get this function from evt = transfer function from X to Y

  NumericalMathFunction constraint =
  ( XtoY(Tinv) - evt.getThreshold() ) / ( XtoY( Pref ) - evt.getThreshold() ) ;

  // An adequacy test needs to be carried out on the optimizer and the marginal distributions, which
  // must be either noncontinuous or derivable
  NumericalPoint Pconception = optim( objectif, constraint, Pstart ) ;

  NumericalPoint res = Normal( 0, 1 ).F(- Pconception.getNorm() ) ;
  // res must be stored in a Result object
\end{verbatim}

\subsection{Computation of a threshold crossing probability using Breitung's SORM}
\emph{Input data} \newline
\begin{tabular*}{\linewidth}{p{1in}p{2.5in}p{2in}}
  \textbf{ID} & \textbf{Type} & \textbf{Value} \\
  evt & FailureEvent & \\
  Pstart & NumericalPoint & \\
  Pref & NumericalPoint &
\end{tabular*}
\newline \newline
\emph{Output data} \newline
\begin{tabular*}{\linewidth}{p{1in}p{2.5in}p{2in}}
  \textbf{ID} & \textbf{Type} & \textbf{Value} \\
  P & NumericalPoint &
\end{tabular*}
\newline
\newline
\emph{Sequence of operations in the use case}
\begin{verbatim}
  UnsignedLong nbMaxRealization = 1000 ;
  OptimizationAlgorithm optim = SQP( nbMaxRealization ) ;
  UncertaintyAlgorithm solv = SORM_Breitung( evt, optim, Pref, Pstart ) ;

  Thread th( solv ) ;
  th.run() ;
  ThreadStatus status = th.query() ;

  NumericalPoint P ;

  if ( status.isDone() ) P = solv.result() ;
\end{verbatim}
\emph{Method SORM\_Breitung::run()}
\begin{verbatim}
  // Prerequisite : we get the random vector X which should be the only terminal
  //                leaf of evt's dependency tree

  DistributionCollection myColl = X.getMarginalDistributions() ;

  FictiveCorrelationMatrix< NormalFammilly > M =
  X.getFictiveCorrelationMatrix() ;

  NumericalMathFunction T    = Nataf( myColl, M ) ;
  NumericalMathFunction Tinv = InverseNataf( myColl, M ) ;

  NumericalPoint stdPref   = T( Pref ) ;
  NumericalPoint stdPstart = T( Pstart ) ;

  NumericalMathFunction objectif = L2_2Norm( stdPref.dim() ) ;
  NumericalMathFunction XtoY = ... ;
  // we get this function from evt = transfer function from X to Y
  NumericalMathFunction constraint =
  ( XtoY(Tinv) - evt.getThreshold() ) / ( XtoY( Pref ) - evt.getThreshold() ) ;

  // An adequacy test needs to be carried out on the optimizer and the marginal distributions, which
  // must be either noncontinuous or derivable
  NumericalPoint Pconception = optim( objectif, constraint, Pstart ) ;

  NumericalPoint gradient  = constraint.gradient( Pconception ) ;
  NumericalPoint hessian = constraint.hessian( Pconception ) ;

  NumericalPoint curvature = CurvatureComputation( gradient, hessienne ) ;

  NumericalPoint res =
  Normal( 0, 1 ).F(- Pconception.getNorm() ) * PRODUIT(0:n)( 1 / sqrt(1-curvature[i]) ) ;
  // res must be stored in a Result object
\end{verbatim}

\section{Prioritization package}

\subsection{Regression-based prioritization}
\emph{Input data} \newline
\begin{tabular*}{\linewidth}{p{1in}p{2.5in}p{2in}}
  \textbf{ID} & \textbf{Type} & \textbf{Value} \\
  sampleY & NumericalSample &
\end{tabular*}
\newline \newline
\emph{Output data} \newline
\begin{tabular*}{\linewidth}{p{1in}p{2.5in}p{2in}}
  \textbf{ID} & \textbf{Type} & \textbf{Value} \\
  C & NumericalPoint &
\end{tabular*}
\newline
\newline
\emph{Sequence of operations in the use case}
\begin{verbatim}
  Regression myReg( sampleY ) ;
  NumericalPoint C = myReg.ranking() ;
\end{verbatim}
\emph{Method Regression::ranking()}
\begin{verbatim}
  NumericalSample sampleX = sampleY.getAntecedent() ;
  NumericalPoint C = OT::Base::R::regression_coeff( sampleX, sampleY) ;
  return C ;
\end{verbatim}
